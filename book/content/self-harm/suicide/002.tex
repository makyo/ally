It's not really so much that I have the need to write about what happened, even, as that, after something of such import, I feel the need to expose myself through writing, to force ideas out into the open whether or not they actually have anything to do with what's going on.

\begin{quote}
It goes beyond a desire. It becomes a necessity.
\end{quote}

Creativity, it seems, is one of those things where, the more you put it to use, the more you \emph{must} use it.

\begin{quote}
After a certain point, it forces itself upon you. Hits you like a ton of bricks.
\end{quote}

Yes.

I toyed with how to write about something like this for a few months after it happened before hammering out a five thousand word essay.

\begin{quote}
You planned on an additional ten thousand.
\end{quote}

In this case, after all, I felt the need to actually write about what really happened. I tried the whole ``write about something else'' thing and it didn't work; it didn't relieve that pressure within myself that needed to be released.

\begin{quote}
You tried venting little bits of it here and there on twitter, on Facebook.
\end{quote}

It didn't work. It kept the pressure from becoming unbearable, perhaps, but only for a few days. After that, the weight of it --- of how easy it was, of how quickly I snapped to, of how badly I could have fucked up --- became too intense to ignore once again.

So.

I tried to kill myself on March 21st, 2012. It was, as the epigram said, not a big deal; it was just my big deal.
