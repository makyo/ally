\noindent On March 21st, 2012, I tried to kill myself.

It's amazing how such a simple statement of fact reflects, months of strange tension, slow recovery, and a whole lot of trying to understand what really happened. It's not a comfortable thing for anyone to discuss, but it's one of those things I need to discuss, need to get off my chest. A little too much of what makes life meaningful for me now is wrapped up in that one night.

\begin{ally}
Even now?
\end{ally}
Even now.

\begin{ally}
You wrote that disclaimer four months after the attempt itself. You copied it from some notes from back then. You even kept the Steve Eisman quote.
\end{ally}
Yes. Nostalgia, remember?

\begin{ally}
Are you nostalgic for those weighty months after you tried to kill yourself?
\end{ally}
If Matthew died on September of that year, then he was sick long before. This was part of his long, slow death rattle.

Perhaps it's not totally accurate to say that I'm nostalgic for that time in particular, but I suppose I am nostalgic for the sense of change that permeated the air around me then. Something big was happening. Something terrible and wonderful.

\begin{ally}
And you got to witness it from the inside.
\end{ally}
Yes. I got to watch the agonal breathing that went on for far too long. I got to see his eyes widen in terror. I got up to fetch the cold compress and came back to a quiet room.

I'm not nostalgic for that pain, no. I'm nostalgic for the fact that I am who I am because I went through that. I'm nostalgic for what it came to symbolize. I'm nostalgic for its part in Madison's birth.
\newpage

It's not really so much that I have the need to write about what happened, even, as that, after something of such import, I feel the need to expose myself through writing, to force ideas out into the open whether or not they actually have anything to do with what's going on.

\begin{ally}
It goes beyond a desire. It becomes a necessity.
\end{ally}
Creativity, it seems, is one of those things where, the more you put it to use, the more you \emph{must} use it.

\begin{ally}
After a certain point, it forces itself upon you. Hits you like a ton of bricks.
\end{ally}
Yes.

I toyed with how to write about something like this for a few months after it happened before hammering out a five thousand word essay.

\begin{ally}
You planned on an additional ten thousand.
\end{ally}
In this case, after all, I felt the need to actually write about what really happened. I tried the whole ``write about something else'' thing and it didn't work; it didn't relieve that pressure within myself that needed to be released.

\begin{ally}
You tried venting little bits of it here and there on twitter, on Facebook.
\end{ally}
It didn't work. It kept the pressure from becoming unbearable, perhaps, but only for a few days. After that, the weight of it --- of how easy it was, of how quickly I snapped to, of how badly I could have fucked up --- became too intense to ignore once again.

So.

I tried to kill myself on March 21st, 2012. It was, as the epigram said, not a big deal; it was just my big deal.
\newpage

I'll be honest, I stole the concept of \emph{thisness}, the phrase, ``See, it is doing \emph{this} now'' from a science fiction book.

\begin{ally}
I honestly expected nothing less.
\end{ally}
I suspect that Neal Stephenson got it from elsewhere, too. I think he basically admits as much, in that he was talking about Husserl at the time. Still, it's proven handy.

The biggest thing I've taken away from therapy has been an increased sense of self awareness. The ability to say ``ah, I am doing \emph{this} now.'' It is the \emph{thisness} of myself. The \emph{thisness} of my mind. I am able to see myself dipping down into the well of depression. I'm able to see the hypomania that starts to creep into my mind, into my life, and forces me to bury myself in projects.

\begin{ally}
Like this one.
\end{ally}
Yes. That's why I'm moving so much more slowly with it now. I have slid off the pedestal and into the slow morass of depression. I can feel it coloring my life with anhedonia.

\begin{ally}
Not coloring, no. Sapping the color. Not even black-and-white, but an absence. A missingness.
\end{ally}
Yes.

\begin{ally}
But you didn't have this back then. You didn't have the thisness of mental health maturity. You weren't able to see what was going on.
\end{ally}
Yes. I was having panic attacks from day to day. I was caught up in those rising swells of anxiety that would lead to me freezing. Occaisonally, I would have to stop in a rest area on my way home just to calm down enough to continue driving.

\begin{ally}
That's when you started your habit of asking others to tell you good things.
\end{ally}
``Tell me good things,'' I'd say, and I'd get a slew of responses. Many were along the lines of ``You! You're good!''

\begin{ally}
But you weren't able to internalize that.
\end{ally}
Not then, no. Not back then, and especially not during panic attacks.

Some of them would be ``A good thing is that I had a good day at work.'' That was what I needed to hear. I needed to hear that others were having a good day. I needed to hear that others were \emph{capable} of having good days. I needed to hear that good days were possible, and that I might be in line for one, myself.

My boss picked up on that, as well as so many other things. ``You're so angry,'' he said. ``You're scaring the project manager at times.'' So he sent me to a psychiatrist.

\begin{ally}
He handed you a check for a thousand dollars and said, ``I know it's expensive, so hopefully this helps you out.'' You never cashed it.
\end{ally}
He sent me to his doctor, doctor Johnston. And he was a pretty good at what he did.

\begin{ally}
You fired him when, after you asked him for a letter of support for hormones, he said, ``I don't know enough about that, and you don't even want to know my feelings about it.''
\end{ally}
Well, yes, but there's no denying the utility of what he gave me.

\begin{ally}
He gave you exactly what you brought to the table, except with context.
\end{ally}
Yes. I brought my anxiety to the table, and he taught me about it. He spoke my words back to me and added footnotes. He wrote in the margins of my speech and I learned. I learned coping mechanisms and breathing techniques. I got my prescriptions.

\begin{ally}
You brought your anxiety, but not your depression. You thought you just had anxiety, not any mood disorders. Boys didn't have moods, right? You were just anxious. Despite years of experience, you didn't tell him about how you felt.
\end{ally}
No, and there's the problem.
\newpage

When I first started therapy, I did what I thought was the right thing by bringing an open mind. It wasn't enough for me to seek help, I had to be told what was wrong with me. So anxious was I to not diagnose myself, I had to let someone do the work to pry the symptoms from me.

I didn't tell Dr Johnston that I was feeling bad. I told him my boss told me I was angry. I didn't tell him that I was depressed, I told him that James was worried about how anxious I was.

\begin{ally}
And so you got treated for anxiety.
\end{ally}
And so I got treated for anxiety. I was given clonazepam to take daily and lorazepam for breakthrough anxiety.

\begin{ally}
You have always had issues with control. You always needed to be on top of a situation.
\end{ally}
And all my deepest fears, all of those things I would ruminate on during a panic attack, would surround the fact that I wasn't in control of a situation, yes. It made sense to treat the anxiety.

\begin{ally}
It hurt.
\end{ally}
Yes. I was given a long-acting anxiolytic and a more powerful, shorter-lasting one for breakthrough anxiety. When things hurt, they calmed and soothed the pain. They removed it.

\begin{ally}
They removed a lot more than just the pain of panic.
\end{ally}
Yes.
\newpage

The problem of working with clients on a task with a specified end-goal, one that is finished and about which you can say, ``ah, it does \emph{this} now'', is that when the project is done, there is nothing left.

\begin{ally}
This is a problem with any task. This is a grander problem.
\end{ally}
Yes, even with self-appointed tasks, even with tasks at a non job-shop. It happened just recently, too. I finished my time at IA. I got home from visiting Barac. I got the contract signed at NV.

If you hit a deadline and succeed, or if you have some work travel, or if you get home from a vacation, suddenly there's this empty bit of your future where there used to be this thing. There's just a void there. A sudden lack of weight.

\begin{ally}
And so, back then, you finished the release at work and also finished the office move in one fell swoop, and went home.
\end{ally}
I went home and took my meds like a good girl, and then proceeded to dissociate right through the evening.

Dissociation is a hell of a drug.

\begin{ally}
It's a dreamy thing. It's a soft thing. It's a cottony thing. It's a muffled thing. It's watching your hands move. It's watching yourself breathe. It's feeling the air move in and out of you with a distant, slightly confused detachment. It's ``ah, it does \textbf{this} now'', except saying that about some strange machine which is not yourself.
\end{ally}
I watched myself sit down in my chair. I watched myself turn on \emph{Babylon 5}. I watched myself mow through two glasses of gin.

\begin{ally}
You watched yourself with a metaphysical quirk of the eyebrow as you reached forward, grabbed the box of X-acto wood-carving tools --- purchased, doubtless, for some long forgotten project --- and flipped it open. You watched numbly as you slashed open the inside of your arm. There was a moment where you marveled at how long it took for the blood to well up, where you could see the white of subcutaneous fat.
\end{ally}
And then the pain snapped me to.
\newpage

Okay, I lied. Just a little bit.

\begin{ally}
Yes. You didn't dissociate through the entire thing. There was no small part of that scene that was horribly, terrfyingly intentional.
\end{ally}
What really woke me up was watching this person-who-was-me somehow go into `fuck it' mode and tear the shit out of his right arm from one end to the other with a very sharp, very new razor blade.

It was like the rush of coming to your senses after a nightmare, the pulling forward and the re-anchoring, the flood of adrenaline in preparation for flight.

It wasn't necessarily the cut that woke me. It was the second or so before when I entered that `fuck it' mode, and I was too slow, too confused and frightened to stop this person-who-was-me from pulling the ultimate embarrassing act: trying to commit suicide while watching a dumb '90s science fiction show.

\begin{ally}
It was a slow awakening. You weren't just too slow, you were not fully awake yet. The dream of dissociation was still clinging, gauzy, to you.
\end{ally}
V I do not know which to prefer, The beauty of inflections Or the beauty of innuendoes, The blackbird whistling Or just after.

I can remember it so clearly.

\begin{ally}
You can remember it because you still live it.
\end{ally}
Yes. I still feel that slide into someone-else-ness, and then the snap back when drawn back into self-ness. Back into here and now.

\begin{ally}
You felt that last night.
\end{ally}
Yes.

\begin{ally}
You felt that slide into dissociation, felt the folding blade click into place with a vague sense of surprise, then jolted as it drew across your leg.
\end{ally}
Yes.

\begin{ally}
You felt that same jolt of humiliation and pain and anger and fear.
\end{ally}
Yes.

\begin{ally}
Especially this time. You cut too deep. Your usual superficial-yet-still-painful scratch had turned into something of a flay.
\end{ally}
Yes.

\begin{ally}
You needed twelve stitches. You lied and said you dropped your knife while cleaning it.
\end{ally}
Yes.

\begin{ally}
Are you writing about this now because you were, on some subconscious level, working up to this most recent little climax?
\end{ally}
I really don't know.

\begin{ally}
Tell me what happened after.
\end{ally}
I started whispering James' name--

\begin{ally}
Both times?
\end{ally}
Both times. I started whispering his name, then eventually swallowed the miniscule bit of pride I had left and called out loud enough to wake him up. ``Can you come help me?'' I asked. It took asking two more times before he got up. I found out later that he thought I had made a mess and just wanted help cleaning up, thinking that I should just clean up my own messes. A good point, that.

Though the rest of the night in March is still sort of a blur --- I hadn't totally gotten out of the state that I was in, just woken up enough to engage with the mechanics --- I do remember James helping me to clean and bandage my arm as we sat on the floor of the bathroom, the dog occasionally wandering in and out. The whole time, I was still sobbing, blubbering out, ``I don't want to leave you, I don't want to leave Zephyr, I don't know why I did that, I'm sorry'' over and over again.
\newpage

I'm so tired.

\begin{ally}
I know.
\end{ally}
Can I let Matthew tell the story? Can I put his words here, and can I catch up on the sleep I missed while in the ER? Can I feel better before I write again?

\begin{ally}
Yes, but don't make a habit of it.
\end{ally}
Okay.

The last thing I did before going to bed that night was to send an email to work saying that I would be in later in the day due to an ``emergency appointment'' in the morning. I certainly couldn't tell them what had actually happened, but I had so thoroughly exhausted myself and still felt so bad that I decided sleeping in would help me out quite a bit.

I wound up at the office around eleven in the morning, and sat down, feeling tired, worn thin, and still traumatized from the fact that I had apparently acted out something I had thought was just one of those persistent negative thoughts that won't go away, one with no grounding in reality. Within minutes, I received a message from my boss informing me that my attitude in the last few weeks was not acceptable. I had been irritable and angry, to the point where my supervisors felt as though they had to word things so that I wouldn't get upset.

I was stuck in a weird situation, here. On the one hand, my boss was totally right and I really did need to take a look at how I was interacting with others at work, but on the other hand, I wasn't in a place to do anything about it at the time, and I certainly didn't feel as though I could talk to my boss about what had happened in order to save the conversation for another time.

I did my best to accept it and trudge through the rest of the day. The plan that was in place before was to follow a friend up to Blackhawk for a free night at a casino hotel that he had available. It seemed like getting out of town might actually help, and it also meant that my workday was significantly shorter than it would've been otherwise.

The drive after work was calming, and I actually got to the point where I felt as though the night out would be a good change of pace to keep me from going too crazy.

And you know? The evening really did help. It was a lot of fun spending \$20 on roulette and walking away with \$60, it was fun eating a ridiculous amount of crab legs, and it was\ldots{}well, it was mortifying, watching some of saddest people I've ever seen in my life sit, lost, in front of their slot machines.

We had planned on going hot-tubbing, but, as became clear when I took off my shirt back at the room and exposed the rather bulky bandage along the underside of my arm, that was pretty much out of the question, so we mostly just sat around talking, and, in my case, trying to feel better about the whole thing.

I was fine until it was time for bed. As is usually the case, the stillness is when I get the worst, in terms of anxiety. That's when it's easiest for my mind to wander, fixate on a subject, and loop over it in all the worst ways for the longest time. The problems started when sleep didn't come.

And didn't come.

And still didn't come.

After a time, I suppose I just lost it. I got up and started pacing the room, walking from the bathroom to the window and back again, clenching and unclenching my hands before I let loose a ``Jesus fucking Christ!''

I locked myself in the bathroom and broke down again.

Both James and Karl checked in on me throughout the next few hours, but it was mostly spent huddled up on the cold tile of the floor feeling awful about both myself and what I'd done --- that it had any effect on those around me was just starting to hit home. I will not lie that, several times throughout the night, I wished that I had succeeded in order to not be going through what I was going through at the time. I simply couldn't stand what I'd done.
\newpage

Things are totally out of control now.

--- Maddy, whose tail is behind her (@drab\_makyo) March 23, 2012

On meds for anxiety now, but that seems to have just let loose something terrible. Tried to kill myself Wednesday night, spent all tonight--

--- Maddy, whose tail is behind her (@drab\_makyo) March 23, 2012

--obsessing about it, woke up Karl and James, then felt guilty and upset about it.

--- Maddy, whose tail is behind her (@drab\_makyo) March 23, 2012

It's not even really about anything, I'm just messed up, I guess.

--- Maddy, whose tail is behind her (@drab\_makyo) March 23, 2012

Days are spent in a surreality, both happy and unreasonably angry.

--- Maddy, whose tail is behind her (@drab\_makyo) March 23, 2012

I'm sorry you'll all wake up to a bunch of Matt freaking out, but I'm stuck :S

--- Maddy, whose tail is behind her (@drab\_makyo) March 23, 2012

\begin{ally}
Where's your tweet from this time?
\end{ally}
As someone who went to the ER last night and got 12 stitches only to find out that insurance ended on the 30th and I haven't received my COBRA paperwork yet so we'll see how fucked I am financially: mood. https://t.co/sil5Yf2617

--- Maddy, whose tail is behind her (@drab\_makyo) October 10, 2019

I'm okay. Just tired.

--- Maddy, whose tail is behind her (@drab\_makyo) October 10, 2019
\newpage

\begin{ally}
You posted about those things publicly, but not privately, not one-on-one.
\end{ally}
I know. I've been called on it before.

\begin{ally}
And since. Why?
\end{ally}
I suppose I need to be seen, but am not brave enough for it to be a conversation. I need to be seen but can't quite ask for help. I've promised everyone that I'm working on it, but the truth is, I don't know how I'd even begin to.

\begin{ally}
Is that what you're doing now?
\end{ally}
Perhaps.
\newpage

\begin{ally}
So, what happened after?
\end{ally}
There was an inpouring of confused and sympathetic replies. Some were simply along the lines of ``You are loved'' and ``There are friends all around the world thinking of you'', while others were more focused on ``But this is all so sudden'' and ``You didn't say anything was wrong.'' Someone mentioned a correlation between my medication and dissociation as mentioned.

You have to understand that, at the time, I was embedded in a casino an hour and a half's drive from work. Casinos are horrifying places to be, even at the best of time. Desperation and sweat. Cigarette smoke and free drinks. The dead eyes of those who must pull the lever, who must pull the lever, who must pull the lever.

So here I was, with an hour's sleep under my belt, seeing people still gambling, still hurting, answering texts and calls from my boss, and a wave of numb dissociation once more washes over me.

I drove numbly down to work

\begin{ally}
You sat in your car in front of the building, talking on the phone with Ash. You somehow made it to your desk, though there was no memory of moving from the car.
\end{ally}
``Come with me,'' Kevin said, and beckoned me out of the office.

``Sorry about all of the freaking out,'' I mumbled, once we were out of earshot. ``I think it has to do with the medication, I'm going to call Dr.-''

\begin{ally}
The office next to
\end{ally}
``I need you to tell me what your plan is,'' my boss asked.

``Plan?''

``Plan to kill yourself.''

``I\ldots{}don't have a plan, I don't know why,'' I managed.

``Well, you need to tell me if anything like that happens again.''
\newpage

I can't do this.

\begin{ally}
Of course you can.
\end{ally}
I can't. I can't talk about this. I thought I was done with it. I thought it would be easy enough to go back over this, but I can't.

\begin{ally}
Tell me why not, then?
\end{ally}
I just\ldots{}I just remember how easy it was to fuck up so badly. I did that a few weeks ago, too. I fucked up real bad, and now I'm stuck with the consequences, all the mechanics of tending to a wound, and all I can think about is how easy it was. It was so easy. It was so easy.

\begin{ally}
Perhaps that's part of what snaps you back into place. Perhaps that's part of what keeps you from following through. The mechanics of wound care. The laser focus on not doing it. Perhaps that's what saves you, in the end: the realization that you have a body leads to the realization that you're alive, confronting mortality leads to the acceptance of life.
\end{ally}
It's harder to \emph{not}.
\newpage

I can't do this anymore.

\begin{ally}
This topic, or this project?
\end{ally}
I don't know.
\newpage

Let's talk about something else. Please.

\begin{ally}
One more question, and then we can.
\end{ally}
Okay.

\begin{ally}
How far have you come since then?
\end{ally}
I think a long ways.

\begin{ally}
You think?
\end{ally}
Well, every time I think I've come a long ways, I do something horribly stupid again. Every time I think I'm over all this, I tear at myself. Every time I think I'm getting good at talking about my mental health, I wind up in this pit where I have to destroy myself, to make it physically evident that I'm unwell in some invisible way. I always have. I tried to blind myself when I was ten, remember? I tried to lose a finger, a leg. I cut. I burned.

\begin{ally}
Is it about proving that you're unwell?
\end{ally}
How could I possibly prove that I'm too depressed to be around others? How could I possibly prove that I'm too anxious and sad and upset and numb to look at a chat lest the read-receipts show that I am okay enough to exist? How could I possibly prove such a thing when you look at me and see me hale and intact?

\begin{ally}
You are talking about self harm. I asked about suicide. How far have you come since your first suicide attempt.
\end{ally}
I still think about it on the daily. I still obsess over it. Now I'm more likely to just go to bed, though.

\begin{ally}
Is it so simple?
\end{ally}
No, of course not, but look, I'm thirty-three. I'm too old for it to be tragic, too young for it to be a midlife crisis, too healthy for it to be understandable, too sick for it to be a surprise. It would just be sad and weird, not to mention mean to those in my life. I've got that perspective now. I'm thirty-three, I've made it this far, I've worked this hard, and I can at least understand that.

It's easier to just go to bed and wait it out, or maybe just get out the soldering iron for a bit, because yeah, it still blows, but at least now I know it'll pass, and five months down the line, I can do the same dance all over again.

\begin{ally}
That seems rather fatalistic.
\end{ally}
I'm tired. I don't even know what to do about this anymore, other than wait it out. My doctor got mad at me for saying I've come to terms with feeling like shit for a few weeks every five months or so, that that's just my life forever now.

I've just never seen any evidence to the contrary.
\newpage
