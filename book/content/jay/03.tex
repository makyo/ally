Our punishment --- my step-siblings and I --- was time-out. Jay had an old church pew rescued from some church in New Mexico that he'd painted a grayish sky blue. ``Go sit on the bench,'' he'd tell us. ``Half an hour.''

\begin{quote}
You measured it with your fingers. You'd judge the width of the plank you sat on by pinching it. Three inches? Four? You'd lay your length on it and count how many Matts it took from one end to another.
\end{quote}

It was a perfect punishment. My dad lamented once that he couldn't send me to my room as a punishment because I'd happily sit in there for hours on end.

\begin{quote}
You'd be away from him. That's a reward.
\end{quote}

I hadn't thought of it that way.

The bench, though, was perfect. It faced a dining table, and across from that, the computer which was kept powered off. No reading. No talking. No moving from the bench. If more than one of us were in trouble at the same time, no looking at each other; we sat on opposite ends.

When he started taking up martial arts, he brought Zach and I with him. He thought\ldots{}well, I don't know what he thought. That it would make us men? That it would teach us to defend ourselves?

In the end, it turned into its own means of punishment. He'd grapple with us. He'd grab me by the front of my shirt and slam me into the cabinets. It was just play, right? Just studying up for the next session, right?

\begin{quote}
Maybe he wanted to hit you from the start. Maybe that's why he got into karate.
\end{quote}

I think part of him did, yeah. I think part of him would rather our punishments would make him feel better at the same time. It took me a while to think of it that way, though. It took me a while to think of it as abuse.

\begin{quote}
It took you no longer being afraid of him. It took you telling your mom that, no, you wouldn't go see him at his feed store in Loveland. It took you until then to think of it as anything other than you not being man enough.
\end{quote}

I'm still afraid of him. Maybe it just took me admitting that.
