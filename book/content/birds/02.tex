\emph{February 13, 2014:}

I wonder if the snow loves the trees and fields, that it kisses them so gently? And then it covers them up snug, you know, with a white quilt, and perhaps it says, ``Go to sleep, darlings, till the summer comes again.''

- Lewis Carroll

I've mentioned ritual before, but I think that's tied into the larger feeing of portentousness. Ritual is one way to sate that sense of intense meaning surrounding an act or an object.

A goose is dumb. A thousand geese darkening the horizon is a portent. Mindless honking, individually directionless, collectively unstoppable

--- Makyo (@drab\_makyo) February 12, 2014

Any little thing can carry meaning for one person far outweighing what it might mean to others. Something about flocks of geese terrifies me. It's not a logical fear, it's a sense of foreboding. It's not the geese themselves, it's the concept of geese, the lack of any ritual to solve the problem of geese.

A goose is tasty. Geese taste like horror. Acrid tang of ill omens \emph{froth}

--- Makyo (@drab\_makyo) February 12, 2014

It's dumb. Geese are dumb. There's no reason I should feel any sort of emotion at all surrounding geese, but I do.

Why are geese so portentous? Why do they cause anxiety? Did I take my meds this morning?

--- Makyo (@drab\_makyo) February 12, 2014

Ritual is like that. There is some level of meaning that's inexpressible except if you can find a way to come at it from the side. Use words like `portent'. Describe it as an odor, a sense, a mystery. Ritual and sensation are wily and wary critters that want nothing less than to be identified, pointed out, made plain. You're supposed to just go along with the ritual and accept the portentous as fact.
