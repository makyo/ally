Here is the difference betwixt the poet and the mystic, that the last nails a symbol to one sense, which was a true sense for a moment, but soon becomes old and false. For all symbols are fluxional; all language is vehicular and transitive, and is good, as ferries and horses are, for conveyance, not as farms and houses are, for homestead. Mysticism consists in the mistake of an accidental and individual symbol for an universal one.

\begin{quote}
Pretty.
\end{quote}

I didn't write it.

\begin{quote}
I know.
\end{quote}

I scramble through great heaps of words and sounds to try and at least pin some of them to fleeting symbols. Maybe then I'll be able to learn to see more of the accidental and individual symbols.

\begin{quote}
Too many words, too many sounds.
\end{quote}

Yes.

\begin{quote}
You wrote four pieces about the winds coming down over the foothills near Boulder (for, of all things, wind quartet), just to try and capture one ecstatic experience.
\end{quote}

I like those. I like the result.

\begin{quote}
You like the first two, most of all. They remind you of how hollow you felt, how you could feel the wind blow through you, vibrating your soul like the pipe of an organ, exciting you to ever higher harmonics.
\end{quote}

Yes.

\begin{quote}
But then you kept writing.
\end{quote}

Yeah. I make a terrible poet.

\begin{quote}
You make a terrible mystic. Your poetry's just okay.
\end{quote}
