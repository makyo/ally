\label{liminal}
\begin{paracol}{2}
  \begin{leftcolumn}

\begin{ally}
If Matthew died on September 6th, 2012, was Madison born then?
\end{ally}
No, I don't think so. Madison was born some years later. Maybe at some point in 2014. The years in between were a sort of liminal time.

\begin{ally}
You found yourself in a place between.
\end{ally}
I did. There was this time in my life when I was figuring out gender. I was figuring out poly. I was figuring out working. I was figuring out not being at school and moving away from music and learning to write and all the interstices of alcoholism. Those little nooks and crannies you never know about until you start drinking in earnest.

It was like a second period of growing up. Something more refined than a rebirth. Something less grand. Something subtler.

\begin{ally}
You also learned the term `hendiatris'.
\end{ally}
I have a style, alright?

\begin{ally}
Right.
\end{ally}
It's the time when I started {[}a{]}{[}s{]}, the time when I started to look at my life in earnest, to give thought to the fact that one might actually enjoy things, have opinions. It was the time I started to let go of irony, bit by bit.

\begin{ally}
It was the time you started to own yourself.
\end{ally}
Maybe. Maybe not. I'm still working on that one. It feels like an ongoing struggle.

\begin{ally}
What's the old saw? You'll finally perfect it six months after death?
\end{ally}
I think that was about when men leave puberty.

\begin{ally}
Let's talk some more about TIASAP.\footnote{Page \pageref{furry:younes}}
\end{ally}
No more, please.

\begin{ally}
Let's talk about puberty.
\end{ally}
That first exploration? I don't know if I'm ready for that, yet.

\begin{ally}
So what \textbf{are} you talking about?
\end{ally}
Well, I was going to talk about that liminal phase, but you seem to have other ideas.

\begin{ally}
That just means you're unfocused.
\end{ally}
Well, yes.

\begin{ally}
Tell me about that place in between, then.
\end{ally}
\newpage

\noindent Shortly after we learned that Margaras died--

\begin{ally}
Less than twenty-four hours. That's pretty short.
\end{ally}
--I wound up in Montreal on the first of many work `sprints'. These were to become a common fixture for the next six years. After all, working from home only gets you so far. Gotta get together, actually learn how the others on your team work. Meet.

\begin{ally}
You had just started at Canonical. Are you sure that wasn't the death of Matthew? Or maybe it was getting married? Creating Younes?
\end{ally}
Matthew was sick for a while. Can we put it that way? He was struggling to hold on, his time was at an end, he was looking rather pale.

\begin{ally}
He was fading.
\end{ally}
Yes.

\begin{ally}
And Madison faded in in 2014.
\end{ally}
I was a transparent person. I was less than real. I was empty, unable to contain an identity. I was a fetch. I was held together with Blu-Tack and paperclips. I was not myself.

\begin{ally}
Are you now?
\end{ally}
Held together with Blu-Tack? I like to think I'm moderately better put together these days.

\begin{ally}
No, yourself. Are you yourself yet?
\end{ally}
Six months after death, remember?

\begin{ally}
Fair. What did you do during your two years as a half-entity?
\end{ally}
Failed. Like, a lot. I failed like it was my job. I failed friends when we moved to Loveland and effectively disappeared from their lives. I failed work when I burned so hard that I burnt out. I failed at communicating. I failed in a lot of ways.

I drank, too. I stopped composing.

\begin{ally}
Was it so negative a time?
\end{ally}
No, of course not. I'm still here. A lot of that failure was the valuable sort. I failed my years at university when I stopped composing, but found that I could still be creative when writing. I failed work when I burned out, but I also learned how to pace myself better (something I definitely hadn't learned up until that point). I learned how to talk, how to listen. At least, how to listen better, how to express myself better.

There's a lot of folks to whom I could credit those being successful failures, if there is such a thing. In a round about way, my boss from the job prior kicking my ass and making me go to therapy, even if not to the ideal therapist, set me on the path to learning how to slow down when I needed to and speed up when that was called for. Writing got me better at putting my ideas --- and, at times, emotions --- into words. Friends, countless friends, helped me become who I am.

\begin{ally}
What's that I'm tasting? Sweet'n Low?
\end{ally}
Is it really that saccharine to be able to look back and say that you sucked, and that you're getting better?

\begin{verse}
She wears a pendant of stamped brass\\
\vin Saying ``Non sum qualis eram.''
\end{verse}

Like, obviously, it sucks to get that regretrospect feeling of looking back and realizing that you were a terrible person, but it's also a good sign that you've improved. If you don't like who you were, at least it's good that you're not that, now.

\begin{ally}
Unless you don't like who you are now.
\end{ally}
That's a different problem. Same class of problem, maybe, but a different problem.
\newpage

\begin{ally}
Was it really so bad to be in this liminal space?
\end{ally}
Of course not. I just got done saying how much I learned during that time.

\begin{ally}
You don't make it sound pleasant.
\end{ally}
It wasn't, I suppose. I mean, obviously there was a lot of good going on in my life. I started a few relationships that are still going strong to this day. I solidified my place in the industry. {[}a{]}{[}s{]} took off. Good stuff came of it. A better me came of it.

\begin{ally}
At what cost?
\end{ally}
Well.

Okay. A lot of that time was bound up in recovery. There was the suicide attempt in March that ate up a lot of my emotional bandwidth on a daily basis for quite a while.

There are a lot of cute metaphors for how pain and grief work on a daily basis. Spoon theory is great and all, but it's starting to lose its luster for me. I like the idea of spell slots. It was like the number of spell slots I had to work with before needing a long rest was reduced by half after that, and it took me two years at least to bring it back up.

\begin{ally}
You remain a parody of yourself.
\end{ally}
It's only been a few days since you reminded me of that.

\begin{ally}
I will never cease to do so.
\end{ally}
Fine.

Another metaphor is that you have a box with a ball in it. On the wall of the box is a button that causes pain, exhaustion, anxiety, your choice. When it starts out, the ball is big and with basically every movement, it bumps up against the button and activates it. Over time, the ball gets smaller and bumps up against the switch less often.

Or maybe you could think of it as endurance. You can hold a glass of water for a few minutes, but after a bit, it becomes painful, and after along time, your arm can start to feel paralyzed. Over time and with training, you might be able to endure that longer and longer.

The last two, in particular, are used often with the idea of grief in mind, which, I suppose, is fitting given how much I still bear over Margaras.

\begin{ally}
Do you feel any for Matthew?
\end{ally}
Less, perhaps.

\begin{ally}
Was it that easy to let go?
\end{ally}
I don't know. Maybe.
\newpage

\begin{ally}
And so when was Madison born?
\end{ally}
On, September 2, 2014, I got this email:

\begin{quotation}
I recently discovered your Twitter page and I wasn't sure if I should say something or not.  When I saw that you are stressing out about telling me about your name change I thought I'd better 'fess up.

I love the name "Madison".  It may take me a while to get used to calling you by your new name so forgive me if I make a mistake.  Madison, whatever direction your life takes you, I'll accept you, support you and love you unconditionally.  Please don't stress out about my reaction.

See you Friday.

Hugs,
Mom
\end{quotation}

And, two days later:

\begin{quotation}
Hey Madison,

Maybe I shouldn't have opened up to you about seeing your Twitter thingy.  I felt like I was being dishonest by not saying anything but it looks like you are really, really anxious about knowing that I've seen it.  Yikes!

Are you OK with me visiting tomorrow?  I'd love to see you but I don't want to add to your anxiety any more than I already have.  Let me know if you have enough spoons.

Love,
Mom
\end{quotation}

\begin{ally}
Did you not want her to come up?
\end{ally}
No, I did. I told her:

\begin{quotation}
Mom,

I'm anxious, but please come up tomorrow. I think I need that more than anything right now.

~M
\end{quotation}

That's when I was born. September 4, 2014 at 3:18 PM. Madison Scott-Clary, 230 pounds, 73 inches.

\begin{ally}
You were born when you could own yourself.
\end{ally}
Yes. I was born when I could share that with my mom. It was all well and good for me to be out on Twitter and what not, and it was great that JD could accept me, but the fact that I could start to regain my biological family without any lies in the way was when I opened my eyes for the first time.

\begin{ally}
How was the visit?
\end{ally}
I don't know. I don't remember. I think it was fine. We talked about me starting hormones--

\begin{ally}
Did you talk about TIASAP?
\end{ally}
\emph{No.}

No, we did not. If she's reading this, which she may very well be, this will be how she learns about that.

How could I possibly talk to my mom about something like that? I hid my arms and legs from her for years before, and it wouldn't be for another year before I could even bring up the concept of self-harm.

\begin{ally}
That's not true.
\end{ally}
I\ldots{}well, no, it's not.
\newpage

\noindent Telling dad was the second time I came out to family deliberately.

\begin{ally}
The third.
\end{ally}
Third?

\begin{ally}
You told Aunt Patty that you were gay back before high school.
\end{ally}
I\ldots{}did not remember that.

\begin{ally}
Not until just now, apparently.
\end{ally}
Apparently. I have no recollection of what I said. I have no recollection of what \emph{she} said.

I have no recollection of her.

\begin{ally}
Hazy images at grandma's.
\end{ally}
I guess.

\begin{ally}
Memories surrounding her.
\end{ally}
Lots of those.

\begin{ally}
Memories of when she and her family got stranded on a sailboat between Cuba and Florida and rescued by a cruise ship. Grandma and dad smug in their assessment that she was stupid and irresponsible.
\end{ally}
A vague, heavily pixelated picture shot by one of the cruise boat attendants.

\begin{ally}
``She's crazy,'' they said. ``She has too many kids. They draw all over the walls. Her house is wild. She's crazy.''
\end{ally}
And me, with with my secret. My little pet lie I kept hidden from them.

\begin{ally}
Tell me about coming out to dad.
\end{ally}
I will.
\newpage

\noindent Coming out to myself and JD was more gradual. A sea-change.

\begin{ally}
Maybe that's what those two years were between Matthew and Madison were.
\end{ally}
Nothing of him that doth fade, but doth suffer a sea-change into something rich and strange.

I suppose so. I explored around the edges of it. I touched it tentatively. I lived my life in widening circles.

\begin{ally}
Surely you mean narrowing.
\end{ally}
Okay, yes. It was too good a line to pass up, though. Shakespeare \emph{and} Rilke in one go?

\begin{ally}
There is nothing new under the sun.
\end{ally}
Ooh, and Ecclasiastes, you spoil me.

\begin{ally}
Treat, as they say, yourself. Carry on.
\end{ally}
There were little fits and starts between James and I. I remember laying on the couch --- that awful, awful yellow couch --- and him getting playful, and then some little movement of his touched a nerve and I started crying because of the way that brushed up against that me that wasn't in focus. It brought it to the forefront the fact that I didn't align with myself, that there was a lag in my proprioception, that I was falling behind myself.

Is there some word for ecstasy that doesn't imply it being positive? Something that captures the feeling of being outside oneself, beside oneself, behind oneself without implying the sense of greatness, of awe that goes along with spiritual \emph{ekstasis}?

\begin{ally}
Dissociation?
\end{ally}
Yeah.

That.

That little bit of panic-colored dissociation that I would later name dysphoria would come in waves. Sometimes it'd be triggered, as it was then. Sometimes it would fade slowly into view and I'd go on a tear making skirts and then it would fade back into the low background static of the anxiety that goes along with being a member of a minority identity group.

\begin{ally}
There \textbf{was} ecstasy, though. There was euphoria as well as dysphoria.
\end{ally}
Yes.

The moment when my hair got long enough to put up in a ponytail.

The utter terror of shaving my legs for the first time, weird as it sounds. Outrageously stupid, and yet the feeling of \emph{having} shaved legs was incredibly validating.

The first time I looked in the mirror and saw the trace of femininity.

The softening of skin.

The first ``she'' on the street.

The first ``ma'am'' on the phone.

Hell, the first time dressing feminine.

\begin{ally}
What, back when you were nine? When you snuck into the spare room and tried on one of Julie's dresses?
\end{ally}
Holy \emph{shit} could you just \emph{shut up}.

\begin{ally}
Wow, touched a nerve, there.
\end{ally}
We will talk about that later.
\newpage

\noindent You know what? No, I take that back. We'll talk about it now.

\begin{ally}
Tell me about the dress.
\end{ally}
It's not even about the dress.
\newpage
\end{leftcolumn}
\end{paracol}
