When I was interviewing for Canonical, the tic had not yet started, or at least not yet to the point where it was affecting my neck or my voice. By the time I \emph{did} start at Canonical, it was well underway.

Much to my chagrin, not only was I stuttering at the time, but the job required daily video calls.

\begin{quote}
You begged off the first few, putting the blame on hardware failures. After the third day, Gary gently suggested that you consider fixing the hardware issues so that the team even knew what you looked like.
\end{quote}

It was embarrassing. Hangouts couldn't even keep up with it. The video was jittery and blurred, my face only in focus for maybe half of the time.

And then, within a few days, it cleared up and went away.

\begin{quote}
The stress of the previous job, of interviewing and those last two weeks, all suddenly relieved in one fell swoop.
\end{quote}

Yes.

\begin{quote}
And then it came back.
\end{quote}

As we all worked from home, the company had us get together in one location four or five times a year for a week at a time in order to work face-to-face and accomplish far more than we would otherwise. They called them sprints, an apt enough comparison.

Copenhagen, though, was different. It was a cascading set of stressors that culminated in, yes, the tic coming back. Two weeks long, with the first half being the developer summit, followed by a week of sprinting. The core product being rewritten. Zephyr getting attacked by another dog while I was away. The hotel, that building \href{/movement/copenhotel.jpg}{canted over to the side at a precarious 15° along two axes}, a nightmare on the acrophobia side.

The tic started up, then got worse and worse.

It was about this time that I started getting closer to Robin, and by the time we had our first real time together at FC 2013, I had shaken my sense of balance from myself and walked with a cane. ``You have a cane,'' she said, part confused, partly out of acknowledgment.

``Yeah, I lost my balance with the tic.''

``That's okay.''

And then we hugged.

\begin{quote}
Not all of it was your balance. Some of it was an apology.
\end{quote}

Yes. Someone with a movement disorder who pretends it isn't there is, in some ineffable way, sadder than someone who at least makes some public acknowledgment that, yes, this is happening. The cane helped. People would see me shaking my head, see me shaky on my feet, and then see the cane and know, ``Ah yes, \emph{this} is happening.''

\begin{quote}
You happened to pass by one of the attendees from the data panel shortly after, and overheard him telling his friend, ``That was a really cool panel, but I think he had Parkinson's or something. Every time he would get more interested in what he was talking about, it would get worse.''
\end{quote}

Yes. Part of me was embarrassed, sure, but part of me was relieved to be seen.
