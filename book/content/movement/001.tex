I will be the first to admit that it is difficult to write about mental health, as is certainly evidenced here already, and in countless other projects where I've tried to get that across. Even when talking about it, my voice is filled with ellipses and my words littered with hedges, fillers, and all sorts of metalinguistic dross.

\begin{quote}
That you later had to learn to use those consciously, to string like-and-if-um-but-so through your words like fairy lights to anchor your pitch is neither here nor there.
\end{quote}

And that's transition stuff. A totally different side-quest. Don't distract me.

\begin{quote}
Right. And yet here you are, distracted, talking about how difficult it is to write about mental health.
\end{quote}

Touché.

That I'll be the first to admit that doesn't excuse the way others treat it. Of course, there's countless words to be spent on the way media treats it, or the way writers treats things like psychosis, but the experience is so often so poorly researched that it hits the point of not even wrong.

Take, for example, Orson Scott Card.

\begin{quote}
There's a juicy one.
\end{quote}

Much to be said on him, yes, but take \emph{Xenocide} and \emph{Children of the Mind} as examples on this topic in particular. Take the World of Path. Take this supposed obsessive-compulsive disorder that plagues some of its inhabitants.

\begin{quote}
Is it wrong?
\end{quote}

It's not even wrong. It's based on a lack of experience. It's based on this societal view of OCD, not the experience of it.

\begin{quote}
You sound bitter.
\end{quote}

I have a problem with compulsions. Not-even-wrong-ness surrounding them touches on a sort of meta-compulsion: a need to be understood strong enough that, when I'm misunderstood, it itches. It gets a liquid flip of my hand and touch of thumb to palm. It triggers cascading compulsions.

To then make that entertainment, to make that a hook for a plot, well.

\begin{quote}
Was it really so off-base? Did the symptoms not fit?
\end{quote}

Not all of them.

\begin{quote}
And yet the plot hook is that it was artificial in the first place. That's sort of the point, right? Fei-tzu and Qing-jao are saddled with this form of compulsive behavior that's the side effect of something else, not OCD in and of itself. Was it really so off-base, or are you just upset at seeing part of --- but not all of --- yourself?
\end{quote}

I don't know.

\begin{quote}
Are you just upset that you can't stay still; that you have other, unrelated problems with compulsion; and that these two are then correlated in a fictional genetic disorder where they are not correlated for you?
\end{quote}

Straight homeward to the symbol essence, is it?

\begin{quote}
Yes.
\end{quote}

Let's talk about movement disorders, then.
