Only five months after I figured out just what akathisia was, the disorder evolved into something more dramatic. Whereas the tic, whether or not it was iatrogenic, affected mostly my neck and only rarely my wrists, this new form of drug-induced movement disorder affected most of my upper body, dystonia alternating between athetosis and chorea; between a fluid, graceful swimming of limbs to a tense, rigid posture with repetitive jerking movements.

It was infuriating and humiliating --- and before you interrupt, no, I will not talk about kink.

\begin{quote}
You know me so well.
\end{quote}

I suppose I do.

To be unable to hold still is one thing. Jerking my head to the side once every few seconds with the tic was embarrassing enough. I often worried that I'd be mistaken for some sort of junkie, hopped up on something or another. I even had my doctor write a letter explaining what was happening that I could bring with me when I traveled.

\begin{quote}
But you were still functional.
\end{quote}

Yes. I could still work. I could still drive and walk and pick things up and eat.

\begin{quote}
Now you couldn't. Now your hand would jerk back from picking things up or hitting the keys. Now you would walk with a hitch in your stride as a spasm rolled along your side. Now you wouldn't feel safe behind the wheel.
\end{quote}

I mostly just shut myself in my house. I left twice. Once to see a friend for some company, and once to go to therapy. I stood in the lobby while my therapist had a small chat with a coworker, struggling to keep still with my hands buried in my pockets, and broke down crying once we made it to the room.

She had a solution --- or a set of solutions --- that we could try. One medication, benzatropine, to start with, one fallback medication, tetrabenazine, and a intensive vitamin regimen to start on right away. Picking them up at the pharmacy on the way home was another source of tears, as the pharmacist, reading off the screen, said, ``This is for twitching? Involuntary movements?'' and I nodded, more a jitter than an intentional motion, as my hands wandered off along strange hyperbolae, unable to speak for the tears.

\begin{quote}
And then, Thanksgiving.
\end{quote}

Yes. Thanksgiving, and my dad visiting.

\begin{quote}
He had seen the tic before, at least.
\end{quote}

Well, yes, but as mentioned, these movements carried along a whole new set of connotations with them. Suddenly I was unable to have a basic conversation without the pauses that come with those moments of fixed posture. Suddenly I was unable to get a bite to eat without engaging in my geste antagoniste, resting my chin on the back of my hand with my wrist twisted around unnaturally.

\begin{quote}
Suddenly you were painfully, visibly vulnerable in front of him.
\end{quote}

Yes.

And at a restaurant. A dinner that cost him eight hundred dollars for the four of us.

\begin{quote}
At one point, he asked you what was wrong and you tried not to cry as you mumbled, ``I'm just having a hard time holding still.''
\end{quote}

No one mentioned it, after that.
