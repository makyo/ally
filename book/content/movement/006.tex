\hypertarget{icd-10-cm-diagnosis-code-f95.0}{%
\section{2019 ICD-10-CM Diagnosis Code F95.0}\label{icd-10-cm-diagnosis-code-f95.0}}

Transient tic disorder

\hypertarget{applicable-to}{%
\subsection{Applicable To}\label{applicable-to}}

\begin{itemize}
\tightlist
\item
  Provisional tic disorder
\end{itemize}

The following code(s) above F95.0 contain annotation back-references that may be applicable to F95.0:

\begin{itemize}
\tightlist
\item
  \textbf{F01-F99}\\
  Mental, Behavioral and Neurodevelopmental disorders
\item
  \textbf{F90-F98}\\
  Behavioral and emotional disorders with onset usually occurring in childhood and adolescence
\end{itemize}

\hypertarget{approximate-synonyms}{%
\subsection{Approximate Synonyms}\label{approximate-synonyms}}

\begin{itemize}
\tightlist
\item
  Recurrent transient tic disorder
\item
  Tic disorder, childhood, transient
\item
  Tic disorder, transient
\item
  Tic disorder, transient, recurrent
\item
  Tic, transient childhood
\item
  Transient childhood tic
\end{itemize}

ICD-10-CM F95.0 is grouped within Diagnostic Related Group(s) (MS-DRG v36.0):

\begin{itemize}
\tightlist
\item
  091 Other disorders of nervous system with mcc
\item
  092 Other disorders of nervous system with cc
\item
  093 Other disorders of nervous system without cc/mcc
\end{itemize}

\begin{center}\rule{0.5\linewidth}{\linethickness}\end{center}

\hypertarget{icd-10-cm-diagnosis-code-g25.71}{%
\section{2019 ICD-10-CM Diagnosis Code G25.71}\label{icd-10-cm-diagnosis-code-g25.71}}

Drug induced akathisia

\hypertarget{applicable-to-1}{%
\subsection{Applicable To}\label{applicable-to-1}}

\begin{itemize}
\tightlist
\item
  Drug induced acathisia
\item
  Neuroleptic induced acute akathisia
\item
  Tardive akathisia
\end{itemize}

The following code(s) above G25.71 contain annotation back-references that may be applicable to G25.71:

\begin{itemize}
\tightlist
\item
  \textbf{G00-G99}\\
  Diseases of the nervous system
\item
  \textbf{G25}\\
  Other extrapyramidal and movement disorders
\item
  \textbf{G25.7}\\
  Other and unspecified drug induced movement disorders
\end{itemize}

\hypertarget{approximate-synonyms-1}{%
\subsection{Approximate Synonyms}\label{approximate-synonyms-1}}

\begin{itemize}
\tightlist
\item
  Acute akathisia caused by drug
\item
  Drug induced acute akathisia
\item
  Drug-induced akathisia
\item
  Neuroleptic induced acute akathisia
\item
  Tardive akathisia
\end{itemize}

\hypertarget{clinical-information}{%
\subsection{Clinical Information}\label{clinical-information}}

\begin{itemize}
\tightlist
\item
  A condition associated with the use of certain medications and characterized by an internal sense of motor restlessness often described as an inability to resist the urge to move.
\end{itemize}

ICD-10-CM G25.71 is grouped within Diagnostic Related Group(s) (MS-DRG v36.0):

\begin{itemize}
\tightlist
\item
  056 Degenerative nervous system disorders with mcc
\item
  057 Degenerative nervous system disorders without mcc
\end{itemize}

\begin{center}\rule{0.5\linewidth}{\linethickness}\end{center}

\hypertarget{icd-10-cm-diagnosis-code-g24.01}{%
\section{2019 ICD-10-CM Diagnosis Code G24.01}\label{icd-10-cm-diagnosis-code-g24.01}}

Drug induced subacute dyskinesia

\hypertarget{applicable-to-2}{%
\subsection{Applicable To}\label{applicable-to-2}}

\begin{itemize}
\tightlist
\item
  Drug induced blepharospasm
\item
  Drug induced orofacial dyskinesia
\item
  Neuroleptic induced tardive dyskinesia
\item
  Tardive dyskinesia
\end{itemize}

The following code(s) above G24.01 contain annotation back-references that may be applicable to G24.01:

\begin{itemize}
\tightlist
\item
  \textbf{G00-G99}\\
  Diseases of the nervous system
\item
  \textbf{G24}\\
  Dystonia
\item
  \textbf{G24.0}\\
  Drug induced dystonia
\end{itemize}

\hypertarget{approximate-synonyms-2}{%
\subsection{Approximate Synonyms}\label{approximate-synonyms-2}}

\begin{itemize}
\tightlist
\item
  Dyskinesia, subacute, drug induced
\item
  Neuroleptic induced tardive dyskinesia
\item
  Subacute dyskinesia due to drug
\item
  Tardive dyskinesia
\end{itemize}

\hypertarget{clinical-information-1}{%
\subsection{Clinical Information}\label{clinical-information-1}}

\begin{itemize}
\tightlist
\item
  Iatrogenic extrapyramidal disorder produced by long-term administration of antipsychotic drugs; characterized by oral/lingual/buccal dyskinesias and choreoathetoid movements of the extremities.
\end{itemize}

ICD-10-CM G24.01 is grouped within Diagnostic Related Group(s) (MS-DRG v36.0):

\begin{itemize}
\tightlist
\item
  091 Other disorders of nervous system with mcc
\item
  092 Other disorders of nervous system with cc
\item
  093 Other disorders of nervous system without cc/mcc
\end{itemize}
