\begin{quote}
And now you're still again.
\end{quote}

Sometimes. One of the treatments worked, though I'm not sure which. One of them caused vertigo and nausea, though I'm not sure which. But even after I went off them, I'm usually still.

\begin{quote}
Is that not enough?
\end{quote}

It's certainly better, don't get me wrong. The stress of driving will bring out the dance-like turn of my arm. An interview a few weeks ago went poorly after the twitching and twirling got bad enough to prevent me from focusing on the problem at hand. A distressing scene in a movie will leave me paralyzed and rigid in my seat, posture unnatural and unnerving.

Judith reassured me that it looked like I was stretching, that it was less distressing than the tic.

\begin{quote}
You still apologized. You apologized to all of your partners the first time they saw it, and countless times after.
\end{quote}

Yes. I explained and explained, hoping they'd forgive me.

\begin{quote}
For what? For being less than perfect?
\end{quote}

For being vulnerable. Even after so long away from my dad and Jay, it's ingrained in me that vulnerability is a personal failing. Or perhaps it's more general: perhaps vulnerability is worth apologizing for because of some hereditary reason. Perhaps I'm apologizing to my ancestors, to the human race, for being less than they hoped for, for being a disappointment.

\begin{quote}
How very human of you.
\end{quote}

My therapist apologized to me on one stressy day when I was visibly struggling to stay still. She said she felt bad for having caused this. I rushed to reassure her that, no, it probably wasn't her fault, that I'd been on the antipsychotics for a while before ever meeting her. That the tic started back in 2012 before I'd even started those.

\begin{quote}
You apologized for the fact that she felt the need to apologize.
\end{quote}

Well, yes.

\begin{quote}
It's not your fault either, you know.
\end{quote}

On an intellectual level, sure. I know. On some deeper level, obviously I don't. Or can't.
