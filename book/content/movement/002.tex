Everyone, I suspect, deals with movement in a different way. Some are content to sit still where others have to move. Some must move, and it is a part of their personality. Some cannot move and it is a part of their physiology.

And some must move because it is an aching necessity. There is no ``if they do not move, then\ldots{}'' statement to be made. They must move. They can't \emph{not} move.

\begin{quote}
It started as a twitch, you said, as a slight nod of the head.
\end{quote}

Or perhaps it started earlier, I don't know.

\begin{quote}
Perhaps it was all caused by the meds, or perhaps it was presaged by some other restlessness that started years before.
\end{quote}

Perhaps, but does it matter?

\begin{quote}
If it was the meds' fault, you could blame them, but if it was unrelated, you would be able blame yourself. If it was the meds' fault, you could stop, if it was unrelated, you would take that as permission to feel broken.
\end{quote}

Yes, I suppose it does matter, then. That said, I have no answer for that. I just know that it started with a twitch, a slight nod of the head. My fingers would duck up away from the keyboard as though suddenly burned by the keys. I would go and sit in my car over lunch and wring my hands over and over again, occasionally trying to force myself to hold onto the shifter and the door handle, and the tremors would travel up my arms.

Eventually, at some undefinable point, it made its way up into my neck.

I never knew how to explain it.

\begin{quote}
How would you now, with seven years' experience under your belt?
\end{quote}

``Transient tic disorder''. Maybe not so transient before it disappeared, back when I thought it was going to just stick around forever.

\begin{quote}
That's what it's called, but how would you get it across?
\end{quote}

Sobbing? Frustration? Humor? I had a whole comedy set prepared for it, in case I, for some reason, needed to do a stand-up routine.

As you can see, I have a motor tick on my neck that makes me jerk my head to the side and do stuff with my hands. This is because I have transient tic disorder, or as I like to call it, tourettes with holidays.

It makes work life interesting. I stare at a screen all day at my job. Or, well, I stare at my screen and also a point on the wall right about \emph{point} there. It's sort of a timeshare.

I could probably get jobs doing other things, though. Some contract work. Like, hey! Need someone to shake their head `no' at something? I'm your gal. Or maybe you need someone to urgently point something out out with their chin over \emph{point} there. I'd be good at that.

Now, there's a few jobs I won't be good at. Surgeon? Probably not. Bomb squad? That's a definite nope. Professional staring competition participant? I'd be right out. I couldn't win a staring competition with a three year old who's just discovered espresso.

I actually learned about all this tic nonsense at work. It started back in 2012 when it slowly started up over the course of a few days. Went on to find out that it's made worse by stress \emph{lean to the side} stand-up, of course, being the least stressful of occupations \emph{lean back} But no, I worked in health insurance. Health insurance in America as Obamacare is kicking in? Yeah, not exactly a stress-free environment.

Now, this is mostly a motor tic. I don't have the verbal tics that folks associate with tourettes. However, it does make me stutter when it gets bad. If you've never stuttered before,I can tell you that it's infuriating, so, honestly, I didn't need a verbal tic to get me cussing all the time.

So there's me sitting in meetings with other insurance companies, shaking my head `no' to everything they say, and when I try to correct myself, it comes out ``I mean ye-yes FUCK sorry''. I got really good at the whole FUCK-sorry combo.

And so on.

\begin{quote}
How effective do you think that would be on those conference calls with Lewis as you were stuttering away?
\end{quote}

I don't think I could manage. At that point, it was embarrassing enough to have picked up a stutter, a movement disorder that I never explained to my boss or the PM. To acknowledge it to the client would have been mortifying.
