There is a certain unique agony to akathisia. When I was in the hospital after surgery, and even for weeks afterwards, I was dead convinced that the problem I was going through was related to temperature. Part of this, no doubt, was due to the weather warming up followed by, toward the end of my inpatient stay there, the climate control in the room going out, leaving it a sweltering (to me) seventy-six degrees.

\begin{quote}
What you didn't take into account was the fact that you have a hard time sitting down for an hour at a time, never mind being confined to bed rest laying on your back only for five days straight.
\end{quote}

Even so, for weeks afterwards, I was desperate to do anything I could to stay cool. I picked up an ice cream habit that I'm still fall into regularly. I installed a window A/C unit. At one point, I even contemplated sleeping in the garage where it was cooler at night due to the lack of insulation.

\begin{quote}
Judith visited toward the end of this period. You did everything you could to keep the rooms you stayed in on the road trip to the bay as cool as possible. The bay, where A/C just isn't a thing.
\end{quote}

Yes. And shortly after that, I learned about akathisia.

I say `shortly after', when it was likely during that trip when I realized I felt the most relief from the symptoms by moving. The constriction imposed upon me by recovery had lessened over time until I was able to go for that hike with Judith, Robin, and Josh, and suddenly I realized that I felt better than I had in a while.

I just learned the word for it shortly after, the name. And by naming a thing, hoped to gain some sort of power over it.

Alv pinned his ears back against his head as he stomped down the length of the block. His boots were too much for the drizzle that the weather offered, but it was that or his threadbare sneakers, and some tiny part of his mind had done the calculation without the rest of him knowing, and he'd tugged the heavy things on for the walk.

\begin{quote}
Because of course you have a furry story about akathisia.
\end{quote}

Write what you know.

The air inside had grown too stuffy for the old fisher, or perhaps his eyes had grown too tired of reading, or maybe it was something in his joints, a feeling of too much space that needed to be compressed down. The solution, no matter the problem, was to move.

His third time around the block, knees and hips aching from walking in work boots that were never meant for the task, and Alv still hadn't figured out what it was that kept driving him out of the house. He'd walk, day after day, until his tail drooped and his feet started dragging. Sometimes, like today, he'd circle the block. Some days he'd drive the mile to the supermarket and walk aimlessly up and down each aisle, eventually picking up a drink or a snack, just to make the trip worth it. Other days, he'd just pace in his building's parking lot.

He didn't think.

Or maybe he thought too much. Maybe that was it. Maybe the fisher's every step was taken to crush too many thoughts beneath the soles of his boots, pressing the life out of them through the sheer weight of his restlessness.

\begin{quote}
And you would, too. You'd walk and walk and walk, hoping that perhaps you could walk the thoughts out of you.
\end{quote}

Yes.

He didn't know what it was that, day by day, drove him to his feet, drove him to walk until he couldn't walk anymore. He just knew that if he didn't, that ache within him, that burning, that itch would continue to grow, and he'd start to feel like his heart was being extruded through his rib cage, like his fur was coming out in clumps, like he couldn't possibly breathe deep enough.

His wife, gone now these five years, had been fond of calling him a restless soul. He wasn't sure that he was capable of believing in a soul, nor that this increasingly restless state of being was confined to something so intangible. He was just restless.

Just. Only.

That's all he was. There was nothing to him except restlessness. After Naomi's death, he'd slowly become less and less of a person, until all that was left was the urge to move, the terror over being confined to one place for any length of time.

His tail starting to sag, the fisher could feel all the calm he'd accumulated through the walk start to ebb, the tide of anxiety creeping in from the edges, from his fur inwards. One last trip around the block, he figured, was all he could manage before resting again.

\begin{quote}
Write what you know.
\end{quote}

Yes. Furry is a framework. Apply an experience to that framework and see what you get.

\begin{quote}
Sure, but we've already been over that.
\end{quote}

Yes.

\begin{quote}
Write what you know. Write about the way pacing slowly moved from its status as nervous habit to a necessity, to an ache. Write about how there was no relief in walking, just a drive, an itch you could never scratch but were nonetheless required to try. Write, and cast those words upon something else, upon someone else, so that you can look on them and say, ``Ah yes, \textbf{this} is happening.''
\end{quote}

By the time he made it around to his building again, Alv was well and truly sore, knees and hips aching from the repetitive motion of stomping around the block. Still, he couldn't bring himself to head up to his apartment quite yet. The idea of being closed in such a space held negative appeal. Something about the thought of four walls was actively repulsive.

So he sat on the damp stoop and watched the trees across the street.

The drizzle had dried up---though he hadn't noticed when---and all that was left was the occasional pat of drop on leaf as some bit of water got too heavy and sought a new home closer to the ground. There was just that gentle sound. Despite the hour, the street was empty of traffic, as though the shoddy weather had chased everyone inside.

``Would that my soul were that calm,'' he mumbled to the bare street at last and levered himself up creakily, climbing the rest of the stairs to head inside.

\begin{quote}
Write what you know but don't yet understand.
\end{quote}

Maybe I can get closer that way.

\begin{quote}
Yes.
\end{quote}
