\emph{Distractions --- Pleasure first and pleasure in all things --- Reeling --- Consequences}

School provided an ample distraction for me from my spiritual pursuits, but even so, I was still left with some free time to explore other interests. The internet still occupied much of my time, and through it, I found myself picking up a few different hobbies. As may be obvious by now, my attention does tend to wander from one topic to another fairly often and I've wound up with a good collection of stuff --- both intangible knowledge and tangible items --- related to all of these brief infatuations. However, I'd have to say that the thing that makes me happiest in the world is this exploration of the different corners of the universe and building my knowledge up higher and higher, as there is always still more and more to learn.

In this way, I consider myself a hedonist, or at least rather selfish. I suppose by garnering all of this knowledge and related materials, I was, as Jesus put it, building up wealth (of a sort) in this life instead of working for the next. It felt good for me to build a wider and wider base of knowledge on which to build myself. It felt good to have tangible evidence of my skill, and to be able to demonstrate it. This, I think, is where the selfishness showed up --- thought it did feel good to have all this, I felt rather bad in having it. It felt as though I was bragging, and continuously searching for new things to brag about. I still struggle with this, and I do my best to keep humility in mind.

Along with this garnering of knowledge, I did my level best to cherish experiences and emotions as well. While it might be slightly contrary to the definition of hedonism, I didn't do anything to avoid depression and pain to focus just on positive emotions and pleasure. Rather, when depression came up, I did my best to dissect the feeling both in an attempt to remedy it as well as cherish the feeling while it was there. With pain, I focused on the pleasure within it and toyed with finding descriptive words and phrases for it. A paper I found on my floor recently offers a glimpse of this: ``Pain is the harsh light that illuminates our lives in a stark contrast of ups and downs; it is the gently persistent glow that brings color to our pleasure; we breathe pain --- the scent of snow on the way in and the taste of blood on the way out, frigid to the core no matter how hot.''

With these descriptions in mind, I began almost subconsciously to attempt to synaesthetically catalog my different emotions and sensations in terms of sensory responses. My early attempts back in high school described emotions and the thoughts tied to them as clouds of color in different locations within and surrounding my body. I think that, by attempting to picture the colors before I tried to decipher the emotions involved helped me to differentiate between separate and related emotions. As an example, I wrote, ``when pondering (attraction), a luminescent fuchsia color that seems to be flowing in the right hemisphere of my brain; when thinking of (a significant other) and snuggling, a warm, earthy brown with a little bit of green in a pine-needle-ish pattern about a foot and a half in front of me and slightly to the left; tiredness is off-white everywhere and blind hopelessness is bright blue wrapped around my mind.'' However, this exercise was rather draining, and I didn't keep it up for long.

This lust for experience and betterment eventually lead into exploration of drugs --- I'll be blunt; mind-altering substances is a nice phrase, but food and water are mind-altering substances --- beginning with the obvious months of research on Erowid and like sites back in high school. Upon the way, I came across a page about Salvia divinorum and its effects, including a chapter from the book Pharmako/poeia by Dale Pendell. I purchased this book and skimmed through the amazingly poetic content (I even began writing in his style --- if anyone remembers my 'ally' --- while reading the book) all while still researching the interestingly bizarre plant that is Salvia. I finally worked up the courage to purchase some Salvia just to see what it was like.

The third time was the charm, and also the most terrifying. The first two attempts at trying the plant resulted in little more than hypoxia, but, as I'd read, there was a bit of reverse tolerance --- the drug got stronger as time went on. Never has anything instilled such fear in me, and, in time, such respect. While I had steered clear of drugs throughout high school, preferring instead to sit and watch from the sidelines as a girl in my world literature class freaked out on mushrooms, I only began to really respect them with this experience.

What exactly happen sounds rather mundane and funny in retrospect: having smoked a little bit of the extracted plant material in an empty room, I was neatly destroyed before I even had a chance to exhale the first breath. I felt that I had lost nearly all sense of my ego, and I was clinging to what remained by the barest of threads while my room tried to eat it. Having fallen over on my side, the baseboard heater had become a mouth, the window a solitary eye, and the vast expanse of the empty room a muzzle and throat of some sort of beast emanating from my chest, intent on eating my ego and any lingering sense of self. With Salvia comes a certain gravity --- it pulls back and to the right, for me --- along with a rhythm of about two or three strikes a second, and this turned into a sensation of being caught in the maw of this beast while it struggled to dislodge me with its tongue in order to swallow.

To be honest, I'm not sure how my deep sense of respect for such a powerful plant emerged from such a situation, other than perhaps the sense of ego-death caused by it. Also, it made me realize what a control freak I can be when it comes to my mind. My worst fear in the world at the time was insanity, of which I was given a brief glimpse. Part of, I believe, my trouble with that experience was the need to hold onto the strand of my ego throughout the process and not let it go.

The next experience, that of psylocybe mushrooms, completely destroyed all of that. Salvia is a quick experiencing. From start to shaky baseline was likely no more than five or ten minutes. With mushrooms, I was clearly not myself for a good three or four hours, and was not back to baseline for another four hours after that. Sometime during this process, I started to break out in a mild case of hives, which, while you're in the process of going crazy, does little to help the situation. While I had been pleasantly goofy before, I suddenly turned into a mess of fear and panic, getting stuck in a time loop in the bathtub, and spending half an hour writing to myself that I had just taken a psychoactive substance in order to convince myself that I was still sane.

It was after this that my respect for Salvia grew even more. It took another year after the episode with the mushrooms, but I finally tried another psychoactive substance again, and this time, I let the herb steal away my ego, placidly going through a sort of ego-death in order to experience the rawer side of myself that is normally buried under the crust of the Self. While the imagery of the `trip' was fairly standard --- floating up through the branches of a limitless tree as the layers of my mind were laid bare to me --- the deeper meaning struck me as a very introspective look at some of the parts of my mind that I don't normally get to see. The next evening, I attended a Sufi zikr (`dhikr' depending on the tradition) ceremony with a very close friend in the music department, and I was tempted to ask for a mystical interpretation of the experience while the leaders of the ceremony engaged in a traditional interpretation of dreams.

My explorations with other substances have also been introspective, but none so deeply. To take a phrase from Dale Pendell, they were, rather, ground-state training. I have toyed with large doses of caffeine and then fasted from it in order to take a look within myself and see what courses my thoughts take both on and off the substance. I have sought empathy in plants such as Kava kava, blue lotus, and pot, and found it in only limited qualities. I have toyed with the concept of addiction --- something my mother warned me ran in the family --- and intentionally gotten myself addicted to alcohol in order to see what the concepts of addiction and withdrawal mean to me, even to the point of having several of my friends worried for me (though I honestly feel that I'm a safer drinker than most college students --- I drink often, yes, but rarely more than two drinks). Oddly, I tried to toy with the same thing with opium (in the form of poppy tea), but never found what was purported to be one of the most intense addictions. The whole experience was rather dull, really. The most comfortable `dull' in the history of my life, yes, but dull. The other substance that one equates with addiction, tobacco, often makes me vomit, so I tend to stay away from it based on a more physical aversion. This ground-state training is more yogic than usual drug use, but certainly pertinent to my explorations. The poison path remains a part of my life.

Of course, none of any of my hobbies came cheap: I've never been one to skimp on quality even when I'm hunting for bargains. Though I come from a rather affluent background, this gave me my first taste of debt, which, to be certain, has gotten rather out of hand as of late. As a result, I've gone through one of the more drastic lifestyle changes yet as of late: while I've tried to get rid of stuff before, I've never done so with as much abandon as I have now. When I began this change in my life to work way from my previous excess and my current comparatively ascetic lifestyle to a happy medium, I laid strict ground rules for myself --- family tradition would hold little to no weight, personal value would be based more on how often I used the item in question, and I would not always try to sell for the highest price, for that would often result in the item not selling. Again, this was quite self-centered of me, intended to get me out of debt and into a comfortable life rather than to make me a more worldly person, but I feel that it has certainly contributed and will continue to contribute to constructive growth as a person.

How does this tie into my personal faith? Well, I don't suppose it does in a direct way, really. However, faith is not the only aspect in life, and other aspects do need to be taken into account. I think that this has all brought to me a grounding in the more tangible word that surrounds me as well as a clearer idea of how my mind and body work on a more basic level than any amount of introspection and reading can gain. While this spirituality business is certainly an important aspect of my life, of life on a whole, it is not all that one can focus on. There are bills to pay, I've found, both literally and figuratively, and one must work out the financial system before one engages in transactions.
