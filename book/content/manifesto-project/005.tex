\emph{Other aspects of being --- The terror of individuality --- The spirituality of fiction --- The beginnings of true creativity}

There's a long space of time after my initial intense exploration of paganism, which is filled with a nebulous sense of spiritual growth not connected to any particular spiritual set. I attribute this to a general ``opening of the mind'' from the gaining of more concrete intelligence. My interests started to shift from their previous areas of simple pleasures of reading, playing outside, making slings out of kite string and the toes of socks into subtler, more complicated pleasures of the more in-depth learning of high school. This is not to say that I enjoyed school --- I considered dropping out at several points --- but I did enjoy the act of gaining more knowledge, and in such diverse subjects. This, for me, was the beginning of learning to think within concrete systems, an idea that I'll certainly come back to later. These were, at first, the more obvious systems of grammars (I began in Latin my sophomore year, and began constructing my own language shortly afterward), history (learning to think and analyze historical data is something I attribute to my one history teacher in high school, Dr.~Carter), and biology (microbiology and biochemistry in particular --- the latter was even my original major in college), not to mention music, which is a topic unto itself.

Such intellectual things were not the only changes going on during my life. From the end of eighth grade and into ninth, a few other changes, both subtle and dramatic, took place. Though I'd suspected for quite a while, my initial feelings of sexuality crystallized into a definitive sense of something out of the ordinary. Beginning as trouble understanding the idea of what was `attractive,' I eventually settled on the label of homosexuality for what I felt, coming out to my mom sometime soon after middle school had ended. This also coincided with my growing infatuation with the internet, something which has, at points, gotten way out of hand. At one point, I was the moderator of an online forum on GovTeen.net with my then-boyfriend Danny, another teenager with similar interests living in New York.

At the same time, my mom and step-dad's marriage started to turn sour for various reasons. While my mom had taken my coming-out fairly well, my step-dad did not; at points during this continuing strife which lasted part-way into my freshman year, he forced me to come out to his children in a rather embarrassing fashion (he told my step-sister, and made my mom force me to come out to my step-brother), checked my email and found emailed replies from the forum I moderated including some very revealing information (the forum was one of many in a group entitled Puberty-101 --- this should explain a good deal about the content), all while refusing to talk to me directly about such things. I harbored an intense dislike for him at this phase and I don't feel that I fully forgave him for all of what happened until much later in my life when I started to incorporate it into myself. Thus, I was very willing to let my mom use my orientation as the reason for breaking off the marriage, though that was only a small portion of the myriad of reasons for divorce. In honesty, I believe this was as high on my list of influences in my life as my previous flight from home, perhaps due to the similarities in how the situation turned out.

The divorce was finalized and we --- my mom and I --- were planning on moving out to a townhouse very close to my high school in the next few days, and until then, my mom was sleeping on the couch in the family room with her two dogs Helen and Hank. My step-dad, perhaps with a belated riposte, came down the stairs to talk to her when, Helen, being out of control in the best of times, began barking and ran up, jumped on him, and, in short, punched him in the crotch with her paw. Humorous in hindsight, the event led to my mom and I having to move out of the house by that evening, while we were both only partially packed at the time. This was halfway through the first semester of my freshman year at Fairview and at the time, it was quite traumatic, particularly with it being a Sunday, meaning that I had to go to school the next day after this frenetic move.

While a good portion of this was going on, my wanderings of the internet led me into the furry fandom, a broad community of folk interested in anthropomorphic animals in various ways and to various levels. Generally an open-minded bunch, if a little dramatic, I fell right in with the ranks. I fit in quite well, being a young, gay male, and a good deal of my closest friends were made through this community, or, as in the case of ****, introduced to it. However, seeing as the majority of furs that I knew who were interested in anything spiritual, were interested in Native American or Asian mythology, both of which are rife with anthropomorphism, and the majority of furs in general were at least agnostic, if not militantly atheistic (I saw this echoed more clearly in the gay community later on, but that's later on), I kept my spiritual explorations separate from this aspect of myself, keeping all of my associations with other furs on a lighter level, and only letting loose on certain occasions, such as the move mentioned above.

This habit was likely built up out of a sort of spiritual downtime. That's not to say that my sense of spiritual self had waned, but rather that it had become tangible instead of based in words and ideas. One of the most unique experiences to come from this shift was the sense of individuality and how terrifying that can be. This was coupled with a budding sense of appreciation for humility, despite being a near-physical sensation for me. It began as a sense of how small I was in the grand scheme of things, which was made particularly evident to me by both mountains and clouds. Boulder, where I lived is right at the base of the Rocky Mountains and I grew up with those looming over me every day of my life. When my mom started to take me on hikes with her in Rocky Mountain National Park, though, I began to realize just how big the mountains were --- and not just the mountains, but the entire world --- compared to myself, and when I brought the entirety of the rest of space off earth into account, I was terrified at just how minuscule I was in comparison to everything else out there. This was emphasized whenever I'd look up at a partially cloudy day and see all the folds and corrugations in the clouds above, knowing that even they were likely larger than my entire high school --- a building large enough to house its 2,200 students and 200 faculty and staff in ten `levels'.

It was a struggle for me to embrace this idea, and I would comfort myself with other near-physical mental wanderings, such as stretching out in bed during a windy night and imagining that the wind was my body --- feeling myself flow in chaotic eddies over mountains and plains, buildings and open spaces. In a sense, not only was I making myself bigger, but I was trying to escape the confines of my body's limited range of motion, imagining the way that the wind is less of an object as a verb, as the air is not the wind, but rather the flow of air. Later in life, I'd discern this as a shallow form of a Kabbalistic exercise, a sort of synaesthetic experiences of Matt-ing. The beginning, as is said, of wisdom is awe.

One of the things that I would do when wind-ing would be to attempt to feel the others around me as a sort of empathy. A selfish empathy, of course: rather than actually attempting to feel for those around me, it'd be more accurate to say that I was feeling my interpretation of those around me rather than them as individuals. Individuality and uniqueness of perception was a concept that I'd struggled with often up until that point, and even continue to struggle with today. Seeing others as completely separate entities rather than projections from within myself is one of those tasks that sounds much simpler than it really is. Our day-to-day lives are lived from within ourselves, in a world where self and other are distinct, and interconnectedness is achieved only on the fragile and shallow level of our tacit agreement that everyone else is just a projection of ourselves onto animate objects. To actually live your life in a continuous sense of seeing others as true individuals with their own unique perspectives --- both physical and interpreted --- seems to me as having the paradoxical effect of creating a deeper sort of interconnectedness born out of true dialog between two separate beings instead of, as E. E. Cummings put it, ``all talking's talking to onesself.''

These were my thoughts at that time in my life, and my spirituality was the spirituality of fiction. In fiction, there are often deeper dialogs that ever happen in person due to the writer attempting to create characters outside of him or herself. This, combined with the fact that one of the goals of fiction is to provide a vehicle for ideas, no matter how fantastic, lead me into this incorporation of ideas from fiction into my own spirituality. The books I started reading began to have a more overt spiritual bend to them, and the ideas became more and more influential on some level or another throughout. The most readily apparent of these are Dan Simmons' novels, all of which contain some sort of spiritual or at least deeply intellectual basis. The Hyperion Cantos, in particular, proved to be an eloquent example of the importance of individuality, not only while one was still living, but after one died. Through the esoteric idea of The Void Which Binds, Simmons' offers a glimpse of what happens after death back on Earth (or `back in Life' may be more appropriate in this sense); more specifically, the importance of the memories of the dead cherished by the living. This fit in nicely with my solidifying stance on death. We don't know what happens after, but we can be proactive about the subject while we're here, cherishing the lives and keeping alive the memories of those who have passed, incorporating their gifts to us all while moving forward in our own lives --- that is, not getting caught up in the past and what can't be changed.

This burgeoning habit of looking deeper into creative works was likely one of the early influences into my own real creativity. I say real because, while I'd been creative in the past, it was always in the sense of following --- singing in choir, playing in band, writing for class. Now, however, I began to apply that creativity into more of a leadership role, as in writing outside of class or composing my own music. In this, I was leaderless and totally without a teacher, which certainly shows in my writings and music from the time, of which little remains, Needless to say, I was all over the map in terms of style and application, and I don't think that any of it shows any sense of my personality. However, it was creativity and I was doing something positive, something which might last. What I lacked at the time wasn't just a teacher or solidified direction in my creations, but the appreciation of such --- I didn't want a teacher, didn't think I needed one, something that would take a good deal more humility and a few really good teachers to appreciate, which didn't arrive until college.

I wonder if my continued attempts at creativity are a stab at immortality in the minds of others, just as Beethoven and Bach are immortal, and that, in turn, makes me wonder how to interpret that goal: is it selfish to want to live on and be remembered? It feels deep down inside that it is, after a fashion, but on a more intellectual level, it seems absurd not to want to do anything constructive, not to leave some lasting impression on the earth, with the time we're given. My thoughts and feelings on this and on music, however, are worth a chapter in their own right.
