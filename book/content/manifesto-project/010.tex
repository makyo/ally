\emph{Arguments and smooth talkin' --- Set, setting, or integral part?}

While my library of relevant books grew from the KJV Bible and the Principia Discordia, my interest in spiritualities continued to swell and, eventually, I began to read more into these different faiths. I came back time after time to the bible, however, having branched the collection out to a nice NIV copy, an Amplified copy (wherein whenever there's a difference in a source material, it's noted, and whenever there are multiple meanings for a word, they follow in parentheses), and several NKJV New Testaments from the Gideons on campus. My reasons for looking so keenly into the Bible were due in large part to the overwhelming presence of Christianity on campus, specifically in the music department.

Perhaps because it was so pertinent in my daily life in school, I was very interested in the `why' of Christianity. Why did people focus so intently on this one book, take the words written in it so seriously? I had gleaned a good bit of information about the history and concepts from Muller's The Sins of Jesus, and I had read a bit of the bible at this point --- the apostles and about half the Torah --- so I could see that there was indeed something there to be learned. My struggle, then, was to find agreement in what I saw in the actions of Christians with the dogma put forth in the Bible.

There was, one spring, a preacher out on the Plaza named Johnny Square. He had the perfect voice for a contemporary evangelical, black preacher: smooth and reassuring with an almost sing-song tone to the important words which brought them out almost as much as the long pauses filling his speech did. Also, unlike the other preachers that usually came to campus, he encouraged one on one discussion, bringing with him a couple of PA speakers, a throat microphone for himself, and a microphone on a stand for whomever he was talking to. This idea of a public 'one-on-one' dialogue was something that intrigued me, as most other preachers were content to just shout at passers by from a central location, usually surrounded at a respectful distance by a ring of students listening, rarely participating.

As I mentioned before, though, many of the people from the GLBT office were rather harsh with these preachers, and today was no exception: what began as a light argument about homosexuality as sin turned into each side throwing logical fallacies at each other mingled with insults. With this apparent stalemate, the folk from the GLBT office headed off to their classes and Mr.~Square was left all worked up.

For some reason I'm not sure of, I got up and went to the microphone. I had little idea of what I was going to talk about, other than I just wanted to make it a more constructive conversation than what had just taken place, perhaps as a means to show that not everyone from the office was so intent on attacking. Not really in the moment, I began by asking him how he was and a few basic questions more to stall for time before I brought up the idea of love in homosexual relationships. While I'm sure we talked for about half an hour or forty-five minutes, I really don't remember much about the conversation except that, at one point, I mentioned that I would be willing to go to hell for the love that I've experienced in this life, to which the preacher responded, ``Hell is the place where Jesus Christ is completely separated from you and absent from the whole of your existence.''

This was, by far, the gentlest description of hell I'd heard, though depending on whom you ask, possibly the most devastating. Our own conversation reached a gentler stalemate soon after, though it was not without a few pieces of scripture --- the standard statement from Leviticus regarding homosexuality included. Certainly not as exciting as the previous discussion, ours left us both feeling a little lighter, and he offered to meet with me over lunch the next day, though our conversation was rather shallow over that.

What I took away from this experience was a few bits of confusion that I'm still thinking about today, all surrounding the definition of Christianity. Granted, such a thing is quite subjective and will change depending on whom you ask, I was left wondering about the connection between Christianity and Judaism. The two are obviously connected --- the first five books of the Old Testament are the Jewish Torah, and, with the rest of the books in that collection, part of the Tanakh, the collected writings which, along with the Talmud and Midrash, serve as the basis of the religion. Jesus himself was a Jew, and the Jews played a major part in the story of his life.

Separating the two, then, becomes a problem. There are a few obvious differences in teachings between the Old and New Testaments: in the former, God is shown to be quick to anger and, in his own words, ``a jealous God;'' while in the latter, he is put forth as a loving abba, or father figure. In Judaism, God talks the people of Israel through prophets, of which there are many, and many instances of groups of people prophesizing, while in Christianity, God is said to be manifest in the form of Jesus (basically --- different denominations, different views on this), making Jesus more than just a prophet. Also, prayer is left to the individual, and, as a consequence, there are less in the way of prophets, not to mention the priest caste that had existed before.

Another difference in the two is the amount and presentation of rules. It is said that there are 613 rules in the Torah that Jews must follow, and they are stated plainly, along with consequences. In the New Testament the rules are muddied and indistinct, though there are certainly commandments, and many of them show up not only in the form of parable, but only appear later in the writings of his followers, such as Paul. This, of course, brings into question the sources for each of these two traditions: for the older, there are the words of God brought to the people by way of the prophets, and in the latter, God spoke directly through Jesus, and the rest, to paraphrase Rabbi Hillel, is just commentary.

These differences lead to the question of how does Judaism (in the context of the Old Testament) factor into Christianity? In the culture at the time, it would be easy to see Jesus as the next prophet, taken from an outsider's perspective --- an insider, of course, having the miracles on his side. With Jesus being a Jew in a Jewish culture, it's easy to look at it that way, but obviously, things have changed --- Christianity is now seen as a separate entity from Judaism, and most Jews certainly don't consider Christians to be Jewish! With its focus on the Israelite community (the oft-quote Leviticus 18:22 is followed with, in the 29th verse paraphrased, ``Whoever commits these acts will be cut off from the people''), what then does this mean for Christians who use this --- obviously a cultural and spiritual influence in Jesus' time --- to condemn people today? Yes, in a later verse (Lev 20:13), it does say that the person who commits this act (a man laying with a man) is to be put to death if they're in the house of Israel and defile the Lord's sanctuary, but how does this fit in with today's Christianity? I honestly am not sure whether the Old Testament is intended as the predecessor and basis behind Christianity or if it is actively considered part of the teaching. It seems to me that it depends on the Christian, and many opt for a combination of both --- using portions such as those listed above as active principles in their faith while the others are simply set-and-setting for Jesus' life.

Even within the New Testament there are things that can be applied both as active principles and set-and-setting. For example, how does one deal with the concept of witnessing? The `against the hypocrites' chapters in Matthew, the sixth and seventh, would seem contrary to what a lot of Christians do, but even later books, the Pauline Epistles in particular, seem contrary to this. Witnessing, it seems, should be done on a one-on-one basis with quiet humility according to what Jesus said, which seems contrary to the shouting preachers on the plaza, condemning us all to hell and praying before us. Perhaps this is why I enjoyed Johnny Square so much more than the others. What he held was more of a public dialog between him and one or two students at a time to talk about the issues at hand, rather than to make a spectacle of witnessing.

These explorations are still new to me, despite having thought about them for so long now. I'm sure that answers will come to me in time and will bring with them all new questions. For now I'll have to keep reading, and perhaps one of these days I'll pluck up my courage again and talk to someone on the other side of the situation. I'm curious to see how both Christians and Jews feel about this issue, and I'm interested to see how they'll react to being asked such a question.
