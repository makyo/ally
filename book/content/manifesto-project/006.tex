\emph{Tee hee}

As a brief vignette, humor has always been important to me. It struck me, sometime in high school, that there wasn't any `real' humor in religion, though. There are plenty of edgy comedians that make fun of it and jokes abound, sure, but within each religion, there's very little to be had in the way of direct humor. There are, of course, exceptions to this rule: Unitarian-Universalists do tend to have a bit of a self-deprecating sense of humor (most of their jokes involve their reliance on committees, coffee pots, or copy machines), and I've seen some really subtle humor in traditional Jewish teachings. Rarely, however, are religions outright humorous.

Well, except those that are.

One of my friends on the internet --- a furry, of course --- introduced me to Discordianism sometime around late freshman year, and I thought that I had finally found a religion I could take seriously. Discordians have a creation myth, a curse to lay on others, a system to live by, apostles, and even a church sanctioned game. The catch, of course, is that none of this is intended to be taken seriously. Basing their deity on the minor goddess of Grecco-Roman mythology, Eris, the goddess of chaos, the Discordians have built up either one of the more elaborate jokes or one of the least elaborate religions in the modern era. Despite being popularized through not only their holy book, The Principia Discordia --- or How I Found Goddess and What I Did to Her When I Found Her, but also the writings of Robert Anton Wilson and Robert Shea in the books of the Illuminatus! Trilogy, the number of serious Discordians is still quite small, and despite that, the church is very fractured, what with every member being a Pope and several of them running their own Cabals. So it was, with humor as a major factor in the religion, I declared myself a Discordian Episkopos and leader of my own `Qabal', the Qabal of Ranna I.

Despite the fact that the majority of the religion is a joke, I did take several things from Discordianism worth mentioning. As mentioned, the deity in question, Eris, is the goddess of Chaos, and the Discordians do take their Chaos seriously, or as seriously as a Discordian takes anything. While most of that is for comedic purposes, there are good points about chaos that need to be brought up when talking about religion and spirituality. Several of these valid points stem from the Discordian's argument that most religions point to all the order in the world and proclaim it the work of some Deity or another, handily ignoring all the chaos inherent in nature. After reading the Principia as well as a few pertinent science fiction books and actively spending a while pondering Chaos in the world, not only am I inclined to agree, but I find that I'm more inclined toward that chaos than toward the order. That is not to say that order has no place: ``To choose order over disorder, or disorder over order,'' the Principia states, ``is to accept a trip composed of both the creative and the destructive. But to choose the creative over the destructive is an all-creative trip composed of both order and disorder. To accomplish this, one need only accept creative disorder along with, and equal to creative order, and also be willing to reject destructive order as an undesirable equal to destructive disorder.'' A fine point, I believe, and something I have integrated as an active principle in my life.

The Church of the Subgenius is Discordianism taken several steps further. What was at first humorous is now intentionally absurd, and where once was disorder is now active strife. The Book of the Subgenius is filled with clip-art, a veritable collage of propaganda posters, diagrams, nonsensical text, and repetitive references to their deity/prophet/ruler J. R. ``Bob'' Dobbs. Their rituals seem to consist of getting drunk and holding devivals, and possibly some waxing poetic about meteors bouncing around inside the Earth. I took nothing from the Subgenii, excepting perhaps a bit of skepticism --- their humor is simply over my head.

During my senior year in high school, several friends and I, all interested in the more esoteric and unique traditions began to get together to discuss such traditions from serious to humorous (they had all heard of and participated in Discordianism), and, at one point, even became a school-sanctioned group, though we were only just barely tolerated --- Prayer at the Pole, on the other hand, was, of course, embraced wholly, which certainly got on our nerves at the time. Once we started advertising, we did hold a few successful true meetings, the most memorable of which involved the various methods of divination in use around the world, or at least those allowable indoors. While Dan spun in circles until he fell down --- gyromancy: his landing would determine the answer to a question --- Toren read tarot, and I conducted crude numerological explorations with a book by Aleister Crowley. Mostly, however, we would just laugh a lot and talk about various odd things about this religion or that cult. I would post 'propaganda posters' consisting of images and phrases from the Principia Discordia and my own contrivance, stamp any poster I saw in the hall with a self-inking stamp which read ``APOTHEOSIS APPROVED'' (for which I got in trouble), and even hand out Pope cards. This was my attempt at adding creative chaos to an otherwise dreary school atmosphere: the prime example of order both constructive and destructive in the world.
