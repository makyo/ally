Relationship anarchy, as a topic, seems to draw heavily from both poly folks and queer folks. In fact, the three ideas are so heavily intertwined that it's difficult to have one without the others. Poly? Well, there's a good chance that there are some queer aspects to your relationship.

And if you're queer and at least of a certain age, relationship anarchy is baked into your soul. If your society sets up a ``natural'' relationship progression and then bars an entire class from entry to that progression, subversive and transgressive relationship structures form as a matter of course.

\begin{quote}
Queer people, queer relationships.
\end{quote}

Yes. June, 2004:

Queer hair, queer mouth, queer brain, queer sleeves, queer shoes, queer toes, queer nails, queer fingers, queer palms, hairy palms, queer wrists, limp wrists, queer arms, queer shoulders, arms around shoulders, queer neck, sensitive neck, queer hair, curly, queer ears, sensitive ears, eargasmic, queer cheek, blushing cheek, queer nose, got it from my dad, queer eyes, queer colors, got them from my grandpa, queer eyebrows, but not as queer as some, queer face, too long, queer chest, too skinny, queer belly, padded, queer crotch, go figure, queer thighs, better believe it, queer knees, queer calfs, queer ankles, queer legs, flexible, queer feet, still smell, queer guy, no surprise.

When you're queer, \emph{being queer} is baked into just about everything about you, but most especially in your relationships. ``Minority identity acts as a force multiplier on social dynamics,'' as Orrery put it.

\begin{quote}
And so?
\end{quote}

And so, being hopelessly queer, I wind up in relationships that are hopelessly queer.

\begin{quote}
Except when you don't.
\end{quote}

Yes. And when I don't, there's such a fundamental mismatch of understanding that I feel uncomfortable in my own skin.

Something that queer relationships miss, or at least reconfigure to their own ends, is the relationship escalator, that heteronormative idea that one gets on at the ground floor of friendship and gets off at the top with marriage, or one can stop off at any of the other floors to stop for a while, or to step off entirely when the relationship ends.

It's not a bad idea, either. It's not as old as some would have you think, but in today's society, it works quite well.

\begin{quote}
Does the divorce rate agree with you there?
\end{quote}

Is that just another step on the escalator?

\begin{quote}
Touché.
\end{quote}

In nonheteronormaitve relationships, the idea is muddied. The friends-dating-marriage-children set of steps, originally shattered whe marriage was made illegal and adoption banned for large swaths of queer folks, just doesn't fit. The barrier between friends and dating, as well as between dating and permanent relationship, is thin, osmotic.

\begin{quote}
Suddenly, you're in a relationship. Suddenly, you're saying ``I love you.''
\end{quote}

Yes. Suddenly, organically, though not for lack of deliberation. There's much talking, if everything goes right, much working out of boundaries. It's just that there are fewer milestones.

\begin{quote}
Why do you bring this up? You're not writing an article. Out with it.
\end{quote}

Right.
