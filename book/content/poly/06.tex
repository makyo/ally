I've been married for seven years. Robin and I have been together for more than five. My polycule has grown steadily over the years, and I have to wonder: how much of my polyamory, my relationship anarchy is a coping mechanism for how I was raised?

\begin{quote}
Does it matter?
\end{quote}

Yes, I think it does. \emph{Early on, I promised myself that I would do anything to not become my dad,} I said. I wanted to stay away from serial monogamy. I wanted to talk more and perform less within my relationships. I wanted to be an improvement upon what I saw growing up.

If I'm poly because I'm coping for my past once again, have I really grown? Or have I fallen into the trap just on the other side of the path?

If I'm coping for my childhood, what would I leave my children coping with?

\begin{quote}
Again, does it matter? You must walk a fine line between the selfish and selfless when working with reality. In order to be happy, you need to not repeat the past, as you've said --- a selfish act. But worrying about counterfactuals with non-existent entities, being \textbf{too} selfless in this, will only set you back in your own growth.
\end{quote}

Perhaps I'm worried that if poly and such are just coping mechanisms, my relationships might be somehow less real, less earnest than if they weren't. Perhaps I'm worried that I'm doing a disservice to my partners by using them to overcome my own failings.

\begin{quote}
This is impostor syndrome, not using people. No relationship is perfect, all that matters is that you're approaching these honestly, earnestly, and with your whole heart. Even then, there will be friction occasionally. Your parents gave you stuff to cope with, and you would give your children stuff to cope with too.
\end{quote}

Guess it's a good thing I don't have kids.

\begin{quote}
Let's talk about kink.
\end{quote}

Oh my \emph{god}.

\begin{quote}
Alas, had I a face, I would be able to smirk. Imagine that for me, will you?
\end{quote}

You know what? Now's as good a time as any.
