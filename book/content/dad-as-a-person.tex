\label{dad:as-a-person}
\index{Family!dad|(}
\renewcommand*{\footnoterule}{%
  \kern-3pt%
  \color[HTML]{ccccdc}\hrule width 0.4\columnwidth
  \kern2.6pt}
\fontspec{Gentium Book Basic}[Color=CCCCDCFF,Ligatures=TeX]
\renewfontfamily\allyFont{Merriweather Sans}[Scale=0.9,Color=BBBBCBFF,Ligatures=TeX]
\begin{quotation}
  Dad,

  It's been a while since we've had the chance to catch up on things. That's on me; not only has life been pretty nuts of late, but I've also kind of lost track of keeping in touch with family and a whole slew of friends.

  There are a bunch of reasons for that. Chief among them is probably that I'm struggling a lot with figuring out where I stand with folks. It seems like there's this whole class of people that I'm just not sure how to interact with. In our case, it's sort of, ``Are we friends? Are we family? Is it cool for us to just chat? Should our relationship be cordial? Friendly? Chatting only when necessary, or regularly?'' Lots of questions like that.

  I think a lot of the reason I've been asking myself a lot of these questions lately has been that I've kinda hit one of those mid-life crisis moments. I know 34 isn't exactly a number preceded by `the ripe old age of', but I suppose this is the type of thing that can strike at just about any time. At least, that's what my therapist promises me.

  I burned out pretty hard at Internet Archive, and left after only a year to go work as a contractor for a small software company based in the UK (still working remote, natch) called New Vector. They work on encrypted communications stuff, with their primary selling point being that their service is federated --- anyone can run a server and talk to anyone else on other servers. It makes for a much more robust network.

  Neat as the opportunity was, I hated it. Every time I opened up my code editor, I'd just stare at it and think about how much I hated my job. Then I'd start feeling hopeless, because this thing I was hating was my chosen career path.

  Burnout's a hell of a drug, I guess.

  Neither work nor I were happy with me there, so rather than renewing my contract, I decided to start looking elsewhere. Rather than looking for yet another software job that I'd probably hate, I started looking at tech writing positions. It'd be a lot of working through a piece of software --- both using it and looking at the code --- and writing documentation, blog posts, etc. My biggest lead right now is actually for a company I used to work for, helping to write the curriculum for their certification program, similar to Microsoft's A+ cert.

  I've been writing and editing a lot lately. I've got a small publishing publishing company that I run (very small; only have three books out so far), and three books of my own out, with another one coming out in a few months. I figure since that's the direction my hobbies have gone, might as well find a synthesis of that and the thing I'm good at in terms of dayjobs. Tech writing sure as hell makes more money than publishing, after all.

  Things are going alright on my end other than that. Found a meds combination that is working really well for bipolar (and doesn't cause any more of those movement disorders!), and a hormone regimen that's been stable for a few years now. We went down to San Jose, CA around my birthday for a convention and to meet up with some of our polycule (if you graph them out, polyamorous relationships start to look like molecules, so the name has stuck). Was good to have a little vacation.

  James is doing alright as well, though he's moved to working almost entirely with property management and real estate these days, rather than machining. He's been working through some health fiascoes. Found out he was low on testosterone, and supplementing that helped out a ton. He was back to the James I met back in 2005 or so. Then he found out he has celiac disease, so we had to go gluten free. Now he's got twice the energy he used to, since his body is actually digesting nutrients.

  The dogs are both slowing down. They're getting pretty old (at least for German Shepherds), and both have arthritis. Still, they're happy and lazy. It seems like a good life. We also adopted a piece of shit cat, dumb as dirt and soft as hell. I love her.

  How are things on your end? Been a bit since we've caught up about the day-to-day stuff. Curious to hear how work is going. How's Maurine?

  It's a bit early yet, but happy upcoming birthday! Hope it treats you well.

  Love,

  Madison
\end{quotation}

\newpage

\begin{ally}
  Why?
\end{ally}

Why what?

\begin{ally}
  Why send this? Why email your dad? Why now?
\end{ally}

This project, mostly.

\begin{ally}
  My fault?
\end{ally}

Well, maybe the book's. The possibility that he may wind up with a copy.

I talk about my dad off and on during therapy. I suppose he comes up with some frequency because of all the hangups I still have. It seems like ever few months I'll discover a new one.

\begin{ally}
  Ain't that just the way of things.
\end{ally}

I think it's a credit to my therapist, honestly. Were I paying all that money to simply go chat about my week with someone, getting nothing out of it but company, I'd feel quite let down by the whole process. That I'm coming away from sessions with improved understandings of myself is a good thing.

That said, a lot of the time those therapy sessions where dad has come up have been productive mostly for me understanding the present through my past without necessarily moving forward.

\begin{ally}
  Do you blame your therapist for that?
\end{ally}

Of course not. She's wonderful, and has helped me out a ton.

I just also think that she's got a different approach to this than you do. Or I do. Whatever.

\begin{ally}
  Whatever.
\end{ally}

On her end, she is happy to help me explore and offer suggestions, but she's less keen on beating me up. She is an ally, yes, but a bit more of a friend than you are. She is happy to help me move forward, but also happy to let me just learn.

\begin{ally}
  ``I think at some point I just need to accept that it's not worth the trouble trying to reconnect with him,'' you said.
\end{ally}

Yes, to which she responded, ``I suppose that's true, though is that something you'd recommend others who are transitioning?''

``Yes,'' was my immediate response. ``At some point, with family, it has to be okay to make the cost-benefit analysis and decide whether it's even worth it to keep trying.''

\begin{ally}
  And did you make that analysis?
\end{ally}

Yes.

\begin{ally}
  And was it worth it?
\end{ally}

No.

\begin{ally}
  So, why the sudden change of heart? Why now?
\end{ally}

That Madison --- the one who struggled to square living earnestly with lying to dad --- is dying. She may have died already. Maybe she died on August 9th of last year, when she first decided to summon her ally.

\newpage

\begin{quotation}
  Madison

  What a nice surprise. Thank you so much for all of the information and insights as to how you have been and are doing. I loved it!

  You happened to catch me down in Tucson. I come down here by myself when Maurine is working just to get used to working remotely with my work crew. It's a bit clunky but the VPN and various tools make it doable. Hope one day in the not too distant future to be able to come down to Tucson for a few months over the winter months and work. I'd have to go back a week a month for meetings but otherwise I should be able to pull it off.

  Without being maudlin, I will always love you and am proud of you and your life. There are no thoughts here but good ones and hope that you are comfortable with our relationship. I'd love to hear from you more but know how life can get in the way. Maurine and I both had a great time seeing you two over Thanksgiving. I know the dinner was a bit over the top but I still think about the visits then. I hope to get out to Seattle again later this year and visiting you was on the top of my wish list. I often think of the times we both went through while you were growing up and I have to smile at the fun we had. Hopefully there is more ahead. You will always be a part of me.

  I know the burnout feeling. I can start to feel that creeping into my work routines. The clients seem to be more demanding and the work more of a grind. Luckily I have two employees that pick up a huge amount of the burden now. I hope to slowly turn much of the day to day stuff over to them. The problem is that Greg is still around and does little if any work. That salary stream keeps me from picking up the additional employee that I need to really step back and relax. Anyway, we have paid off both the Lakewood and Tucson houses so the slow retirement plan is starting to look like something that can be done. Now I just need to learn how to value my self-worth without it being tied to the company.

  Overall I still am pretty healthy. I am getting over a stomach reflux problem that was probably related to stress and my getting high. Got both of those sources under control and picked up my exercise routine. That has helped quite a bit. Only smoke a couple of hits at night now and that's it. The exercise also seems to help the hand tremors that I have at times. The doctor thinks it was related to anxiety but the drug they prescribed did not go with my life. So I continually to learn to relax and take things easier. You'd think I would have learned all of this by now. Life can be a squirrely thing.

  Maurine is doing well and is probably closer to a retirement change than I am. They made her the shop teacher at the school so that has given her a new lease on work but she is getting tired of that also. The kids are not what they used to be. They talk back and argue with her constantly and many are really rude. She is lucky that she hasn't lost her cool and slapped the shit out of one of them so far. As a result, she is going to let her teaching certificate expire next year so she has about a year and a few months left to work. I think she will probably become a substitute teacher and work part time. She will also be coming down to Tucson more. I told her that you wrote and she wanted to make sure that I let you know she says hi and is looking forward to seeing you again.

  A publisher --- Who'da thunk.

  Love Dad
\end{quotation}

\newpage

\begin{ally}
  Is this what you were expecting?
\end{ally}

Not at all. Or perhaps some very small part of me was hoping for something like this, but it was one of those 'hope against all hope' type things.

\begin{ally}
  What were you expecting?
\end{ally}

I suppose I was expecting something along the lines of what I got after my dumb-as-hell coming-out letter: an acknowledgment of receipt and thank you for the information. Perhaps I was expecting a phone call in return, and I'm not sure whether that would be better or worse than a response, no matter how curt.

Were I to get a call, I would have frozen up and not been able to talk about anything of import.

\begin{ally}
  And so what does this mean?
\end{ally}

I suppose it means a few things.

It means that I was spending rather a lot of time catastrophizing. That I spent all of my time defaulting to the idea that he was somehow unwilling to engage with me on a very real level may have been informed by times in the past, but clearly is not the default.

This, in turn, means that I need to somehow reorganize my conceptualization of my dad around this new version of reality. I was holding this picture of him in my head that was based solely on those times with him that left the strongest impression. My view of him was limited to the man I ran away from juxtaposed against the man who was finally able to interact with me on an equal level when we were able to drink together. It was not based on an interpretation of him as someone who was constantly improving --- constantly striving to improve --- and who, yes, may have been able to interact with me better as an adult but who nonetheless enjoyed the fact that I was his kid.

\begin{ally}
  And?
\end{ally}

And it also means that there is far more that my dad doesn't know about me that I had first imagined.

\newpage

Does my dad know that I'm trans? Does he truly, \emph{truly} know? Does he accept it?

Does my dad know know about HRT? About surgery?

Does my dad know I'm poly? Is that something he has internalized?

Does my dad know about self-harm? Does he know about suicide? Has he seen the scars?

Does he know about you?

\begin{ally}
  Does it matter?
\end{ally}

The joy that I felt at his response is tempered by a whole new set of anxieties.

\begin{ally}
  Did you feel joy?
\end{ally}

Honestly? Yeah.

It was a relief, in a way to see that he was not the dad I grew up with. That I could see change in him is not only something that's good for our relationship, but also something that makes me feel better about myself. It makes me think that I, too, have the ability to change, to grow and become a better person.

\begin{ally}
  Was that in doubt?
\end{ally}

Yes.

\begin{ally}
  Really? Given this project? The core theme of the death of Matthew?
\end{ally}

Oh yes. So many times when I was writing about that, it felt like I was writing about someone else. I feel so stuck sometimes. So static. It's easy to lose perspective until it's rubbed in your face.

\begin{ally}
  Will you talk to him about your anxieties?
\end{ally}

Yes. After hearing back from him, I think I probably should, too.

Just over time.

Slowly.

Carefully.

\begin{ally}
  Take your time.
\end{ally}
\index{Family!dad|)}
