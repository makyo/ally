\end{leftcolumn}
\begin{rightcolumn*}
  \begin{flushright}
    \emph{December 29, 2013}
  \end{flushright}
\end{rightcolumn*}
\begin{leftcolumn}

\noindent First of all, let me state that I'm feeling pretty good as I write this. I feel the need to state such because a lot of my tweets and a lot of my previous entries could be construed as worrisome, and probably legitimately so, because I have the tendency to vent freely. If I feel bad, I write, and if I'm not at a computer, sometimes that ends up on Twitter. It's never my goal to freak anyone out, so much as to simply cope with what's going on. Writing, putting things in words and stringing those words together into some form meaningful to others, is a good way for me to cope with what's happening in my life. That said, although I try to be frank about symptoms, I know that some are disturbing taken at face value or to their logical extremes, so I promise: I'm feeling pretty good now!

I'm torn.

I feel as though one of the most important things in my life is ritual, process, or repetition. It's not so much that these things are comforting in isolation, as that there is a certain feeling of being tethered to reality in them that comforts in its own way.

I've been asked what I mean by reality, or what I mean when I say ``that makes me feel real'' or ``it's important to me that I feel real''. A lot of my response must, by necessity, rely on analogy, by its very surreality - there's no way I can describe how I feel without using metaphors and similes.

In short, it's part of life that we sort of perceive the world around us as a spatial, temporal thing. There are three axes of movement, one axis of time (though sometimes it gets a little twisted up), and that's just sort of how we interface with much of the world. The feeling of surreality, then, is a pulling away on some fifth dimension, a cocooning, a means by which one has or has been made to withdraw from the rest of reality. From the inside, it feels like being wrapped up in cotton. Senses aren't dulled, as that might imply, so much as that all connections through reality, all input must pass through a high-latency barrier that introduces its own artifacts, requires its own decoding. Again, it's not that I can't \emph{hear}, it's that the words that are coming in must be run through an additional filter to associate them first with meanings, and then to tie them back through the perception of reality (the rest of which must, of course, go through its own decoding process).

This surreality is, of course, nothing more than anxiety. I talk often in terms of bandwidth, and that's rather applicable here. If I am spending all of my emotional and intellectual energy on cycling over counterfactual universes that I've constructed in my consciousness, then I have little energy left to deal with the one I'm actually living in. My doctor insists, and I heartily agree, that I not think of this as anything other than anxiety and panic, which I'll get to in a moment.

I said that I'm torn above because the result of this is a desire to get back to reality. The problem is that the anxiety gets in the way quite a bit. I think, ``There must be a way back to clarity and reality, there has to be some sort of path or action I can take.'' That, too, is anxiety, but it's as yet too subtle to recognize as such unless I'm holding still and doing very little else (which is hardly productive).

As a result, a lot of my day-to-day life is spent focusing on the idea of ritual. Ritual is the one thing that my mind has latched onto as some sort of way through or way out, and I think it plays a large role in the events of my past, though I was less conscious of it at the time - such is life, when it comes to any sort of personal advancement. I ritually check the stove to make sure it's off. I check the doors and windows. I get up once a night and check on JD and the two pups to make sure they're inside (just in case Falcon has rappelled out the window and is terrorizing the neighborhood - seriously).

It's not just checking that drives me, though. Anyone who has been to my house knows that it's not cleaning, of course, but, well, it all comes back to the audible aberrations that I'd mentioned before.

For a few months now, I've been `hearing' voices, but I'm always careful to mention that they're not audible hallucinations. They're not. They're what's called expansion: the inner dialog that goes on in our brains as we go about life is usually one that takes place in abstract images. In this case, however, that has broken down into something more simplistic, as though I'm telling myself a story. The voices have character and gender (though they're usually boring), and hover \emph{just} below the level of hearing, something closer to remembering that I had \emph{just heard} someone say something.

It's fantastically hard for me to write about this in any sort of open way. I want to hide it. It's fucking ridiculous. I hate it, and I want it gone, and it's embarrassing. Embarrassment is, however, a primarily social reaction, and a harmful one in this case (after all, this is a health problem). That is, more than I want to hide all of this, I want to tell that embarrassment to get fucked and talk openly and freely about all this, because it's even \emph{more} ridiculous that I feel I can't.

Anyway, as I listened to someone drone on tonight about how I should cut my hair off, how it would hurt in just the right way, how that would be my penance, and that would be just what I needed to gain touch with reality again, I think I finally understood the tie to ritual. This was all I had to do. In fact, this was all these stupid aberrations were ever `urging' me to do. It was this sense of ritual become words. When I feel as though I'm instructed to tease apart my skin like burlap cloth with a knife-point, to solve a cramp or a gas-pain with violence, to kill myself before an upcoming trip to London, that's not just an expansion of some random, totally out there thought, that's the feeling of ritual, the ``there must be something I can do to stop this panic'' sense expanded from an abstract concept back into language.

I've been shifting wildly along the spectrum of following these rituals to the letter to outright ignoring them. As I said, I feel good: I'm not going to kill myself before London or stab myself with a syringe to ease gas-pains. However, I'm still getting up to check on the windows and doors and stove and dogs. In the middle, I've taken to trying to subvert the desire for ritual with other rituals: rather than tease apart my skin like lose-woven cloth with the tip of a knife, I use a pen and just kind of draw on myself. It offers enough catharsis for me to get to the point to realize that it's actually really, really ludicrous; that I'm drawing symbols or lines of the utmost importance on my limbs with a pen pilfered from my bank. That's usually enough to break through the panicked ritual and leave me just feeling silly (which is, while uncomfortable, still a million times better than that inner tension that required the ritual in the first place).

Ritual is a salve. It's an ice cube held against a burn. It's something that provides instant relief, but only so long as it's present. I can't \emph{solve} any of these problems by acting out a ritual. Checking on the dogs does not ultimately leave me satisfied that they're all comfortably asleep, because then I need to make sure the windows and doors are shut to ensure that they don't float away. That done, I need to check the stove to make sure that it's off, because if it's on and the windows are shut, how will we escape when the house burns down?

You see, there's no solution. There's no ritual to make me feel good, or real, or better, or not-anxious. There's only anxiety, and coping, and panic, and sleep. There's reality, and that's where I dwell, and then there's my perception of reality, which drifts rather more than perhaps it ought. Cutting my hair wouldn't hurt - it's hair, for Pete's sake - and it would not be the penance I need, the right amount of pain to bring me back to reality. It's hair! I know that. That's the case I argue to the voice demanding such. That's what makes it panic, and not psychosis: ultimately, there is no break from reality. There's none. I know these aberrations aren't real; I know the dogs aren't going to go carousing out the windows; I know, for sure, that cutting my hair is not going to stop any of this. I know it. The voices are a nuisance, the panic is a problem, but it doesn't control me. There is \emph{no} ritual that will solve anything: the ritual is a symptom. It's important, yes; I live my life by process. But it's a symptom.

That's why I'm torn.
\newpage

\end{leftcolumn}
\begin{rightcolumn*}
  \begin{flushright}
\emph{February 13, 2014}
\end{flushright}
\end{rightcolumn*}
\begin{leftcolumn}

\begin{quotation}
\noindent I wonder if the snow loves the trees and fields, that it kisses them so gently? And then it covers them up snug, you know, with a white quilt, and perhaps it says, ``Go to sleep, darlings, till the summer comes again.''

--- Lewis Carroll
\end{quotation}

I've mentioned ritual before, but I think that's tied into the larger feeing of portentousness. Ritual is one way to sate that sense of intense meaning surrounding an act or an object.

\begin{quotation}
\noindent A goose is dumb. A thousand geese darkening the horizon is a portent. Mindless honking, individually directionless, collectively unstoppable

--- @drab\_makyo February 12, 2014
\end{quotation}

Any little thing can carry meaning for one person far outweighing what it might mean to others. Something about flocks of geese terrifies me. It's not a logical fear, it's a sense of foreboding. It's not the geese themselves, it's the concept of geese, the lack of any ritual to solve the problem of geese.

\begin{quotation}
\noindent A goose is tasty. Geese taste like horror. Acrid tang of ill omens \emph{froth}

--- @drab\_makyo February 12, 2014
\end{quotation}

It's dumb. Geese are dumb. There's no reason I should feel any sort of emotion at all surrounding geese, but I do.

\begin{quotation}
\noindent Why are geese so portentous? Why do they cause anxiety? Did I take my meds this morning?

--- @drab\_makyo February 12, 2014
\end{quotation}

Ritual is like that. There is some level of meaning that's inexpressible except if you can find a way to come at it from the side. Use words like `portent'. Describe it as an odor, a sense, a mystery. Ritual and sensation are wily and wary critters that want nothing less than to be identified, pointed out, made plain. You're supposed to just go along with the ritual and accept the portentous as fact.
\newpage
