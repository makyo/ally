\fontspec{Gentium Book Basic}[Color=DCCCCCFF,Ligatures=TeX]
\renewfontfamily\allyFont{Merriweather Sans}[Scale=0.9,Color=CBBBBBFF,Ligatures=TeX]

\begin{ally}
Tell me about rape.
\end{ally}
No.

\begin{ally}
Talk in circles around it, then, and then tell me why you won't tell me about it. Or vice versa. I don't care. I'm not picky as to the order.
\end{ally}
Fine.
\newpage

\noindent Let's say, as we have already, that you spend much of puberty up in your head, and then when you start branching out into engaging sexually with others, you do so in a purely intellectual way. One which involves some sort of platonic ideal of sexuality. You never feel awkward. Everything always just works.

Let's just take that for granted.

Let's also take for granted that this mechanism of interaction is one wherein getting out of a sexual interaction that is uncomfortable, or pressured, or hell, even nonconsensual is a matter of just\ldots{}stopping. Come up with an excuse. Come up with some lie. Eschew the truth in favor of making the other person happy, as you would your father.

\begin{ally}
That's not possible. Being pressured into typefucking is just as easy as it is to be pressured into sex in the embodied world.
\end{ally}
I'll agree with that. Take it for granted, then that this is what you believe. You believe that consent is implicit in the act, because to revoke consent is as simple as signing off or pretending that your parents walked in on you.

\begin{ally}
Okay.
\end{ally}
Now take the type of person who takes all that for granted, and put them in a situation with someone who has an overbearing personality, who gets what they deserve, and who deserves you. Take that type of person and put them in a situation where sex is expected of them.

What do you suppose happens?

\begin{ally}
The topic at hand.
\end{ally}
Yes.

Now, what do you suppose happens to such a person who gets taken advantage of, who winds up in a situation they shouldn't be in, who gets raped, and then put them out into a world full of sexual people, where it is expected that one be sexual.
\newpage

\begin{ally}
Do you think that you are asexual because you were raped?
\end{ally}
No.

\begin{ally}
That was quick.
\end{ally}
No, I can promise you that, if there is a simple cause for me being ace (and there emphatically isn't), it's my reliance on TS. I found sex confusing, baffling, and kind of gross long before I had my own little struggle with consent.

Being ace, being autochorissexual, even if I didn't have the words for it, even if I didn't believe in such a thing, even if such a thing couldn't possibly apply to me, was the case from the very beginning of my embodied sexual interactions. It was the case from the very beginning. It was the case from when I lost my virginity, however slippery the concept is.

\begin{ally}
Ah yes, was it the first time you masturbated with someone? Was it the first time you had oral sex? Anal?
\end{ally}
Life's complicated for a gay boy.

\begin{ally}
So much easier for a trans girl.
\end{ally}
We've been over that.

\begin{ally}
Fair enough. Do you think that being raped prevented you from coming to terms with your asexuality?
\end{ally}
I think so, yes.

\begin{ally}
Less quick.
\end{ally}
It's unclear to me. It's something of a new thought I've had lately. Was part of what kept me struggling and striving to have a healthy sexual existence due to me trying to overcome this aspect of my past?

Beyond that, was TIASAP me accepting that I wasn't succeeding?

Perhaps.

\begin{ally}
Perhaps. Perhaps you needed exposure to a certain level of knowledge surrounding identity before you could truly accept it. Perhaps you needed to circle around it like you're circling around the event at hand. Perhaps you needed to side-eye it, because looking at it directly would surely blind you. It was too bright. It was the wrong color, some impossible shade of blue. It made your head hurt and your gorge rise.
\end{ally}
Perhaps.
\newpage

\begin{ally}
So why \textbf{are} we talking circles around it?
\end{ally}
Because, at some level, the experience itself is unimportant. I was young, I was dumb, he was an asshole.

What \emph{is} important is the ramifications. What is important is the fact that I have to live with the person I became when I was disabused of all of those silly, romantic notions of implied consent and this strange idea that I could just stop an act, even if it meant lying.

\begin{ally}
Lying always worked so well with your dad, did it?
\end{ally}
No, and now I was finding out that this was the case in relationships beyond just typefucking. It made me realize, on some level, how superficial my interactions up until this point had been. I had gone from being the type of person who believed she was living an earnest life with earnest people, enjoying deep relationships, falling in love.

\begin{ally}
Were you not?
\end{ally}
Perhaps I was on some level, but I was missing this key component: that my actions have consequences.

Not that I'm blaming myself for what happened, of course. I was young, I was dumb, he was an asshole, after all. But non-action is still an action. Not saying no was still an action. Being unwilling to learn about the fact that my actions have consequences was an action.

It called into question how passive I had been in the past. It called into question how little I had been saying no in the past. It called into question how little I had actually learned about how the world worked.

\begin{ally}
``Coming to terms with being a terrible person,'' you wrote.
\end{ally}
Yes, and I wrote that in the thick of this realization. At that point, I was coming to terms with all of these things, the passivity and the willful ignorance.

I was coming to terms with how much I was hurting those around me, and just how much I had to learn.

\begin{ally}
And boy howdy.
\end{ally}
Yeah. I would continue to hurt those around me for years. I still do. I'm getting better, though. I'm willing to learn, now.

\begin{ally}
``I cannot possibly bow low enough, I cannot possibly apologize with enough sincerity to make up for the hurt I've caused you,'' you wrote.
\end{ally}
Yes. And I stand by it.

I have much to learn, but I've come a long ways from who I used to be.

The specifics of what happened aren't really important. What is important is the moment before, and the moment after.

\begin{ally}
The blackbird whistling, or just after.
\end{ally}
\newpage
