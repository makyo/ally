Cathleen Schine writes in \emph{The Evolution of Jane}:

I resented the state of childhood wonder. It was insatiable, yet it seemed to me to be no more than a puerile affliction, like baby teeth. My ignorance struck me as a bizarre anomaly, for I felt, with utter certainty, that I was --- how can I say this? --- that I was \emph{sufficient}. Evidence to the contrary forced itself on me every hour of every day, but that seemed to me some preposterous misunderstanding.

And while I don't necessarily have fond memories of childhood--

\begin{quote}
Clearly not
\end{quote}

--some part of me does rather miss the childlike curiosity with which I was able to approach sexuality early in puberty. It was all so abstract and confusing. Every time I'd try something new, there would be this thrill of danger, this rush of excitement. The lone copy of \emph{Joy of Sex}'s assurances aside, was each burst of pleasure actually something going \emph{horribly wrong}?

\begin{quote}
Ah, to be young and anxious.
\end{quote}

And I really was. Like many kids, I suspect, my first orgasm was terrifying. I thought I'd broken myself.

\begin{quote}
You got over it.
\end{quote}

Boy did I. I soon learned to love masturbation.

But still, the bit I yearn for was the utter simplicity of my explorations. There was a lot of \emph{does this feel good} and \emph{let's try this} and so on, as I spent hours just trying to figure out what the hell bodies even are.

\begin{quote}
And the best part of it all is that it didn't involve anyone else. Your fantasies were about feeling good, or perhaps about some vague idea of sex as a concept, but it was all so abstract. The orgasm --- later, the delaying of such --- became the highest goal, the purest art. Other people just got in the way.
\end{quote}

It was a bit telling, wasn't it?
