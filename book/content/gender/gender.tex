\begin{paracol}{2}
\begin{leftcolumn}

\begin{ally}
How did we get here?
\end{ally}
What?

\begin{ally}
How did we get here? How did we get to this topic? Trace for me the route you took to get to the point where you felt able to talk about gender.
\end{ally}
Well, I suppose I started by talking about furry, which led to me talking about Younes, right? He was sort of the beginning of my more serious explorations into gender as something other than a tool for enjoying sex.

\begin{ally}
Yes, but that's not where gender is on the map, is it?
\end{ally}
Why are you trying to get me to do this?

\begin{ally}
Because we must take care to place ourselves in our time: now that we are done with writing about one of the hardest parts of our lives. And we must take special care that we locate ourselves within our place: having come at this conversation about gender through self-harm.
\end{ally}
Then yes. We got here through furry, which opened up the path before us to even begin exploring gender, and then we finally reached this topic through that of self-harm, wherein I came face to face with so many aspects of my body. It's so easy to disappear within one's own head for days, weeks, months at a time, but one eventually comes to terms with the fact that one is stuck with a body, and thus one must deal with it. Live with it and inhabit it.

What better way to experience that sudden, jarring dissonance of body-ownership than to re-inhabit it and discover it to be wrong in so many ways?
\newpage

\noindent I stand by the fact that not every trans, non-binary, or queer person experiences gender through a negative lens. Dysphoria is not a requirement for being trans. It has to be the case that there be a positive way to experience gender, or transition would be simply an exercise in futility. There has to be a flip side. There has to be gender euphoria.

\begin{ally}
There has to be the little thrill of typing \texttt{morph\ female} and being able to interact with the world around you --- even if that's only in the instance of a furry text-base role-play game --- as something other, something truer. There has to be that even when you still enjoy the body you've got.
\end{ally}
Or are at least okay with it being yours on a day-to-day basis, yes.

And I was. I thought I looked okay. I was reasonably fit. I was tall and I liked it. I was a baritone and happy with my voice.

\begin{ally}
``Was''?
\end{ally}
There has to be some flip-side, right? There has to be a flip-side to the gender euphoria that I was feeling, and that was a slowly mounting dysphoria.

If we got here through any one part of the trail I mentioned, it was through Younes specifically, more than \emph{just} furry or \emph{just} self-harm, because with Younes, so much started to hit me in a very visceral, physical way. It was one thing for me play as a girl online, to touch on aspects of gender and fertility and even sexism. It was another to be confronted with the fact that maybe the body that I had wasn't okay.

\begin{ally}
``I remember laying on the couch,'' you said. ``That awful, awful yellow couch, and {[}JD{]} getting playful, and then some little movement of his touched a nerve and I started crying because of the way that brushed up against me wasn't in focus.''
\end{ally}
Why do you bring my words back to me?

\begin{ally}
``It brought to the forefront the fact that I didn't align with myself,'' you said. ``That there was a lag in my proprioception, that I was falling behind myself.''
\end{ally}
I did. But why?

\begin{ally}
Because you wrote that in the section about liminality.
\end{ally}
Yes, but I wrote it two days later than I wrote about Younes.

\begin{ally}
The time scale is not what I'm pointing at right now.
\end{ally}
Can you point?

\begin{ally}
Are you looking at my finger, or the moon? Don't dodge this. I'm pointing at the fact that you came at gender through furry, then through self-harm, and yet this quote, this realization of ``oh, shit, I might actually be trans'', is all the way on the other side of that goofy map you make, and from there, you headed into talking about your dad.
\end{ally}
So?

\begin{ally}
And you headed from there to talking about your dad.
\end{ally}
So?

\begin{ally}
By way of talking about a dress you tried on as a kid.
\end{ally}
I think I see where you're going, but it's important that you make your point.

\begin{ally}
Gender is woven throughout this entire project. Gender is woven throughout your entire life. You build a map of this site like a web, and it is gender that is helping to hold it together.
\end{ally}
It is identity that is holding it together.

\begin{ally}
Name a part of your identity that figures larger in your life than gender.
\end{ally}
Ah.
\newpage

\noindent So, if we've talked about furry and we've talked about the dress and we've talked about dad and self-harm and the yellow couch, then what is there to talk about when it comes to gender?

\begin{ally}
Talk about what happened.
\end{ally}
Are those not things that happened?

\begin{ally}
They are things that happened before. They are precursors and doormats and signs. They all point to gender. Talk about gender. Talk about what happened.
\end{ally}
Alright.

I remember laying on the couch --- that awful, awful yellow couch --- and him getting playful, and then some little movement of his touched a nerve and I started crying because of the way that brushed up against that me that wasn't in focus. It brought it to the forefront the fact that I didn't align with myself, that there was a lag in my proprioception, that I was falling behind myself.

\begin{ally}
As you said.
\end{ally}
I remember scooting back up into a sitting position, facing JD, with us sitting by the picture window in the living room. I remember words coming out in a jumble. I remember leaning heavily on similes. I remember taking lots of breaks as though I was collecting my thoughts when really I was trying to talk without my voice going all gross with tears. That horrible, bubbly, trapped-in-my-chest sound that comes with trying to talk while crying.

I remember explaining to him that I'd been spending so much time online having different parts than I actually had, that it was super jarring to have it brought into focus that that was actually not the case. I tried to say how, feeling him aroused and pressing against me, pressing between my legs, it hurt on a very emotional level that he was pressing only against my perineum and not against a vulva.

\begin{ally}
Emotional isn't the right word there. It hurt on a visceral level. On a primitive level. It hurt in the sense that you had all of the reactions to pain except for the physical sensation of pain itself. There was the panic, the need to get away, to stop whatever was happening to cause that pain.
\end{ally}
I remember saying that I was having some complicated feelings about gender, but being largely unable to explain what they were.

They were things that I could feel and not say. They were as yet ineffable. They were liminal. They had yet to surface completely.

\begin{ally}
And they were frightening. Too frightening to say.
\end{ally}
Yes, had I the words, I would not have been able to say them out of fear. Fear that they might drive JD away, but also fear that they might be true, because if they were true, I was fucked.
\newpage

\begin{ally}
So were you?
\end{ally}
Was I what?

\begin{ally}
Fucked. Were you fucked?
\end{ally}
I think that's still to-be-determined.

\begin{ally}
You don't seem fucked. I mean, life is harder now, I suppose. You've got to contend with a minority identity you never particularly wanted.
\end{ally}
There's no denying that. I don't quite like that this is what I'm stuck with, but I do alright with it. I try to keep going as best I can, and I try to help others as much as I can along the way. Robin likes to call me a ``trans psychopomp'', but I suspect that's due in part to the word `psychopomp' is really fun to say. I would say that she falls under that title as well.

\begin{ally}
Do you see yourself as one? Do you see yourself as someone who guides others?
\end{ally}
Not particularly. I feel like I'm doing everything by accident. I feel like I'm accidentally visibly trans. Like I can't help but be visibly trans, like that's what I've got to work with. That that helps others long the way is still something of a mystery. A pleasant one, but a mystery.

Still, the least I could do is not hurt, might as well put in the effort to be a help.

\begin{ally}
Do you think that others see you as a resource?
\end{ally}
Perhaps, though that has me worried. That's an awful lot of responsibility.

\begin{ally}
Permit me to take a tangent.
\end{ally}
Do I have a choice?

\begin{ally}
You always have a choice.
\end{ally}
If I say no, what will happen?

\begin{ally}
Nothing.
\end{ally}
You'll let me just carry on with what I was saying?

\begin{ally}
Sure.
\end{ally}
Do you have the power to stop me?

\begin{ally}
No, but do you?
\end{ally}
Ah.
\newpage

\begin{ally}
Do you see yourself as a woman?
\end{ally}
I see where you're going with this.

\begin{ally}
And?
\end{ally}
It's a good direction.

\begin{ally}
So. Do you see yourself as a woman?
\end{ally}
No.~I'm a giant lump. I'm a rectangle. I'm more than six feet tall. I'm a baritone. I barely have breasts. I don't pass.

\begin{ally}
Do you want to?
\end{ally}
No.

\begin{ally}
That was easy.
\end{ally}
It's not.

\begin{ally}
No, it isn't.
\end{ally}
\newpage

\begin{ally}
Start at the beginning.
\end{ally}
And when I get to the end, stop. Yes.

As soon as I got surgery, literally when I was in the hospital, laying in bed on my five days strict bed-rest, something changed about the ways in which trans women interacted with me. I was, in some indescribable way, no longer trans.

\begin{ally}
Or, perhaps, no longer trans enough.
\end{ally}
Yes. I became a \emph{persona non grata} in a way that didn't involve actually cutting me out of trans spaces.

\begin{ally}
You were done. You were finished. You had beat the game.
\end{ally}
I was a woman now. What could I possibly bring to a trans space, now that I was just a woman? I was appropriating their spaces. I was trespassing.

\begin{ally}
So. Do you see yourself as a woman?
\end{ally}
You just asked me that.

\begin{ally}
And I didn't like your answer. Do you see yourself as a woman?
\end{ally}
I don't. I see myself as a trans woman.

\begin{ally}
Why?
\end{ally}
Do you want the scientific answer(s), or the personal?

\begin{ally}
\ldots{}
\end{ally}
Right.

I see myself as a trans woman because that's who I am. That's \emph{what} I am. I can't change that. I can't suddenly become interested in mechanical engineering. I can't suddenly be a dog. I can't even slowly become those things, I can't \emph{learn} to be a mechanical engineer, because I'm not interested in it.

I can't become a woman.

This isn't some essentialist, transphobic bullshit. Trans women are women, period. I'm not denying that.

I'm just not a woman. I'm a trans woman. I'm \emph{specifically} a trans woman. That's who I am. That's \emph{what} I am. I don't want to pass. I don't want to be stealth. I don't want to be a woman, because that's very specifically not what I am.

To have someone say, ``I just see you as a woman'' is to have a portion of my identity erased. It's reductionist to describe someone as something they aren't. That's one of the lessons we learned from folks coming out, from folks learning about identity.

\begin{ally}
You just also learned that other trans women are as apt to do the same.
\end{ally}
Yes. I left chats. I stopped talking with some people. I didn't feel welcome, no matter how friendly folks were. Where I had been leaning heavily on Maddy, that cis-female character, I started drifting back towards Makyo, towrads portraying the explicitly transfeminine.

\begin{ally}
All because they believed you were something that you weren't.
\end{ally}
Yes.

\begin{ally}
And did you ask them?
\end{ally}
No.

\begin{ally}
Why not?
\end{ally}
I didn't feel that I needed to. It was one of those types of ostracization where you're part of a circle, and then slowly people stop referring to you, and then maybe someone leans over to nudge the person standing on the other side of you and then doesn't quite lean back all the way, and then somehow you're standing just outside this circle of your very own friends, holding your red solo cup, wondering what it is that you did wrong.

\begin{ally}
Did you make your voice heard.
\end{ally}
Not for more than a year after.

\begin{ally}
Why not?
\end{ally}
Because perhaps I was appropriating their space. Perhaps I was taking this venue that was for these pre-op trans women to talk about their struggles and stepping into it unwanted. Perhaps I was stepping out of my lane.

\begin{ally}
Were you?
\end{ally}
I don't know.
\newpage

\begin{ally}
What did you do?
\end{ally}
I think the correct question is ``What didn't I do?''

\begin{ally}
I'll bite. What didn't you do?
\end{ally}
I didn't practice my voice. I didn't give up dyeing my hair. I didn't stop dressing like a mess. I didn't do all of those things that are supposed to help you get by in the world without all that added baggage of being trans.

I didn't try to pass.

I didn't try to be a woman.

I didn't want to. I want to be a trans woman. It's not masochism. It's not appropriation. I don't think so. I think it's living true to myself. I think it's being honest and saying that who I am involves being trans, and that ignoring that would be doing myself a disservice.

\begin{ally}
``I was not Madison,'' you said. ``I am not Matthew. I can't deny his existence, though. He was him, and to erase that, to toe the party line and say I've always known that I was Madison, would do a disservice to him.''
\end{ally}
Yes, but it goes beyond that. I'm not saying simply that I was not a woman and then either at some point did become one or that, at some point, \emph{will} become one. I'm saying that I live in that liminal space between. I can't be anything other than what I am. I can't live anywhere else.

\begin{ally}
There's a lot of talk in your circles about internalized transphobia. That sense that one should hate this aspect about oneself and try to get away from it. Have you not just internalized some sort of trans euphoria? Have you not simply bought into the sense of being different for being different's sake?
\end{ally}
Are you playing at being devil's advocate?

\begin{ally}
Yes.
\end{ally}
Why?

\begin{ally}
I want you to justify yourself.
\end{ally}
Why?

\begin{ally}
Because it's important that you be able to explain yourself.
\end{ally}
Why?

\begin{ally}
Because if you can't, how can you say you understand yourself?
\end{ally}
\newpage

\noindent You are playing devil's advocate because you are handily ignoring genderqueer people in order to get me to explain my identity.

\begin{ally}
I am, yes. So, explain.
\end{ally}
We, as gender-nonconforming people, talk often about gender dysphoria. There is a flip side to that. There is gender euphoria. There is that sense of rightness when you glimpse the you who was meant to be in the mirror, rather than the you who you've been trained to be.

I look in the mirror and I see a woman sometimes, and that makes me happy. I look in the mirror and I see a man sometimes, and that makes me unhappy.

\begin{ally}
Does that not make you a woman?
\end{ally}
\ldots{}And sometimes, when I look in the mirror, I see this rockin' queer person, someone who is unabashedly, unashamedly trans, and \emph{that} is when I feel euphoria.

I don't fit in cisgender spaces. I never will. I fit in trans spaces. That's the `square hole', as it were. that's where I belong.

\begin{ally}
Are you not gender-queer, then?
\end{ally}
Am I? So be it. That is not mutually exclusive with being a trans woman.

But to have that part of myself be erased by other trans women because I reached some magical stage on the gender escalator and stepped off hurts as much as being misgendered as a man by the worst TERF out there.
\newpage

\begin{ally}
I'm happy for you.
\end{ally}
What? Why?

\begin{ally}
You're proud. For the first time, you're proud of who you are.
\end{ally}
\newpage
\end{leftcolumn}
\end{paracol}
