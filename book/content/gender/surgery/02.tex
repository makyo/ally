I don't think it hit home that surgery was real until six weeks beforehand. Not that I thought it was not going to happen --- though there was some of that, of course --- but that it was something truly surreal. Some unknown and unknowable procedure would happen, and then I would be on the other side. It was almost eldritch: I would close my eyes to miss the madness and awake changed.

I say six weeks because that, specifically is when I got a call from my surgeon's office reminding me that I needed to bring my approval letters in with at the pre-op appointment so that they'd have them on file.

``But I already gave you them,'' I said. ``Don't you still have those?''

``Well, yes, but they expire after a year.''

\begin{quote}
Fuck.
\end{quote}

Yeah, fuck. Thus began a two-week scramble to find new doctors to write new letters to send in to the surgeon's office. After all, I'd moved states since I'd gotten the first letters written, and even if I hadn't, one of the doctors who had written one had retired.

I wound up getting four additional letters, as there were some questions about the validity of some of the therapists' statements and credentials.

\begin{quote}
So it felt real then?
\end{quote}

Yes, coming to terms with the fact that the surgery might have been cancelled is what made it seem as though it was something real and tangible. Real things can be cancelled. Real things can be destroyed.
