\label{movement}
\renewfontfamily\pagenumfont{Gentium Book Basic}[Color=220000FF]

\backgroundcolor{c[0]}[HTML]{e6e6fa}
\backgroundcolor{C[0](10000pt,10000pt)(0.6\columnsep,10000pt)}[HTML]{e6e6fa}
\backgroundcolor{c[1]}[HTML]{e6e6fa}
\backgroundcolor{C[1](0.6\columnsep,10000pt)(10000pt,10000pt)}[HTML]{e6e6fa}
\begin{paracol}{2}
\begin{leftcolumn}

\fontspec{Gentium Book Basic}[Color=220000FF,Ligatures=TeX]
\renewfontfamily\allyFont{Merriweather Sans}[Scale=0.9,Color=442222FF,Ligatures=TeX]

\noindent I will be the first to admit that it is difficult to write about mental health, as is certainly evidenced here already, and in countless other projects where I've tried to get that across. Even when talking about it, my voice is filled with ellipses and my words littered with hedges, fillers, and all sorts of metalinguistic dross.

\begin{ally}
That you later had to learn to use those consciously, to string like-and-if-um-but-so through your words like fairy lights to anchor your pitch is neither here nor there.
\end{ally}
And that's transition stuff. A totally different side-quest. Don't distract me.

\begin{ally}
Right. And yet here you are, distracted, talking about how difficult it is to write about mental health.
\end{ally}
Touché.

That I'll be the first to admit that doesn't excuse the way others treat it. Of course, there's countless words to be spent on the way media treats it, or the way writers treats things like psychosis, but the experience is so often so poorly researched that it hits the point of not even wrong.

Take, for example, Orson Scott Card.

\begin{ally}
There's a juicy one.
\end{ally}
Much to be said on him, yes, but take \emph{Xenocide} and \emph{Children of the Mind} as examples on this topic in particular. Take the World of Path. Take this supposed obsessive-compulsive disorder that plagues some of its inhabitants.
\newpage

\begin{ally}
Is it wrong?
\end{ally}
It's not even wrong. It's based on a lack of experience. It's based on this societal view of OCD, not the experience of it.

\begin{ally}
You sound bitter.
\end{ally}
I have a problem with compulsions. Not-even-wrong-ness surrounding them touches on a sort of meta-compulsion: a need to be understood strong enough that, when I'm misunderstood, it itches. It gets a liquid flip of my hand and touch of thumb to palm. It triggers cascading compulsions.

To then make that entertainment, to make that a hook for a plot, well.

\begin{ally}
Was it really so off-base? Did the symptoms not fit?
\end{ally}
Not all of them.

\begin{ally}
And yet the plot hook is that it was artificial in the first place. That's sort of the point, right? Fei-tzu and Qing-jao are saddled with this form of compulsive behavior that's the side effect of something else, not OCD in and of itself. Was it really so off-base, or are you just upset at seeing part of --- but not all of --- yourself?
\end{ally}
I don't know.

\begin{ally}
Are you just upset that you can't stay still; that you have other, unrelated problems with compulsion; and that these two are then correlated in a fictional genetic disorder where they are not correlated for you?
\end{ally}
Straight homeward to the symbol essence, is it?

\begin{ally}
Yes.
\end{ally}
Let's talk about movement disorders, then.
\newpage

\noindent Everyone, I suspect, deals with movement in a different way. Some are content to sit still where others have to move. Some must move, and it is a part of their personality. Some cannot move and it is a part of their physiology.

And some must move because it is an aching necessity. There is no ``if they do not move, then\ldots{}'' statement to be made. They must move. They can't \emph{not} move.

\begin{ally}
It started as a twitch, you said, as a slight nod of the head.
\end{ally}
Or perhaps it started earlier, I don't know.

\begin{ally}
Perhaps it was all caused by the meds, or perhaps it was presaged by some other restlessness that started years before.
\end{ally}
Perhaps, but does it matter?

\begin{ally}
If it was the meds' fault, you could blame them, but if it was unrelated, you would be able blame yourself. If it was the meds' fault, you could stop, if it was unrelated, you would take that as permission to feel broken.
\end{ally}
Yes, I suppose it does matter, then. That said, I have no answer for that. I just know that it started with a twitch, a slight nod of the head. My fingers would duck up away from the keyboard as though suddenly burned by the keys. I would go and sit in my car over lunch and wring my hands over and over again, occasionally trying to force myself to hold onto the shifter and the door handle, and the tremors would travel up my arms.

Eventually, at some undefinable point, it made its way up into my neck.

I never knew how to explain it.

\begin{ally}
How would you now, with seven years' experience under your belt?
\end{ally}
``Transient tic disorder''. Maybe not so transient before it disappeared, back when I thought it was going to just stick around forever.

\begin{ally}
That's what it's called, but how would you get it across?
\end{ally}
Sobbing? Frustration? Humor? I had a whole comedy set prepared for it, in case I, for some reason, needed to do a stand-up routine.

\begin{quotation}
\noindent As you can see, I have a motor tick on my neck that makes me jerk my head to the side and do stuff with my hands. This is because I have transient tic disorder, or as I like to call it, tourettes with holidays.

It makes work life interesting. I stare at a screen all day at my job. Or, well, I stare at my screen and also a point on the wall right about \emph{point} there. It's sort of a timeshare.

I could probably get jobs doing other things, though. Some contract work. Like, hey! Need someone to shake their head `no' at something? I'm your gal. Or maybe you need someone to urgently point something out out with their chin over \emph{point} there. I'd be good at that.

Now, there's a few jobs I won't be good at. Surgeon? Probably not. Bomb squad? That's a definite nope. Professional staring competition participant? I'd be right out. I couldn't win a staring competition with a three year old who's just discovered espresso.

I actually learned about all this tic nonsense at work. It started back in 2012 when it slowly started up over the course of a few days. Went on to find out that it's made worse by stress \emph{lean to the side} stand-up, of course, being the least stressful of occupations \emph{lean back} But no, I worked in health insurance. Health insurance in America as Obamacare is kicking in? Yeah, not exactly a stress-free environment.

Now, this is mostly a motor tic. I don't have the verbal tics that folks associate with tourettes. However, it does make me stutter when it gets bad. If you've never stuttered before,I can tell you that it's infuriating, so, honestly, I didn't need a verbal tic to get me cussing all the time.

So there's me sitting in meetings with other insurance companies, shaking my head `no' to everything they say, and when I try to correct myself, it comes out ``I mean ye-yes FUCK sorry''. I got really good at the whole FUCK-sorry combo.
\end{quotation}

And so on.

\begin{ally}
How effective do you think that would be on those conference calls with Lewis as you were stuttering away?
\end{ally}
I don't think I could manage. At that point, it was embarrassing enough to have picked up a stutter, a movement disorder that I never explained to my boss or the PM. To acknowledge it to the client would have been mortifying.
\newpage

\noindent When I was interviewing for Canonical, the tic had not yet started, or at least not yet to the point where it was affecting my neck or my voice. By the time I \emph{did} start at Canonical, it was well underway.

Much to my chagrin, not only was I stuttering at the time, but the job required daily video calls.

\begin{ally}
You begged off the first few, putting the blame on hardware failures. After the third day, Gary gently suggested that you consider fixing the hardware issues so that the team even knew what you looked like.
\end{ally}
It was embarrassing. Hangouts couldn't even keep up with it. The video was jittery and blurred, my face only in focus for maybe half of the time.

And then, within a few days, it cleared up and went away.

\begin{ally}
The stress of the previous job, of interviewing and those last two weeks, all suddenly relieved in one fell swoop.
\end{ally}
Yes.

\begin{ally}
And then it came back.
\end{ally}
As we all worked from home, the company had us get together in one location four or five times a year for a week at a time in order to work face-to-face and accomplish far more than we would otherwise. They called them sprints, an apt enough comparison.

Copenhagen, though, was different. It was a cascading set of stressors that culminated in, yes, the tic coming back. Two weeks long, with the first half being the developer summit, followed by a week of sprinting. The core product being rewritten. Zephyr getting attacked by another dog while I was away. The hotel, that building \href{/movement/copenhotel.jpg}{canted over to the side at a precarious 15° along two axes}, a nightmare on the acrophobia side.

The tic started up, then got worse and worse.

It was about this time that I started getting closer to Robin, and by the time we had our first real time together at FC 2013, I had shaken my sense of balance from myself and walked with a cane. ``You have a cane,'' she said, part confused, partly out of acknowledgment.

``Yeah, I lost my balance with the tic.''

``That's okay.''

And then we hugged.

\begin{ally}
Not all of it was your balance. Some of it was an apology.
\end{ally}
Yes. Someone with a movement disorder who pretends it isn't there is, in some ineffable way, sadder than someone who at least makes some public acknowledgment that, yes, this is happening. The cane helped. People would see me shaking my head, see me shaky on my feet, and then see the cane and know, ``Ah yes, \emph{this} is happening.''

\begin{ally}
You happened to pass by one of the attendees from the data panel shortly after, and overheard him telling his friend, ``That was a really cool panel, but I think he had Parkinson's or something. Every time he would get more interested in what he was talking about, it would get worse.''
\end{ally}
Yes. Part of me was embarrassed, sure, but part of me was relieved to be seen.
\newpage

\noindent Bit by bit, little by little, the tic once again slid from my life. Enough stressors had gone or were on their way out that I was gaining stillness.

I spent more and more days with fewer and fewer tics. I relished in the stillness.

\begin{ally}
Like that glass of water that's the perfect temperature. Like fresh-from-the-vine tomatoes. Like city-glow reflected on a winter cloud ceiling while you're under the covers in bed.
\end{ally}
It left for quite a while, and when it did come back, it did for only a day or two at a time. I eventually went a year without. Maybe two. I don't remember.

\begin{ally}
And then you forgot.
\end{ally}
And then I forgot.
\newpage

\noindent My journey through medication has been long and storied.

\begin{ally}
Tell me.
\end{ally}
In time.

All meds come with side effects, of course. If you take too much lithium, I found, you cycle rapidly through moods, start vomiting, and the right side of your body goes weak. When you go off fluoxetine, you get what are called brain zaps, which is rather like the feeling of missing a step on a staircase and slipping safely down to the one below it; that sense of unbalance and terror and near miss, followed by relief and surety repeated once every few seconds.

\begin{ally}
When you take anxiolytics and your life is a mess beyond simple anxiety disorders, you dissociate so hard that you try to kill yourself.
\end{ally}
I said later.

\begin{ally}
Continue.
\end{ally}
Thank you.

Well, when you take antipsychotics for long enough, you run the risk of movement disorders. That was something that had originally crossed my mind when the tic first started, except I wasn't on any of the relevant meds at the time.

\begin{ally}
And you didn't think to bring it up when you started on olanzapine.
\end{ally}
No.

\begin{ally}
Nor when you switched to quetiepine, or from there to lurasidone.
\end{ally}
No.

\begin{ally}
Why?
\end{ally}
\end{leftcolumn}
\begin{rightcolumn*}
  \emph{March 10, 2018}
\end{rightcolumn*}
\begin{leftcolumn}
\begin{quotation}
  \noindent I stayed away from reading too much about my own mental health problems for a long time because I'm not a doctor, and have seen what trying to be smarter than one's doctor can do. In fact, I stayed away from reading most anything about these things for a long time, until I realized I needed SOME language to describe what was going on to my docs.
\end{quotation}

\begin{ally}
And how did that work?
\end{ally}

\begin{quotation}
  \noindent With a recent physical health problem cropping up, I decided that my embargo wasn't worth keeping up in that instance. Of course, almost immediately after, I suffered a crash and decided to do a bunch of reading on bipolar, and you know, it's a real shitmess.

  I had thought I'd have a chance at normalcy, that I'd get better over time, that - and here I should've been tipped off to the impossibility of the scenario - I'd be able to return to some previous golden era of Madison.
\end{quotation}

\begin{ally}
And the physical health problem?
\end{ally}
A movement disorder.
\newpage

\hypertarget{icd-10-cm-diagnosis-code-f95.0}{%
\subsection{2019 ICD-10-CM Diagnosis Code F95.0}\label{icd-10-cm-diagnosis-code-f95.0}}

Transient tic disorder

\hypertarget{applicable-to}{%
\subsubsection{Applicable To}\label{applicable-to}}

\begin{itemize}
\tightlist
\item
  Provisional tic disorder
\end{itemize}

\noindent The following code(s) above F95.0 contain annotation back-references that may be applicable to F95.0:

\begin{itemize}
\tightlist
\item
  \textbf{F01-F99}\\
  Mental, Behavioral and Neurodevelopmental disorders
\item
  \textbf{F90-F98}\\
  Behavioral and emotional disorders with onset usually occurring in childhood and adolescence
\end{itemize}

\hypertarget{approximate-synonyms}{%
\subsubsection{Approximate Synonyms}\label{approximate-synonyms}}

\begin{itemize}
\tightlist
\item
  Recurrent transient tic disorder
\item
  Tic disorder, childhood, transient
\item
  Tic disorder, transient
\item
  Tic disorder, transient, recurrent
\item
  Tic, transient childhood
\item
  Transient childhood tic
\end{itemize}

\noindent ICD-10-CM F95.0 is grouped within Diagnostic Related Group(s) (MS-DRG v36.0):

\begin{itemize}
\tightlist
\item
  091 Other disorders of nervous system with mcc
\item
  092 Other disorders of nervous system with cc
\item
  093 Other disorders of nervous system without cc/mcc
\end{itemize}

\begin{center}\rule{0.5\linewidth}{\linethickness}\end{center}

\hypertarget{icd-10-cm-diagnosis-code-g25.71}{%
\subsection{2019 ICD-10-CM Diagnosis Code G25.71}\label{icd-10-cm-diagnosis-code-g25.71}}

\noindent Drug induced akathisia

\hypertarget{applicable-to-1}{%
\subsubsection{Applicable To}\label{applicable-to-1}}

\begin{itemize}
\tightlist
\item
  Drug induced acathisia
\item
  Neuroleptic induced acute akathisia
\item
  Tardive akathisia
\end{itemize}

\noindent The following code(s) above G25.71 contain annotation back-references that may be applicable to G25.71:

\begin{itemize}
\tightlist
\item
  \textbf{G00-G99}\\
  Diseases of the nervous system
\item
  \textbf{G25}\\
  Other extrapyramidal and movement disorders
\item
  \textbf{G25.7}\\
  Other and unspecified drug induced movement disorders
\end{itemize}

\hypertarget{approximate-synonyms-1}{%
\subsubsection{Approximate Synonyms}\label{approximate-synonyms-1}}

\begin{itemize}
\tightlist
\item
  Acute akathisia caused by drug
\item
  Drug induced acute akathisia
\item
  Drug-induced akathisia
\item
  Neuroleptic induced acute akathisia
\item
  Tardive akathisia
\end{itemize}

\hypertarget{clinical-information}{%
\subsubsection{Clinical Information}\label{clinical-information}}

\begin{itemize}
\tightlist
\item
  A condition associated with the use of certain medications and characterized by an internal sense of motor restlessness often described as an inability to resist the urge to move.
\end{itemize}

\noindent ICD-10-CM G25.71 is grouped within Diagnostic Related Group(s) (MS-DRG v36.0):

\begin{itemize}
\tightlist
\item
  056 Degenerative nervous system disorders with mcc
\item
  057 Degenerative nervous system disorders without mcc
\end{itemize}

\begin{center}\rule{0.5\linewidth}{\linethickness}\end{center}

\hypertarget{icd-10-cm-diagnosis-code-g24.01}{%
\subsection{2019 ICD-10-CM Diagnosis Code G24.01}\label{icd-10-cm-diagnosis-code-g24.01}}

\noindent Drug induced subacute dyskinesia

\hypertarget{applicable-to-2}{%
\subsubsection{Applicable To}\label{applicable-to-2}}

\begin{itemize}
\tightlist
\item
  Drug induced blepharospasm
\item
  Drug induced orofacial dyskinesia
\item
  Neuroleptic induced tardive dyskinesia
\item
  Tardive dyskinesia
\end{itemize}

\noindent The following code(s) above G24.01 contain annotation back-references that may be applicable to G24.01:

\begin{itemize}
\tightlist
\item
  \textbf{G00-G99}\\
  Diseases of the nervous system
\item
  \textbf{G24}\\
  Dystonia
\item
  \textbf{G24.0}\\
  Drug induced dystonia
\end{itemize}

\hypertarget{approximate-synonyms-2}{%
\subsubsection{Approximate Synonyms}\label{approximate-synonyms-2}}

\begin{itemize}
\tightlist
\item
  Dyskinesia, subacute, drug induced
\item
  Neuroleptic induced tardive dyskinesia
\item
  Subacute dyskinesia due to drug
\item
  Tardive dyskinesia
\end{itemize}

\hypertarget{clinical-information-1}{%
\subsubsection{Clinical Information}\label{clinical-information-1}}

\begin{itemize}
\tightlist
\item
  Iatrogenic extrapyramidal disorder produced by long-term administration of antipsychotic drugs; characterized by oral/lingual/buccal dyskinesias and choreoathetoid movements of the extremities.
\end{itemize}

\noindent ICD-10-CM G24.01 is grouped within Diagnostic Related Group(s) (MS-DRG v36.0):

\begin{itemize}
\tightlist
\item
  091 Other disorders of nervous system with mcc
\item
  092 Other disorders of nervous system with cc
\item
  093 Other disorders of nervous system without cc/mcc
\end{itemize}
\newpage

\noindent There is a certain unique agony to akathisia. When I was in the hospital after surgery, and even for weeks afterwards, I was dead convinced that the problem I was going through was related to temperature. Part of this, no doubt, was due to the weather warming up followed by, toward the end of my inpatient stay there, the climate control in the room going out, leaving it a sweltering (to me) seventy-six degrees.

\begin{ally}
What you didn't take into account was the fact that you have a hard time sitting down for an hour at a time, never mind being confined to bed rest laying on your back only for five days straight.
\end{ally}
Even so, for weeks afterwards, I was desperate to do anything I could to stay cool. I picked up an ice cream habit that I'm still fall into regularly. I installed a window A/C unit. At one point, I even contemplated sleeping in the garage where it was cooler at night due to the lack of insulation.

\begin{ally}
Judith visited toward the end of this period. You did everything you could to keep the rooms you stayed in on the road trip to the bay as cool as possible. The bay, where A/C just isn't a thing.
\end{ally}
Yes. And shortly after that, I learned about akathisia.

I say `shortly after', when it was likely during that trip when I realized I felt the most relief from the symptoms by moving. The constriction imposed upon me by recovery had lessened over time until I was able to go for that hike with Judith, Robin, and Josh, and suddenly I realized that I felt better than I had in a while.

I just learned the word for it shortly after, the name. And by naming a thing, hoped to gain some sort of power over it.

\begin{quotation}
  \noindent Alv pinned his ears back against his head as he stomped down the length of the block. His boots were too much for the drizzle that the weather offered, but it was that or his threadbare sneakers, and some tiny part of his mind had done the calculation without the rest of him knowing, and he'd tugged the heavy things on for the walk.
\end{quotation}

\begin{ally}
Because of course you have a furry story about akathisia.
\end{ally}
Write what you know.

\begin{quotation}
  \noindent The air inside had grown too stuffy for the old fisher, or perhaps his eyes had grown too tired of reading, or maybe it was something in his joints, a feeling of too much space that needed to be compressed down. The solution, no matter the problem, was to move.

  His third time around the block, knees and hips aching from walking in work boots that were never meant for the task, and Alv still hadn't figured out what it was that kept driving him out of the house. He'd walk, day after day, until his tail drooped and his feet started dragging. Sometimes, like today, he'd circle the block. Some days he'd drive the mile to the supermarket and walk aimlessly up and down each aisle, eventually picking up a drink or a snack, just to make the trip worth it. Other days, he'd just pace in his building's parking lot.

  He didn't think.

  Or maybe he thought too much. Maybe that was it. Maybe the fisher's every step was taken to crush too many thoughts beneath the soles of his boots, pressing the life out of them through the sheer weight of his restlessness.
\end{quotation}

\begin{ally}
And you would, too. You'd walk and walk and walk, hoping that perhaps you could walk the thoughts out of you.
\end{ally}
Yes.

\begin{quotation}
  \noindent He didn't know what it was that, day by day, drove him to his feet, drove him to walk until he couldn't walk anymore. He just knew that if he didn't, that ache within him, that burning, that itch would continue to grow, and he'd start to feel like his heart was being extruded through his rib cage, like his fur was coming out in clumps, like he couldn't possibly breathe deep enough.

  His wife, gone now these five years, had been fond of calling him a restless soul. He wasn't sure that he was capable of believing in a soul, nor that this increasingly restless state of being was confined to something so intangible. He was just restless.

  \emph{Just. Only.}

  That's all he was. There was nothing to him except restlessness. After Naomi's death, he'd slowly become less and less of a person, until all that was left was the urge to move, the terror over being confined to one place for any length of time.

  His tail starting to sag, the fisher could feel all the calm he'd accumulated through the walk start to ebb, the tide of anxiety creeping in from the edges, from his fur inwards. One last trip around the block, he figured, was all he could manage before resting again.
\end{quotation}

\begin{ally}
Write what you know.
\end{ally}
Yes. Furry is a framework. Apply an experience to that framework and see what you get.

\begin{ally}
Sure, but we've already been over that.
\end{ally}
Yes.

\begin{quotation}
  \noindent By the time he made it around to his building again, Alv was well and truly sore, knees and hips aching from the repetitive motion of stomping around the block. Still, he couldn't bring himself to head up to his apartment quite yet. The idea of being closed in such a space held negative appeal. Something about the thought of four walls was actively repulsive.

  So he sat on the damp stoop and watched the trees across the street.

  The drizzle had dried up---though he hadn't noticed when---and all that was left was the occasional \emph{pat} of drop on leaf as some bit of water got too heavy and sought a new home closer to the ground. There was just that gentle sound. Despite the hour, the street was empty of traffic, as though the shoddy weather had chased everyone inside.

  ``Would that my soul were that calm,'' he mumbled to the bare street at last and levered himself up creakily, climbing the rest of the stairs to head inside.
\end{quotation}

\begin{ally}
Write what you know. Write about the way pacing slowly moved from its status as nervous habit to a necessity, to an ache. Write about how there was no relief in walking, just a drive, an itch you could never scratch but were nonetheless required to try. Write, and cast those words upon something else, upon someone else, so that you can look on them and say, ``Ah yes, \textbf{this} is happening.''
\end{ally}
\begin{ally}
Write what you know but don't yet understand.
\end{ally}
Maybe I can get closer that way.

\begin{ally}
Yes.
\end{ally}
\newpage

\noindent Only five months after I figured out just what akathisia was, the disorder evolved into something more dramatic. Whereas the tic, whether or not it was iatrogenic, affected mostly my neck and only rarely my wrists, this new form of drug-induced movement disorder affected most of my upper body, dystonia alternating between athetosis and chorea; between a fluid, graceful swimming of limbs to a tense, rigid posture with repetitive jerking movements.

It was infuriating and humiliating --- and before you interrupt, no, I will not talk about kink.

\begin{ally}
You know me so well.
\end{ally}
I suppose I do.

To be unable to hold still is one thing. Jerking my head to the side once every few seconds with the tic was embarrassing enough. I often worried that I'd be mistaken for some sort of junkie, hopped up on something or another. I even had my doctor write a letter explaining what was happening that I could bring with me when I traveled.

\begin{ally}
But you were still functional.
\end{ally}
Yes. I could still work. I could still drive and walk and pick things up and eat.

\begin{ally}
Now you couldn't. Now your hand would jerk back from picking things up or hitting the keys. Now you would walk with a hitch in your stride as a spasm rolled along your side. Now you wouldn't feel safe behind the wheel.
\end{ally}
I mostly just shut myself in my house. I left twice. Once to see a friend for some company, and once to go to therapy. I stood in the lobby while my therapist had a small chat with a coworker, struggling to keep still with my hands buried in my pockets, and broke down crying once we made it to the room.

She had a solution --- or a set of solutions --- that we could try. One medication, benzatropine, to start with, one fallback medication, tetrabenazine, and a intensive vitamin regimen to start on right away. Picking them up at the pharmacy on the way home was another source of tears, as the pharmacist, reading off the screen, said, ``This is for twitching? Involuntary movements?'' and I nodded, more a jitter than an intentional motion, as my hands wandered off along strange hyperbolae, unable to speak for the tears.

\begin{ally}
And then, Thanksgiving.
\end{ally}
Yes. Thanksgiving, and my dad visiting.

\begin{ally}
He had seen the tic before, at least.
\end{ally}
Well, yes, but as mentioned, these movements carried along a whole new set of connotations with them. Suddenly I was unable to have a basic conversation without the pauses that come with those moments of fixed posture. Suddenly I was unable to get a bite to eat without engaging in my geste antagoniste, resting my chin on the back of my hand with my wrist twisted around unnaturally.

\begin{ally}
Suddenly you were painfully, visibly vulnerable in front of him.
\end{ally}
Yes.

And at a restaurant. A dinner that cost him eight hundred dollars for the four of us.

\begin{ally}
At one point, he asked you what was wrong and you tried not to cry as you mumbled, ``I'm just having a hard time holding still.''
\end{ally}
No one mentioned it, after that.
\newpage

\begin{ally}
And now you're still again.
\end{ally}
Sometimes. One of the treatments worked, though I'm not sure which. One of them caused vertigo and nausea, though I'm not sure which. But even after I went off them, I'm usually still.

\begin{ally}
Is that not enough?
\end{ally}
It's certainly better, don't get me wrong. The stress of driving will bring out the dance-like turn of my arm. An interview a few weeks ago went poorly after the twitching and twirling got bad enough to prevent me from focusing on the problem at hand. A distressing scene in a movie will leave me paralyzed and rigid in my seat, posture unnatural and unnerving.

Judith reassured me that it looked like I was stretching, that it was less distressing than the tic.

\begin{ally}
You still apologized. You apologized to all of your partners the first time they saw it, and countless times after.
\end{ally}
Yes. I explained and explained, hoping they'd forgive me.

\begin{ally}
For what? For being less than perfect?
\end{ally}
For being vulnerable. Even after so long away from my dad and Jay, it's ingrained in me that vulnerability is a personal failing. Or perhaps it's more general: perhaps vulnerability is worth apologizing for because of some hereditary reason. Perhaps I'm apologizing to my ancestors, to the human race, for being less than they hoped for, for being a disappointment.

\begin{ally}
How very human of you.
\end{ally}
My therapist apologized to me on one stressy day when I was visibly struggling to stay still. She said she felt bad for having caused this. I rushed to reassure her that, no, it probably wasn't her fault, that I'd been on the antipsychotics for a while before ever meeting her. That the tic started back in 2012 before I'd even started those.

\begin{ally}
You apologized for the fact that she felt the need to apologize.
\end{ally}
Well, yes.

\begin{ally}
It's not your fault either, you know.
\end{ally}
On an intellectual level, sure. I know. On some deeper level, obviously I don't. Or can't.
\newpage
\end{leftcolumn}
\end{paracol}
\resetbackgroundcolor

\renewfontfamily\pagenumfont{Gentium Book Basic}[Color=000000FF]
