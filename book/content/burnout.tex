How did I get here?

\begin{ally}
How did you get where?
\end{ally}
How did I get here? How did I get to the point where I loathe my job? How did I get to the point where I loathe my life, but mostly only when I'm working?

\begin{ally}
Start from the beginning.
\end{ally}
Which beginning?

\begin{ally}
Madison's beginning. For this, I don't think you need to go any further back for any reason other than to confirm what you already know. Or perhaps just a bit before. Start with the insurance company.
\end{ally}
What, working with Kevin?

\begin{ally}
Yes. Start from there.
\end{ally}
In 2011, I graduated --- or, well, left --- university and jumped straight into a job doing software for a subsidiary of a subsidiary of a company that made software for health insurance companies. I had a whole weekend off.

It was thrilling, in a away, to be seen as competent at something. It was nice to be able to drive to an office, sit down at a computer, type away for a few hours, drive home, and then see money in my bank account after the fact, knowing that I had done something that was useful.

\begin{ally}
Were you not doing anything useful before? You were working, you were at school. You were getting paid.
\end{ally}
I was. But even when I looked at that money in my bank account, I couldn't then count it out and say, "Ah, yes, this was earned creating something." Work was spent living on the edge of failure, trying to push it back just one step further. That's the curse of IT.

\begin{ally}
And school? You were creating something there.
\end{ally}
And paying a pretty penny for the privilege to do so.

\begin{ally}
Right.
\end{ally}
But this was something new, I was given a list of things that they wanted to be able to do and given basically total freedom to pull that off. I was put in front of their raw materials and, when I showed them progressively more and more refined creations, they all stood back and applauded, and I could bow and say that I had created something for them.

\begin{ally}
And then?
\end{ally}
And then\ldots{}well, I don't know. And then the tasks got smaller and smaller, and the clients grumpier and grumpier about more and more inconsequential things. They needed twice as many new features done in half the time and could we work the weekends? After all, they had their QA people sleeping in the office in cots in the bathrooms. Shouldn't we do the same?

At some point that must have changed, but it all changed so gradually as to not be noticeable.

\begin{ally}
And then you started to see how capitalism worked, perhaps? That you weren't doing this because it was fun or because you were good at it, even if it was and you might have been, but because you had to.
\end{ally}
I think that may be getting a bit ahead of the game, but in a way, I suppose so. I started to see that it was very easy to use up all of one's spell slots. I started to see just what purpose free time had in one's life. I started to talk about work-life balance and to schedule vacation time that wasn't simply holidays and to dream about the office.

\newpage

\begin{ally}
At what point would you say you burned out?
\end{ally}
That's one of those surprisingly difficult questions. I can't point to a day or week when things went bad, nor even a month. At some point, I just looked around me, at my office and my coworkers and my job and said, "I hate all of this."

\begin{ally}
When did you notice it, then?
\end{ally}
Does when I tried to kill myself" count?

\begin{ally}
Not my department.
\end{ally}
I spent a lot of time trying to fix it. I spent a lot of time changing little bits about my day or my desk or my tasks, and there was just not much that could put a dent into that mixture of loathing and anxiety that surrounded my day.

\begin{ally}
And eventually, you just dumped the whole thing in favor of something else.
\end{ally}
Yes.

\begin{ally}
Did it work?
\end{ally}
Oh, definitely. I jumped at the opportunity to stop working for an insurance company that just happened to need some software and to start working for a software company with a name that folks knew making products that I believed in.

Moving to Canonical came on such a whim, too. I met up with John Wright --- such a nice man --- at Mayor of Old Town and we talked over pints about the good and the bad of our respective jobs.

"I've been thinking about applying at Canonical," he said, twisting his glass between his hands. "I'm not unhappy at where I am, I'm just\ldots{}not happy either."

I nodded, and made silent note to check out their postings later that night.

\begin{ally}
Did you wind up stealing John's idea?
\end{ally}
Oh, totally. I apologized to him after the fact, too, for taking his idea and actually winding up with the job. He laughed and said that he didn't think he'd be able to work from home anyway.

\begin{ally}
Whereas that saved you.
\end{ally}
Yes.

In a way.

For a while.

\newpage

I could very easily get into talking about the ins and outs of working at Canonical and in software, but I don't think that's the point.

\begin{ally}
No, it's not.
\end{ally}
No. The point is that, slowly, quietly, without me even noticing, I started hating what I do for a living. It snuck up on me once more. I once more found myself in a paralyzing mixture of anxiety and dread and anger. Every minute spent in front of my editor was spent filled with anger and frustration at not being able to work, and every minute spent away from it was spent dreading the next time I'd have to go back, fretting over how little I had gotten done.

I spent day after day on branches that should have been small and yet somehow, inexplicably, seemed insurmountable. Coworkers and bosses got upset at me. I did all I could to keep interested and invested in the company.

\begin{ally}
Even as you drifted your separate ways? Canonical stopped doing things that were relevant to you before you even moved to Seattle. They started focusing on things you didn't believe in. They laid off dozens of your coworkers. They started courting Microsoft.
\end{ally}
Sure, I suppose. There's no doubt that Canonical was changing. They were certainly not blameless in me losing my interest and investment in them.

\begin{ally}
And from what JC says, you would hate them now.
\end{ally}
I would, yes.

\begin{ally}
And yet here you speak only of yourself. Only of your failures.
\end{ally}
Is this not a selfish project? I think that it's fair to just talk about how I feel when I talk about burnout.

\begin{ally}
Burnout does not happen in a vacuum.
\end{ally}
I hardly believe that the things that Canonical was doing were so new as to be causing my burnout. They were doing as tech companies do. They were doing everything they could to maintain the same amount of velocity they had at the beginnings of projects later on. They were trying to change with the times while remaining exactly the same.

Perhaps it was just the honeymoon period finally coming to an end.

\newpage

\begin{ally}
The third time was not the charm.
\end{ally}
No, it was not. Canonical stopped doing something I believed in, so I switched to a company --- Internet Archive --- that \emph{was} doing something that I believed in, but the process was crap. Now, here I am at a company that's got a great process and is doing something that I really believe in it, and\ldots{}

\begin{ally}
And you hate it.
\end{ally}
I hate my career. I don't hate my company. I love them. They're great people doing great things and doing them well. I just can't stand programming anymore.

\begin{ally}
I don't believe you.
\end{ally}
You don't?

\begin{ally}
I don't. You, who have at least two open programming projects you poke at with some regularity.
\end{ally}
I suppose I do, yeah.

\begin{ally}
So what do you hate, if you don't hate programming?
\end{ally}
It's not work. I don't hate working.

It's not programming, you're right there. I still love the idea of making something that does what I tell it.

It's not computers, even if I'm a bit ambivalent on them.

It's\ldots{}well, I definitely hate devops.

\begin{ally}
Why?
\end{ally}
It feels\ldots{}messy. It feels like I'm doing all I can to drag these ephemeral things into line, and none of them want to do it. It feels like all these people have grandiose ideas about what goes into running a system, and none of them agree with each other, and all we can do is to pick the least-bad one.

It destroys this idea that computers are a thing that you can ask to do something, and they can do it. There are more non-deterministic bugs in devops than in any other area of dealing with computers than I've experienced.

It makes me want to take up Haskell.

\begin{ally}
All very sensible.
\end{ally}
If such a thing can be said of it.

\begin{ally}
Is that why you're burnt out, then?
\end{ally}
No.

\begin{ally}
Then why?
\end{ally}
I don't know.

Perhaps I'm only good for seven years at a time, like I said.

\begin{ally}
Did you burn out on music?
\end{ally}
I would say that I was burnt, but I placed that on the performers at my recital.

\begin{ally}
Had your recital gone perfectly, would you still have felt burned out, though?
\end{ally}
Perhaps.

\begin{ally}
Would you still have gone into computers?
\end{ally}
Definitely.

\begin{ally}
Would you still be composing?
\end{ally}
I don't know.
\newpage
