My senior recital did not go well.

\begin{quote}
Understatement.
\end{quote}

It was a failure from very early on. I was commissioned to write a work for two friends in the music department. French horn and contrabass are an unlikely combination, so I figured it'd be a good challenge. It turned into a nightmare with astonishing speed.

They dictated what I wrote to a large extent, and when Dr.~David heard about it, he explained that that's not quite how it was supposed to work. I flailed and finished the piece as best as I could.

I couldn't find performers to commit to any of my pieces. When I did, they didn't practice. The two who commissioned that work from me only practiced once: half an hour before the concert itself.

The performance itself was a disaster.

\begin{quote}
You grabbed the recording and left to dinner with your mom and dad, Bob, Maurine, JD, his dad, and Diane. Diane said, as politely as she could, that many of your pieces sounded ``so dark'', and it was all you could do not to cry and say that it wasn't supposed to be that way.
\end{quote}

I gave up after that. I stopped going to class regularly. I stopped doing homework. I started programming more. I worked as many hours as I was allowed. I applied for tech jobs.

\begin{quote}
You kept singing.
\end{quote}

I did, but my heart wasn't in it.

I left music.

I stopped composing.

It took a year, but I stopped performing.

I couldn't do it.

All of the work I had put into it, all of the time and effort and blood and sweat and tears, and as soon as I had something I was proud of, I was shown just how little the world thought of me. My community didn't change, and yet it felt hateful to me. I had no guarantees at all that it would get any better, so I got out while I was at least only a little behind.

\begin{quote}
In writing, you were later told, the worst that could happen if you submitted a story was that the editors would say no. This was worse than the editor saying no. This was the editor sneering at you, looking you directly in the eye, and slowly tearing your story to shreds, long strips of paper dropping from their hands as you watched.
\end{quote}

And I had to smile as I did so. I had to smile and shake hands and gesture for the performers to bow. I had to keep talking to the audience, explaining the significance and features of each piece throughout the recital even as it continued to get worse and worse.

\begin{quote}
You stopped writing music.
\end{quote}

Why wouldn't I? Life told me what it thought of me doing so. Why would I willingly continue to fail?

\begin{quote}
You were not strong enough.
\end{quote}

I was not strong enough.

\begin{quote}
You started programming.
\end{quote}

Website after website.

\begin{quote}
You started writing.
\end{quote}

I splashed around in great heaps of words.

\begin{quote}
You promised yourself you were okay with the outcome.
\end{quote}

Seven years was enough.

\begin{quote}
And now it's seven years since you got into tech.
\end{quote}

Yes.

\begin{quote}
And you started writing music again.
\end{quote}

Yes.

\begin{quote}
A few pieces. Miniatures. Stuff you can finish without getting tired of it first.
\end{quote}

Yes.

\begin{quote}
Something to try and capture the agony and the ecstasy.
\end{quote}

Yes.

\begin{quote}
You still write for choir.
\end{quote}

Yes.

\begin{quote}
Stuff that will never get performed.
\end{quote}

Yes.

\begin{quote}
You promise yourself you are okay with this.
\end{quote}

Yes.
