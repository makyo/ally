When you're a choir kid, you're a choir kid.

\begin{quote}
The first rule of the tautology club is the first rule of the tautology club.
\end{quote}

You have to understand. There's a level of identity, a level of expression that goes along with being a choir kid. It's writ on your face. It's in the way you walk. It's an aura that emanates from you. It hovers about your head in a halo. It colors your perception, and others' perception of you.

You don't do choir. You \emph{are} choir.

\begin{quote}
Just as you \textbf{are} furry?
\end{quote}

There's plenty of comparison that can be made there, yes.

\begin{quote}
Like how, fresh out of middle school, fresh out of your mom's messy divorce with Jay, fresh after your mom's diagnosis, so soon after running away, you found yourself once again largely alone. It was more complex now, too. You weren't simply physically alone. You were a newborn and you were alone in the world. You were alone on some ineffable level. You craved a family. You craved a community. You needed to not be alone. You needed those things to grow up, whether you knew it or not --- and you didn't --- so you latched onto whatever you brushed up against, arms hard around it, and you refused to let go. You refused to let it let \textbf{you} go.
\end{quote}

I\ldots{}well. Huh.

\begin{quote}
Carry on.
\end{quote}

Give me a second.

\begin{quote}
Take your time.
\end{quote}

I suppose I was going to go on to say that when you're a choir kid and a boy, something happens inside people's heads. They go a little bit crazy.

There are other identities within school, after all. There's band, of course. Band is pretty egalitarian (in some ways; obviously individual instruments have their own gender roles). There's some of the sports, too, where a girl joining the team would be quite out of place, if it's even allowed. Nerds fall along similar lines --- or fell, I suspect this is changing --- in that a girl nerd is considered something more unique.

High-schoolers, however, seem to be intensely aware of gender roles, even if they don't realize. This includes the power dynamic instilled in them in the west. A girl ``striving'' to be ``something greater'' by taking part in a supposedly masculine activity--

\begin{quote}
Nice qualification quotes.
\end{quote}

--is a curiosity, perhaps gently encouraged, perhaps the source of patronizing.

A boy ``falling back'' to ``something less'' by taking part in a supposedly feminine activity is a cause for alarm, a cause for concern, a cause for laughing and jeering and taunting.

\begin{quote}
That you transitioned later in life being, of course, irrelevant.
\end{quote}

It sort of is, it sort of isn't.

It is, because I don't think I know any other choir friends who transitioned. And not just those like me who transitioned and then dropped out of choir because boy is \emph{that} fraught.

It isn't, because in a lot of people's eyes, that's confirmation that joining choir was an early sign of my weakening masculinity. It's self-reinforcing that way.

\begin{quote}
As are a lot of social roles. Furries are nerdy because they're expected to be, and so they attract nerds. Nerds are awkward because of course they are, and so awkward kids are more likely to become nerds.
\end{quote}

When you're a choir kid and a boy and \emph{gay}, after all, well\ldots{}pff, of course. A boy in choir \emph{would} be gay.
