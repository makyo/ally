This chapter of ally takes place in the git commit messages, but here are their contents for completion's sake.

\begin{ally}
I'm ashamed to know you.
\end{ally}
It's a stretch even for me, but hey, here we go.

\begin{center}\rule{0.5\linewidth}{\linethickness}\end{center}

\begin{ally}
Are you having fun with this?
\end{ally}
Did you really expect me to not approach the idea of writing about software in any other way? Did you expect me to not be something of a nerd about this?

\begin{ally}
I suppose not. Tell me about software, then.
\end{ally}
What's to say? Mom decided that, since I was showing an interest in computers, it might be a good thing to let me use her copy of VisualBasic 4. From there, I just kept on going.

\begin{ally}
Well, hold on, you're skipping over a whole bunch of stuff.
\end{ally}
I suppose so.

\begin{ally}
You're skipping over your dad joking that, since you spent so much time on the computer, that he was always worried that the FBI would come knocking on the door one day.
\end{ally}
Well, he was the one who got me the computers in the first place. He bought me a copy of RedHat 6.2 on a CD at Circuit City.

\begin{ally}
Oh, my aching bones.
\end{ally}
I know. Every single bit of that sentence was ancient.

Still, it's largely his fault. We strung coax throughout the house in a simple network. He bought a file server, a copy of Windows NT, and we worked on setting up IIS together so that we could have both a file share as well as a way of getting those files from work for him, and my mom's house for me.

\begin{ally}
Very kind of him. Forward thinking.
\end{ally}
He wanted me to be an engineer. What better way to get me into the mindset? Besides, \emph{stuff} was his game. Our relationship was not yet mature enoug that we could be buddies, so instead, he did what he thought parents were supposed to do and punished, instructed, and showered with gifts. It's just that some of those were computers.

As many gifts bounced off of me as those that stuck and proved useful.

Either way, start a kid on VisualBasic and give her access to AngelFire, and you're bound to wound up with at least \emph{some} kind of nerd.
\newpage

Matthew was pretty keen on Perl at the start. Something about all the delicious punctuation, all the built-in obfuscation was appealing. Something about how you could write an incantastion that was difficult to read unless you had the proper knowledge tickled him.

\begin{ally}
He wasn't very good at it.
\end{ally}
Well, no. He was pretty terrible at it. He uploaded some samples to Perl Monks and mostly got yelled at. From then on, he developed alone, with little to no communication about what he was doing with anyone who might be able to help.

\begin{ally}
A solipsistic software engineer? Color me surprised.
\end{ally}
Right.

Perl filled high school. Dumb scripts to walk a directory (despite a module already existing in CPAN). A guestbook. A forum. A terrible website.

\begin{ally}
Was it that bad?
\end{ally}
\href{https://web.archive.org/web/20050202100148/http://ranna.babylonia.flatirons.org/}{RF!P}? Oh yes.

\begin{ally}
At least you can see the dull adherence to monochromatic web design started early on.
\end{ally}
Listen. Color is hard.

Either way. There was a brief PHP phase toward the end of high school, and then it was off to university and John Wright teaching him about Python and Django, and he was lost.

It made it so easy to start projects.

\begin{ally}
Too easy.
\end{ally}
Yes. They littered his computer, his \href{https://github.com/makyo-old/}{git repositories}. Started and abandoned, sometimes even before any code was written. There exist more than one project which is simply a skeleton of a Django application with a name. No code. No documents. No info.

\begin{ally}
No motivation.
\end{ally}
Or maybe only the false motivation that comes along with hypomania.
\newpage

At some point in late 2005, I got my first job in computers--

\begin{ally}
Well, hold on. What about that summer job at Rational?
\end{ally}
That was before birth, remember. That happened to someone else. That happened somewhere else.

\begin{ally}
You have nothing to say about your mom getting you a job testing software with one of her friends? You have nothing to say about learning the boredom of menial tasks? You have nothing to say about the time you found a rendering bug in Java, some part of the windowing system, but you couldn't file it because the bug was that characters from the PuTTY screen showing your MUCK connection showed through, scattershot? You have nothing to say about bagel mornings, about the breakfast burritos you still think about, about stopping at the hot dog cart on the way home and getting to know Mikey, who sold them, about the countless jokes you shared about how awful ketchup was on a hot dog?
\end{ally}
Clearly you do.

\begin{ally}
You thought it was great at first. No restaurant work for your first job, but something in computers. Something you could be proud of. That pride your dad taught you. Then you learned about what goes into a QA tester's job. Then you learned about how boring computers could actually be. Then you learned how to resent them for how much of a mistake they were in the first place.
\end{ally}
Bit harsh, but true enough.

\begin{ally}
``Computers were a mistake'', right? That's how you put it?
\end{ally}
Yes.

\begin{ally}
So you got your first job in computers shortly after you were born --- don't try to tell me it wasn't. It was the summer after your Freshman year. Your metaphor won't always hold up.
\end{ally}
\ldots{}Ah. Right.

\begin{ally}
And then you never got a summer job again until university. You kept looking, but there was little for you to do that would hold your interest if computers were so spoiled for you. You applied at coffee shops. You applied at Blockbuster. You applied at the YMCA.
\end{ally}
And every summer, I disappointed my mom further.
\newpage

Well, then I suppose my second job in computers was in late 2005, when I got that job at the library. That was far more comfortable.

\begin{ally}
Or you were far more mature, perhaps.
\end{ally}
Maybe. Either way, it was something that I was able to actually focus on, do a good job on. There was downtime, and sometimes it got crazy. Sometimes we'd come into the library long before it opened and blast music while we installed or reimaged whole swaths of computers.

Sometimes we'd dick around. Nerf footballs, library cart racing. One time Josiah locked the surplus filing cabinet we had but did not have the key for and we had to drill out the lock. When we got it unlocked, the first thing he did was to lock it again. We hollered and chased him from the room as we struggled desperately to unlock the cabinet once again.

\begin{ally}
It was fun.
\end{ally}
For the most part, yes. I did some development for them, too. It was my first software job as well as my first job in computers. I did the Atmospheric Sciences Reading Room site. I did some campus mapping. I was enjoying it.

Enjoying it enough that, when my future in music burned down around my ears, I was ready enough to jump on any job offer in tech that I could manage to pull off.

\begin{ally}
Whether or not it was something you might actually enjoy.
\end{ally}
Yes.
\newpage

At least I enjoyed it at first.

\begin{ally}
You did, yes. You worked ten, twelve hours a day.
\end{ally}
I was doing something. I was actually producing something, and it was being recognized by people. Music was fine, sure, but no one really paid it much attention.

\begin{ally}
Is anybody paying attention to your writing?
\end{ally}
You are.

\begin{ally}
If you say so.
\end{ally}
A few others, maybe.

\begin{ally}
If you say so.
\end{ally}
Don't be cruel.

\begin{ally}
If you say so.
\end{ally}
\newpage

I enjoyed it until I didn't. It turned into a grind, it turned depressing. I started getting angry. I tried to commit suicide --- we'll get to that later, just to preempt you distracting me.

\begin{ally}
You know me too well.
\end{ally}
Do I?

\begin{ally}
Don't lose focus. You left UHG for Canonical, and started all over again.
\end{ally}
I lasted longer this time, in terms of burnout. I was productive for a lot longer. I liked the job a lot better. Even after I left, I think I liked it better at its worst than I liked IA at its worst.

\begin{ally}
And at least you did rather like some of the coworkers.
\end{ally}
But we can talk about that later. Distraction, remember?

\begin{ally}
Sure, sure.
\end{ally}
But it's been seven years, and it appears that's all I'm good for. I was good for music for seven years. It's been seven years, and I'm not sure I'm good for programming. Will writing fade from me, too? Seven years down the line?

When will you fade?

\begin{ally}
When will you fade?
\end{ally}
\newpage
