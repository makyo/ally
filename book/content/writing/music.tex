\end{leftcolumn}
\begin{rightcolumn*}
\noindent\includegraphics[width=2in]{assets/static/miniatures/1-1.png}

\noindent\includegraphics[width=2in]{assets/static/miniatures/1-2.png}
\end{rightcolumn*}
\begin{leftcolumn}
  \noindent I did not fall into music of my own accord, my dad bought me a saxophone.

\begin{ally}
As his dad bought him before you.
\end{ally}
He wanted us to be alike in so many ways.

\begin{ally}
But you already knew that.
\end{ally}
He got me a saxophone and he and my mom pooled resourses to get me lessons.

\begin{ally}
And showed you to all his friends.
\end{ally}
I played at his Christmas parties. I played at his neighbor's Christmas parties.

\begin{ally}
Once, he was going to show you off to his friends at a barbeque, and you got so anxious and upset that you bent the octave key out of shape. You could only produce squeaks. You said it was an accident.
\end{ally}
I did it to get out of playing for the party, and instead it got me in trouble for being careless.

\begin{ally}
You were anything but. You were very careful. You acted with intent.
\end{ally}
I kept playing. Sometimes it was fun, sometimes it wasn't.

\begin{ally}
Once, you told your mom you weren't sure why she or your dad bothered with you learning to play the saxophone when all life was meaningless, anyway.
\end{ally}
How old was I, then? Ten? Eleven?

\begin{ally}
Dad made you apologize to her. I don't think either knew what to do with a nihilistic preteen.
\end{ally}
But it worked, in a roundabout way. I wound up in music. I wound up playing the saxophone and even sometimes enjoying it. I moved from that to the oboe.

And not just playing. I listened to tapes until they wore out. I made mixtapes of my dad's music after he taught me how to program his six-disc CD changer. After that, it was mix CDs, which I'd listen to on the bright yellow Sport Discman I carried everywhere. I fell asleep with headphones on more than once.

Music held --- continues to hold --- this sense of mystery about it. It worked on some level below spoken language, understandable without being text, affecting emotions without the cadence of words.

\begin{ally}
So why'd you quit?
\end{ally}
I can't just say ``computers'' and beg off, here, can I?

\begin{ally}
Nope.
\end{ally}
\newpage
\end{leftcolumn}
\begin{rightcolumn*}
\noindent\includegraphics[width=2in]{assets/static/miniatures/2-1.png}

\noindent\includegraphics[width=2in]{assets/static/miniatures/2-2.png}
\end{rightcolumn*}
\begin{leftcolumn}

Okay, you're right. It's not quite true that I left because of computers. I stopped playing the oboe after I ran away and moved schools. Band was already well underway, after all, and I couldn't join in partway through. They let me play the cymbal in one concert, but I basically gave up after that. We returned the rental oboe. I wouldn't touch an instrument in all seriousness until well into university.

And really, during all that time, there was no sense of regret, no sense of loss.

\begin{ally}
Your dad bought you a pair of drumsticks after that concert, but while you played with them for a few weeks, you soon lost interest. You had moved on.
\end{ally}
I had moved on.

I was trying to square being gay with being the type of person my parents would like. I was trying to figure out how to make friends after transfering into a school. I was trying so hard to settle down and just become someone, to just be born already.

\begin{ally}
You told your mom and Jay that, when you complained about karate in the future, they should remind you that you do enjoy it sometimes, that it just comes and goes. You just wanted to cling to something and have it stick.
\end{ally}
Computers were all well and good. They certainly offered me a route to explore so much that I might otherwise have not. They got me Danny. They got me into furry. They got me into programming.

\begin{ally}
You're still a furry. You still program. Hell, you still think about Danny. Does that not count as sticking?
\end{ally}
Oh, it definitely does, don't get me wrong. Some of the things I launched myself into did stick, even if some of them did not. I was too busy getting ready to be born to focus on what, I suppose.

And then, two weeks into my freshman year at high school, a few girls stopped me in the hall during my only free period and asked me to join choir with them.

\begin{ally}
And you said yes.
\end{ally}
Lord help me, I have no idea why, but I did.
\newpage
\end{leftcolumn}
\begin{rightcolumn*}
\noindent\includegraphics[width=2in]{assets/static/miniatures/3-1.png}

\noindent\includegraphics[width=2in]{assets/static/miniatures/4-1.png}

\noindent\includegraphics[width=2in]{assets/static/miniatures/4-2.png}

\noindent\includegraphics[width=2in]{assets/static/miniatures/4-3.png}

\noindent\includegraphics[width=2in]{assets/static/miniatures/4-4.png}

\noindent\includegraphics[width=2in]{assets/static/miniatures/5-1.png}

\noindent\includegraphics[width=2in]{assets/static/miniatures/5-2.png}
\end{rightcolumn*}
\begin{leftcolumn}

When you're a choir kid, you're a choir kid.

\begin{ally}
The first rule of the tautology club is the first rule of the tautology club.
\end{ally}
You have to understand. There's a level of identity, a level of expression that goes along with being a choir kid. It's writ on your face. It's in the way you walk. It's an aura that emanates from you. It hovers about your head in a halo. It colors your perception, and others' perception of you.

You don't do choir. You \emph{are} choir.

\begin{ally}
Just as you \textbf{are} furry?
\end{ally}
There's plenty of comparison that can be made there, yes.

\begin{ally}
Like how, fresh out of middle school, fresh out of your mom's messy divorce with Jay, fresh after your mom's diagnosis, so soon after running away, you found yourself once again largely alone. It was more complex now, too. You weren't simply physically alone. You were a newborn and you were alone in the world. You were alone on some ineffable level. You craved a family. You craved a community. You needed to not be alone. You needed those things to grow up, whether you knew it or not --- and you didn't --- so you latched onto whatever you brushed up against, arms hard around it, and you refused to let go. You refused to let it let \textbf{you} go.
\end{ally}
I\ldots{}well. Huh.

\begin{ally}
Carry on.
\end{ally}
Give me a second.

\begin{ally}
Take your time.
\end{ally}
I suppose I was going to go on to say that when you're a choir kid and a boy, something happens inside people's heads. They go a little bit crazy.

There are other identities within school, after all. There's band, of course. Band is pretty egalitarian (in some ways; obviously individual instruments have their own gender roles). There's some of the sports, too, where a girl joining the team would be quite out of place, if it's even allowed. Nerds fall along similar lines --- or fell, I suspect this is changing --- in that a girl nerd is considered something more unique.

High-schoolers, however, seem to be intensely aware of gender roles, even if they don't realize. This includes the power dynamic instilled in them in the west. A girl ``striving'' to be ``something greater'' by taking part in a supposedly masculine activity--

\begin{ally}
Nice qualification quotes.
\end{ally}
--is a curiosity, perhaps gently encouraged, perhaps the source of patronizing.

A boy ``falling back'' to ``something less'' by taking part in a supposedly feminine activity is a cause for alarm, a cause for concern, a cause for laughing and jeering and taunting.

\begin{ally}
That you transitioned later in life being, of course, irrelevant.
\end{ally}
It sort of is, it sort of isn't.

It is, because I don't think I know any other choir friends who transitioned. And not just those like me who transitioned and then dropped out of choir because boy is \emph{that} fraught.

It isn't, because in a lot of people's eyes, that's confirmation that joining choir was an early sign of my weakening masculinity. It's self-reinforcing that way.

\begin{ally}
As are a lot of social roles. Furries are nerdy because they're expected to be, and so they attract nerds. Nerds are awkward because of course they are, and so awkward kids are more likely to become nerds.
\end{ally}
When you're a choir kid and a boy and \emph{gay}, after all, well\ldots{}pff, of course. A boy in choir \emph{would} be gay.
\newpage

I tried to let go of choir when I went to university. I was all set to begin anew. I was going to live up to my parents' dreams of becoming an engineer.

That, and I heard the choir perform during All-State my senior year of high school, and they weren't that good. the All-State choirs were better. My school's choirs were better. I didn't want to tarnish my feelings on choir by having my last few years in it be less than what I was used to.

\begin{ally}
Yeah. How'd that work out?
\end{ally}
I lasted a semester.

Part of it was, of course, that I started the same year they hired Dr.~Kim, who turned the choral department around. Suddenly I had something I wanted to reach for.

Part of it was that, on graduating, one of my chosen families disappeared. I still had furry, of course, and I still had Ash and Shannon, but I was missing a core part of myself, and I wasn't strong enough to not have that in my life.

\begin{ally}
You weren't strong enough to do a lot of things, then.
\end{ally}
No, I wasn't. I wasn't strong enough to tamp down my mania or pull myself up by my bootstraps through depression. I wasn't strong enough to buckle down on my math and chemistry studies. I wasn't strong enough to treat my friends and lovers as well as they deserved. Not on my own, at least.

So I joined choir.

\begin{ally}
You did more than that. You took ownership of your life.
\end{ally}
I changed my major to music. I started taking singing lessons. I gained strength from my community, and I got better. I got strong enough to at least learn, bit by bit, how to deal with each of those things. I'm still working on some of them, but that's where I started learning.

I got strong enough to make it into voice lessons with Dr.~Morrow-King.

I got strong enough to get into Chamber Choir.

I got strong enough to go on two choir tours in South Korea.

I got strong enough to leave the music education program and move to music composition.

I got strong enough to talk to the department chair about why I wasn't getting lessons through the school.

I got strong enough to stand up to Dr.~Wohl when he was called on it and not selected to be the new professor.

\begin{ally}
Not strong enough to suffer defeat.
\end{ally}
No.

Not the one I experienced.
\newpage

My senior recital did not go well.

\begin{ally}
Understatement.
\end{ally}
It was a failure from very early on. I was commissioned to write a work for two friends in the music department. French horn and contrabass are an unlikely combination, so I figured it'd be a good challenge. It turned into a nightmare with astonishing speed.

They dictated what I wrote to a large extent, and when Dr.~David heard about it, he explained that that's not quite how it was supposed to work. I flailed and finished the piece as best as I could.

I couldn't find performers to commit to any of my pieces. When I did, they didn't practice. The two who commissioned that work from me only practiced once: half an hour before the concert itself.

The performance itself was a disaster.

\begin{ally}
You grabbed the recording and left to dinner with your mom and dad, Bob, Maurine, JD, his dad, and Diane. Diane said, as politely as she could, that many of your pieces sounded ``so dark'', and it was all you could do not to cry and say that it wasn't supposed to be that way.
\end{ally}
I gave up after that. I stopped going to class regularly. I stopped doing homework. I started programming more. I worked as many hours as I was allowed. I applied for tech jobs.

\begin{ally}
You kept singing.
\end{ally}
I did, but my heart wasn't in it.

I left music.

I stopped composing.

It took a year, but I stopped performing.

I couldn't do it.

All of the work I had put into it, all of the time and effort and blood and sweat and tears, and as soon as I had something I was proud of, I was shown just how little the world thought of me. My community didn't change, and yet it felt hateful to me. I had no guarantees at all that it would get any better, so I got out while I was at least only a little behind.

\begin{ally}
In writing, you were later told, the worst that could happen if you submitted a story was that the editors would say no. This was worse than the editor saying no. This was the editor sneering at you, looking you directly in the eye, and slowly tearing your story to shreds, long strips of paper dropping from their hands as you watched.
\end{ally}
And I had to smile as I did so. I had to smile and shake hands and gesture for the performers to bow. I had to keep talking to the audience, explaining the significance and features of each piece throughout the recital even as it continued to get worse and worse.

\begin{ally}
You stopped writing music.
\end{ally}
Why wouldn't I? Life told me what it thought of me doing so. Why would I willingly continue to fail?

\begin{ally}
You were not strong enough.
\end{ally}
I was not strong enough.

\begin{ally}
You started programming.
\end{ally}
Website after website.

\begin{ally}
You started writing.
\end{ally}
I splashed around in great heaps of words.

\begin{ally}
You promised yourself you were okay with the outcome.
\end{ally}
Seven years was enough.

\begin{ally}
And now it's seven years since you got into tech.
\end{ally}
Yes.

\begin{ally}
And you started writing music again.
\end{ally}
Yes.

\begin{ally}
A few pieces. Miniatures. Stuff you can finish without getting tired of it first.
\end{ally}
Yes.

\begin{ally}
Something to try and capture the agony and the ecstasy.
\end{ally}
Yes.

\begin{ally}
You still write for choir.
\end{ally}
Yes.

\begin{ally}
Stuff that will never get performed.
\end{ally}
Yes.

\begin{ally}
You promise yourself you are okay with this.
\end{ally}
Yes.
\newpage
