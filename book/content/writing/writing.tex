\noindent Today, my therapist asked what the plot was to this new writing project.

\begin{ally}
Me!
\end{ally}
Pretty sure you're just the antagonist.

\begin{ally}
Come now, don't say that about yourself.
\end{ally}
Right.

I stammered something about how it was more about overriding themes. I wrote about alcoholism. I wrote about dad. I wrote about all those little side-quests. ``It's about the way creativity affects and is affected by all these different things in my life,'' I said.

``Were you not creative when you drank?''

``Certainly not as much as I am now that I've stopped.''

``This sounds exhausting,'' she said.

``Well, it is, in a way. It's very easy to write. It flows onto the screen far easier than any fiction or article I've written before, but it leaves me totally beat afterward.''

\begin{ally}
You're really good at wearing yourself out. You spin in circles around the smallest things. You wind up exhausting yourself on the daily.
\end{ally}
I suppose I do, at that.

\begin{ally}
Well? You sound unsure of how you answered her.
\end{ally}
This project is sort of ill-defined.

\begin{ally}
You are ill-defined.
\end{ally}
Not going to deny that.

I'd say a lot of this project is accidental, unintentional. I stumble about at the end of your lead and, as you say, spin circles around the smallest of things. It's hard to come at this with some sort of idea of a plot. I can't even work chronologically, because if we work from the beginning of Matthew's life back in 2000, we keep having to double back and look at proto-Matthew's life before that, and to understand that, we keep having to look at all these other people.

\begin{ally}
There are too many of you.
\end{ally}
Says my ally.

\begin{ally}
Point well taken.
\end{ally}
All the same, I'm not sure that I answered her incorrectly. The core conceit of this project is one of creativity. Not anything so guided and structured as \emph{writing} or \emph{composing} or \emph{programming}, but that raw, primal thing from which the others spring.

\begin{ally}
Or seep, depending on the day.
\end{ally}
It's about the ways in which this idea, this entity impinges itself upon various things in my life. It's about the ways I shape and am shaped by it. It's about turning it back in on itself, as much as I can, and applying creativity to the idea of creativity itself.

\begin{ally}
Using words.
\end{ally}
Well, mostly words so far, yes, though I'm slowly incorporating bits of other things in there, too.

\begin{ally}
There's another metaphor to be made here. Remade, actually. You keep winding up stuck on these very abstract concepts. You keep talking about your complex feelings on your dad or on the way Margaras' death affected you or on mysticism, and then you circle them again and again, now narrowing, now widening, in an attempt to triangulate some imagined center.
\end{ally}
Writing, composing, programming, those are all inexact tools to apply toward inexact goals, though. Is that so wrong? Is it wrong to try and focus through words? Is it wrong to try and figure out more of how you think through something creative?

\begin{ally}
No, but it \textbf{is} important that you be cognizant of that fact.
\end{ally}
\newpage

\noindent All of writing, all of creativity is selfish. To take some idea or some concept and to set it down on paper and say, ``I made this'' is selfish, of course, but to then take that thing and show it to others with the expectation that they might get something out of it as well is taking that several steps further.

To sit down in front of the keyboard and to say, ``I am going to write a story about a person who runs away from home to escape her fundamentally unhappy life'' and to then take that story, post it on the internet, submit it to anthologies, publish it in a collection and attempt to get others to read it, is selfish. It's an act of improvement for the writer, sometimes on a very real basis, if there is money to be made in the process.

To sit down in front of the keyboard, however, and say, ``I am going to write a story about me when I ran away to escape my fundamentally unhappy life'', well, now we're up to three levels of selfishness. I try and nail down an idea to paper or screen and say, somehow, that it is \emph{right} and \emph{good}, I make that idea about \emph{myself}, and then I try to show that idea to \emph{others}\ldots{}

\begin{ally}
Is there no good to be had from memoirs, then? From any autobiographical content?
\end{ally}
There's certainly good to be had for the writer, for the creator. On my end, I'm making something that I both feel proud about and am learning from. I'm learning more about this art, I'm learning more about all of these problems I'm tackling --- I didn't know, for instance, just how conflicted I was about my dad until I started writing that section of the site. I though, \emph{oh, I'll write about my past and make the final point that I've had to accept that there's a certain amount of my dad that I'm comfortable having in my life, a certain level of relationship that's acceptable}. I was not expecting to learn, through writing, just how conflicted I am about him still.

\begin{ally}
And for others? Is there not enjoyment to be gained from that which you create?
\end{ally}
\emph{Disappearance} was good, I thought. I got a lot of good words sent my way from some folks that mean a lot to me for it. The story left an impact on them, they came away from it with some sort of enjoyment, or at least some level of emotional resonance.

This project, though? I don't know. there are bits that I've tried to make enjoyable. I had fun with the koans and birds. I put a lot of emotional investment into the bits about Margaras and my dad. I tried to do some fun mixed-media stuff with the fursoña animations and the mysticism stuff. I can see those being enjoyable.

\begin{ally}
And the rest?
\end{ally}
I don't know. Honestly.

\begin{ally}
What about applicability?
\end{ally}
I\ldots{}hmm.

\begin{ally}
You came into this page thinking, ``Ah yes, time to dunk on myself again'', didn't you?
\end{ally}
I guess I did. Self-deprecation runs deep in queer lives. Self-doubt plagues artists. Self-deception runs in the family.

\begin{ally}
Selfishness is defensible when it leads to entertainment, applicability, or self-improvement.
\end{ally}
To an extent. At some point, it's just narcissism. At some point gets so ``treat yourself'' that one loses sight of collective improvement.

\begin{ally}
Of course. Are you really in danger of such?
\end{ally}
Constantly, feels like.
\newpage

\noindent The first poetry I remember writing was back before high school. At some point I picked up the poetry bug and decided I was going to try my hand at it. Finding it hard, I quit after the first poem I wrote. It was something really, \emph{really} bad, too. Something where all I knew about poetry was that it should rhyme, so I sacrificed\ldots{}well, everything in search of a rhyme. Readability. Sense. It was horrifying.

\begin{ally}
You find a lot of your old stuff horrifying. Play can be creative.
\end{ally}
Sure. Play teaches us how to be creative. A lot of creativity is playful.

This went a step back from that. Play is important, sure, but it didn't make anything I'd actually call a poem. It was an innocent mockery in the same way as a boy trying on his dad's shoes and blazer.

I suppose it's a good thing that a lot of my early works are lost to time.

\begin{ally}
You filled reams of paper and countless blank books with drawings and doodles and words. You drew maze after maze on copy paper. You grew exceptionally fond of creating parabolic curves with straight lines. You went through a phase of drawing elaborate worlds of ramps and springs and houses for tiny spherical creatures with horns for mouths. Do you miss none of that?
\end{ally}
In a cute sort of way, I suppose. It was fun. I would laugh at it now, but I wouldn't find anything new to build off of it. After all, this project is built off writings after I was born. All that is from proto-Matthew.

\begin{ally}
You drew an entire comic set in the world of Garth Nix's Abhorsen trilogy, except the main characters were foxes. You filled a few notebooks with furry art, too. You kept a diary well after your dad destroyed the first one, intended originally as letters to send to your friend. You called it Julene. You later feared that would be creepy, and changed it to Kai. Do you miss none of that?
\end{ally}
I kept some, of course. Some of it is irrevocably online. I couldn't remove it if I wanted to.

I burned the journal, though. It was a remnant of proto-Matthew. It was from before I was born.

\begin{ally}
At what point did play cease being just play, then? At what point did creativity assert itself?
\end{ally}
When I started singing. When I first heard Madrigals sing during my first choir concert. When I stopped drawing and started writing. When I realized that there was more to art than playing at art.
\newpage

\begin{ally}
I assume you went looking for one of these execrable poems of yours?
\end{ally}
I did. I wasn't really able to find much from The Before Times, but I found a few from shortly after while prowling through my LiveJournal and archives of my old site in high school.

\begin{ally}
\href{https://web.archive.org/web/2005*/http://ranna.babylonia.flatirons.org/}{RedFox! Productions}, right?
\end{ally}
Gah, yeah. I was a kid, alright?

\begin{ally}
If you say so.
\end{ally}
\end{leftcolumn}
\begin{rightcolumn*}
\emph{September 26, 2003}
\end{rightcolumn*}
\begin{leftcolumn}
\textbf{I.}
\begin{verse}
  Borne through air,\\
\vin   Close my eyes.\\
  Wind ruffles hair\\
\vin   Soul sighs,\\
\vin   Heart flies;\\
  I’m the wind.

  I flow east:\\
\vin   Over the plains,\\
  Over land creased.\\
\vin   Current refrains,\\
\vin   Cloud stains\\
  As I build.

  Trees bow at my\\
\vin \vin Will\\
  To move drives me\\
\vin \vin Onward\\
  I push through\\
\vin \vin Mountains\\
  Do nothing but\\
\vin \vin Divert\\
  The rain as I\\
\vin \vin Flow.\\
\end{verse}
\newpage

\textbf{II.}
\begin{verse}
  Borne through air ---\\
  \vin Rise up high ---\\
  Driven there,\\
  \vin Earth nigh,\\
  \vin I sigh;\\
  I'm the wind.

  I flow west:\\
  \vin Past the lakes,\\
  Water my guest;\\
  \vin Thunder makes\\
  \vin Noise, wakes,\\
  As I storm.

  Sand flies at my\\
  \vin \vin Force\\
  Builds as I\\
  \vin \vin Push\\
  Across the\\
  \vin \vin Land\\
  Flows beneath my\\
  \vin \vin Self\\
  Means nothing to\\
  \vin \vin Wind.
\end{verse}
\newpage

\textbf{III.}
\begin{verse}
  Borne through air,\\
  \vin Through the night\\
  And dawn fair.\\
  \vin No fight,\\
  \vin Only flight;\\
  I'm the wind.

  I flow south\\
  \vin On the ocean,\\
  On delta's mouth\\
  \vin My motion\\
  \vin Just notion\\
  As I breathe.

  Waves break as I\\
  \vin \vin Drive\\
  Past the thin\\
  \vin \vin Sands\\
  Lift themselves to my\\
  \vin \vin Body\\
  Waxes as I\\
  \vin \vin Press\\
  Through the stillness of\\
  \vin \vin Night.
\end{verse}
\newpage

\textbf{IV.}
\begin{verse}
  Borne through air,\\
  \vin Around the world\\
  And forests I tear;\\
  \vin Ferns furled,\\
  \vin Trees burled;\\
  I am the wind.

  I flow north,\\
  \vin Across the ice;\\
  I roll forth\\
  \vin Past spice ---\\
  \vin So nice ---\\
  As I change.

  Men bask as I\\
  \vin \vin Warm\\
  Drops of rain\\
  \vin \vin Fall\\
  Colored leaves\\
  \vin \vin Shiver\\
  Because of the\\
  \vin \vin Chill\\
  Wind blows on\\
  \vin \vin Past.
\end{verse}
\newpage

\begin{ally}
It's not without its own sense of charm.
\end{ally}
I suppose. It's crude. It's a bit heavy-handed.

\begin{ally}
Your others are not?
\end{ally}
Well, okay, fair. I like to think that I've improved nonetheless.

\begin{ally}
Are these old ones not creative? Are they still just play?
\end{ally}
The more I think of it, the more I think it's that they're just too\ldots{}work. They're not creative, because they're too mechanical. I had realized that writing wasn't just play, so I stopped playing altogether.

\begin{ally}
Wrong answer.
\end{ally}
Tell me about it.
\newpage


\emph{January 11, 2003}

\begin{verse}
What hath man wrought!\\
\vin When faced with the question of love\\
\vin Or seeking peace with the answer thereof,\\
Or faced with life peril-fraught,\\
\vin Created a god, or several, to satisfy\\
\vin Some need to fulfill or deny\\
\vin \vin A lacking ---\\
\vin \vin A slacking\\
\vin On someone else's behalf,\\
\vin Or his own behalf ---\\
And on the world a question of faith brought.

And when a man, endowed\\
\vin With the ability to make his own God,\\
\vin Does so with nary a nod,\\
And finds the god shan't be cowed,\\
\vin What does he then?\\
\vin And when a group of men\\
\vin \vin Make their God\\
\vin \vin With nary a nod,\\
\vin And cow him easily, rightly\\
\vin To them, and find him tightly\\
bound, what then, with a god bowed?

What then, indeed, should a God,\\
\vin Now lesser than his creators, do\\
\vin When his creators move to gods new?\\
Is he then still a God?\\
\vin Or is that when God dies,\\
\vin Not bloated with swarms of flies,\\
\vin \vin But forgotten?\\
\vin \vin Not rotten,\\
\vin Forgotten and immortal, what then?\\
\vin Does he hope to come again,\\
Rising a second time, perhaps again to be God?

One would hope that the God, being omniscient\\
\vin Would realize he was no longer, otherwise\\
\vin Might he become destructive? Likewise,\\
A god, waiting patient\\
\vin Could become restless,\\
\vin Try to leave his creators breathless,\\
\vin \vin Again,\\
\vin \vin But then,\\
\vin Be pronounced a heretic\\
\vin By all but the hermetic\\
And others of the new God ignorant.

So hence a people divided\\
\vin Those of Whispers and those of Nanon,\\
\vin Fight to the tooth and fight to the bone,\\
Until over Whispers Nanon presided;\\
\vin And when those of Nanon took\\
\vin Speech from the Whispers so as to look\\
\vin \vin And not hear,\\
\vin \vin They here\\
\vin Those of Whispers with\\
\vin Supposed powers of myth\\
Of creation with speech's remnants provided.

So it was before the fall of Whispers that\\
\vin Faith of most all lay in technology,\\
\vin Remnants of religion lay in astrology\\
And superstitious fears like the black cat.\\
\vin Only after the fall did the faiths\\
\vin Of only the Whisperers turn to mysterious wraiths\\
\vin \vin And gods,\\
\vin \vin But the odds\\
\vin That one of the gods was taken more seriously\\
\vin Than the rest was small, and not mysteriously,\\
The small bit of Faith quickly passed as society's scat

Now, it's come that those of Nanon have all but forgotten\\
\vin Those of Whispers except perhaps in myth\\
\vin Maybe portrayed as consorting with\\
Black cats or something equally rotten.\\
\vin But for the Whisperers, the city\\
\vin Of Nanon is very real, also denial of pity\\
\vin \vin Of sunlight,\\
\vin \vin For sunlight\\
\vin Is blocked by the city directly overhead\\
\vin And the Whisperers know of only shadow instead;\\
Only death out from beneath the city to be gotten.\\

The magic that's spoken of those\\
\vin Of the Whispers, is often made\\
\vin Out to be more, but because of their stayed\\
Speech, only whispers remain in quite prose.\\
\vin So through the long stretches of time,\\
\vin The Whisperers, through long stretches of rhyme\\
\vin \vin Can make ---\\
\vin \vin Only make ---\\
\vin What they wish, with words divine,\\
\vin Benign, or malign,\\
And in their creations complete trust repose.

So begins a story, often told but never yet writ\\
\vin Of a divided people still the same\\
\vin And the rise and fall of a god played like a game.\\
While not true itself, it is truth lit:\\
\vin As men continue to create and live under gods,\\
\vin What would happen if the gods, at odds,\\
\vin \vin Warred and fell,\\
\vin \vin Raising hell\\
\vin In the process? What would happen\\
\vin In a society misshapen\\
If a wrathful god fell and no one cared a whit?
\end{verse}

\begin{ally}
Ah yes, your Keats phase.
\end{ally}
It was a mixture of Keats and Larry Niven, I think.

\begin{ally}
That is intensely Madison.
\end{ally}
Thanks.

I had recently read \emph{The Ringworld Throne}, so I was thinking about vertically stratified cities, and had also been on a Keats kick ever since reading \emph{The Hyperion Cantos}, so I decided I would write a sci-fi epic poem to support my conlang.

It's a mess.

\begin{ally}
Could be worse.
\end{ally}
Could be better.
\newpage

\begin{ally}
If you went from a mockery of creativity to a mockery of play, when did you settle down and just write a damn story?
\end{ally}
I think it wasn't too long after, actually. I wrote \href{https://writing.drab-makyo.com/fiction/all-of-time-at-once/}{\emph{All of Time at Once}} in April of 2004, and that was the first time I started to think, \emph{ah-hah, okay, there's a rhythm to this, a pace, a set of mechanics as well as an art.}

And from then on, I basically dropped writing in favor of music for months. Sure, there were a few others scattered around there. \href{https://writing.drab-makyo.com/fiction/tu-pater-et-mater/}{\emph{Tu pater et mater}} in May of 2003, and \href{https://writing.drab-makyo.com/fiction/light/}{\emph{Light}} in June of 2004, but other than that, I kind of just dropped it.

\begin{ally}
Why?
\end{ally}
I graduated. I left language arts classes behind. I went to school for an engineering major.

\begin{ally}
One you were supremely unhappy in.
\end{ally}
Right. And then when I started writing again, it was music.

I wrote a few essays I was reasonably proud of, but it took another four years before I decided to actually sit down and give writing a go in a more structured setting, and then only because of NaNoWriMo.

\begin{ally}
Ah yes, your ``boy meets girl with a twist'' story.
\end{ally}
Yeah, \emph{The Consequences of Dissonance}.

\begin{ally}
You originally named it \textbf{Coming to Terms with Being a Terrible Person}.
\end{ally}
I did, yeah. I was fresh off my relationship with Kayla and well into a relationship with Kanja, and had a head full of hatred for who I used to be.

\begin{ally}
And who you were becoming.
\end{ally}
Well, it wasn't \emph{Coming to Terms with Having Been a Terrible Person}, was it?

\begin{ally}
Fair enough.
\end{ally}
It wasn't a bad story, really, nor even that poorly written. I've even thought of revisiting it sometime. It was sort of a coming out story, but a coming-out-for-the-second-time sort of thing. Gay boy starts dating a girl and has to go through the social process of coming out as bi.

\begin{ally}
As Madison?
\end{ally}
I suppose. I went through my own series of comings-out, so maybe I have more insight into that now.

\begin{ally}
And you're less of a terrible person.
\end{ally}
Doubt.

\begin{ally}
There are perfectly cromulent reasons for you to think of yourself as a terrible person in the past, and even as a terrible person in 2008. Or even one now, really. You're just less of one.
\end{ally}
Always improving, I guess.

\begin{ally}
How did it feel to come up with a schedule, a goal, and a plan, and then to stick to it?
\end{ally}
I never finished the story.

\begin{ally}
But you won NaNoWriMo that year. You went over by eight thousand lines.
\end{ally}
I guess.

\begin{ally}
And you're dodging the question.
\end{ally}
That's why, though. It felt good while it lasted. It felt good during that hypomanic rush to actually complete something, to come up with an outline and actually work through it.

Then I finished NaNo with several hours to spare and tried to keep going, but there was something missing. I felt rudderless. I kept trying to poke at it, but I think I was working as well as I was because of the deadlines. I was still trying to balance the work with the fun that go into a creative endeavor.

\begin{ally}
Did you stop having fun, or did you stop doing the work?
\end{ally}
I think it's more complex than that. There was fun to be had in the race to the finish line. I think that's why NaNo is so popular. And doubtless it was work, of course.

But with the fun of having already won gone, I was faced with the fact that I had less outline than I had originally thought. Pantsing, as the community so eloquently puts it, may work well for some folks, but I was mostly left feeling uninspired and unmotivated once December hit. The same thing happened with \emph{Getting Lost} and \emph{Inner Demons}. I started strong enough with the basic idea as I tried to write by the seat of my pants, but without a direction or even any goal, I lost steam and wound up disheartened.

\begin{ally}
Do you not do well without goals, then? You don't seem to have one for this project.
\end{ally}
It's not necessarily that. More that, the shorter the project, the less planning that's required. I do much better with articles and short stories than I do with novels. At least so far, given the amount of planning that goes into each.

This project is working as well as it is because of my heavy reliance on these side-quests. I can break a story down into manageable chunks so that, by the time I might start losing direction, they're about overwith anyway.

Besides, I have you to help.

\begin{ally}
Me? Little old me?
\end{ally}
Yeah. It's much easier to have a conversation than it is to plan out a story. You keep taking me in directions I don't mean to go.
\newpage

\begin{ally}
So if the goal of this project is to write about the ways in which creativity interacts with various facets of your life, what are your goals when it comes to creativity itself?
\end{ally}
Huh.

I'll have to think on that one.

\begin{ally}
I'd say I'll be patient, but you know I won't be.
\end{ally}
Yeah.

I think the goals for my creativity are to find a happy medium of entertaining and applicable for others to consume as well as enjoyable for me to create.

\begin{ally}
Vague.
\end{ally}
I guess. I could list specifics, but I don't think that's quite what you're asking after.

\begin{ally}
No, vague is good. It's good to have something you'll always fall short on, because that'll always give you reason to strive for improvement.
\end{ally}
That ``if you hate who you were in the past, it's a good sign that you've improved as a person'' sort of thing?

\begin{ally}
In a way. If you hate your old work, it's a good sign you've improved as a writer, musician, developer, whatever.
\end{ally}
That makes sense.

Though I do have concrete goals. I'd like to write a book. I'd like to finish some outstanding music I've still got hanging around. I'd like to maybe work toward getting a job in something other than tech.

\begin{ally}
So what you're saying is that you'd like to be happy?
\end{ally}
I suppose so.

\begin{ally}
Good luck, kid.
\end{ally}\newpage

\begin{ally}
If this is about creativity, then tell me about composing.
\end{ally}
Shall I do so in song?

\begin{ally}
Please.
\end{ally}
No thanks, but I'll tell you all the same.
\newpage
