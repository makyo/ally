Apophenia

\begin{quote}
What?
\end{quote}

Apophenia. Connections. Imaginary lines traced from topic to topic in cheap butcher's twine.

\begin{quote}
And the topics?
\end{quote}

Imaginary. Or real, but only half remembered. I'm spinning a web.

\begin{quote}
Are you catching something?
\end{quote}

You?

\begin{quote}
Are you answering with a question?
\end{quote}

I'm unsure.

\begin{quote}
You're not catching me in that.
\end{quote}

You sound so final.

\begin{quote}
Not my department.
\end{quote}

Right. Is that a fact, then? I'm not catching you in this web. Are you the web?

\begin{quote}
Not my department.
\end{quote}

The spaces between, then. The negative spaces outlined by twine wrapped around pins. There are connections--

\begin{quote}
Or not.
\end{quote}

--or not, and there are topics, imaginary or not, and then there's you, there, in the place between. You, the liminal creature. You, defined by absence.

\begin{quote}
Presence and absence are not my department, either.
\end{quote}

Are you some cousin to apophenia, then? Some relative to that \emph{unmotivated seeing of connections accompanied by a specific feeling of abnormal meaningfulness}? Are you that numinous, abnormal meaningfulness?

\begin{quote}
I am easier to define in negatives. I am not presence and absence, but between them. Beyond them. Your ally, but not your friend. Real enough to impinge on your reality, but totally imaginary. \textbf{Not} here. \textbf{Not} doing. \textbf{Not} thinking, feeling, acting.
\end{quote}

So, are you?

\begin{quote}
Anything else is just pareidolia.
\end{quote}
