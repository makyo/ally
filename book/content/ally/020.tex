\label{ally:20}
\begin{paracol}{2}
  \begin{leftcolumn}

\begin{ally}
Let me ask this another way, perhaps. Why are we doing this? Why are we talking? Why did you start?
\end{ally}
Let's put a pin in just why exactly you're asking these questions. I'd like to know what the origin after I give you the whys and wherefores.

\begin{ally}
Okay.
\end{ally}
To the question at hand, though, I think I covered that before, right? I started this project in a fit of nostalgia\index{Nostalgia} and one of the end results of an unstoppable wave of nostalgia\index{Nostalgia} plus a sort of graphomania is the need to write about the past, and to do so in such a way as to invoke the past in the process.

\begin{ally}
I guess I'm trying to decide whether or not to believe you.
\end{ally}
What's not to believe here? I spend page after page digging through old LJ entries, old poetry, old pictures and art and logs--

\begin{ally}
Let's talk about TS.
\end{ally}
Don't derail me. These are your questions.

\begin{ally}
Point.
\end{ally}
What's not to believe about a project filled to overflowing with nostalgia\index{Nostalgia} being borne from nostalgia\index{Nostalgia}?

\begin{ally}
I don't doubt the roots in nostalgia\index{Nostalgia}, I doubt the intentionality.
\end{ally}
You doubt that I started this on purpose?

\begin{ally}
Did you summon me? Answer truly.
\end{ally}
I don't know.

\begin{ally}
I say that I've always been here, but that's only a part-way truth. That's only half-meaning drizzled over too many words. It's easy enough for someone to say that an abstract concept, a loose portion of someone's personality has always been there. Of course that's the case. Why did you summon \textbf{me}, though? Are you in need of an ally?
\end{ally}
I'm surrounded by friends and chosen family, these days. Most of them are my allies.

\begin{ally}
Well, maybe we should disentangle what exactly an ally is before we continue down the path of why you summoned me.
\end{ally}
Okay. I was going to call you my shadow, but that's not exactly right, is it?

\begin{ally}
No.
\end{ally}
You share some similarities, I guess. You have these aspects of myself that are submerged beneath the surface, usually. You see me from a distance. You know everything about me.

\begin{ally}
I do. But by its very definition, I'm not your shadow. Like I told you, I'm not your id.
\end{ally}
And like I told \emph{you}, it was a joke.

\begin{ally}
You'll have to imagine me laughing.
\end{ally}
Right.

\begin{ally}
I'm not your shadow or your id because those are not necessarily things you can see. They are the things that are, by definition, unknown and unknowable by the ego.
\end{ally}
Or at least heavily obscured. Dr Jekyll knew of Mr Hyde. Perhaps you're not my shadow, but maybe the personification of enantiodromia. Perhaps this is my process of assimilation. Perhaps this is me airing my dirty laundry.

\begin{ally}
It's not \textbf{not} that. There are enough parts of me that are opposite of you for the similarities to be more than superficial. Enantiodromia carries too many implications of balance and equilibrium, however. That there are parts of me that are opposite of you does not make me the opposite of you. You could not press us together, merge us completely, and wind up with some more complete self.
\end{ally}
Right. You'd have to be the same size as me, and you're not.

\begin{ally}
I don't have a size.
\end{ally}
You'd have to be in the same place as me, and you're not.

\begin{ally}
I don't have a place.
\end{ally}
Right. \emph{You're not person shaped,} I said. \emph{You're the shape of my hands displaced half an inch behind my own, navy blue and trimmed with sea-foam green.}

\begin{ally}
I don't have physicality. I don't have boundaries.
\end{ally}
You are bounded by me. I am your boundaries.

\begin{ally}
Are you?
\end{ally}
Can an ally move beyond a mind? Can allyship --- true, individual allyship --- move beyond the allegiance?

\begin{ally}
You tell me.
\end{ally}
I don't know that I can.

\begin{ally}
I am a liminal creature. I told you that. I'm almost a shadow but miss the mark. I'm near to the concept of a back-stage persona but miss the mark. I get close to being you, but never quite come into focus enough for the outlines to match up.\index{Liminal}
\end{ally}
Are you not just me? Just a part of me?

\begin{ally}
There is no me without you.
\end{ally}
Is there a me without you?

\begin{ally}
Can you imagine so dull a life?
\end{ally}
You're not that exciting.

\begin{ally}
Not my department.\index{ally!Not my department}
\end{ally}
Right.

So an allegiance in the orthocosmic sense\footnote{wiki.postfurry.net/wiki/Metacosmology} is a relationship two entities where they help each other. Or at least trust that they can rely on the help of the other at need. It's not contingent upon friendship, as you are so fond of saying, but that's not to say that they're mutually exclusive.

\begin{ally}
I am an endocosmic ally.
\end{ally}
Are you helping me, then?

\begin{ally}
Do you not feel my aid?
\end{ally}
I suppose I do. Sometimes it feels like you're just here to kick my ass.

\begin{ally}
Ass-kicking is well within the bailiwick of an ally. To not kick your ass when you need it would be to fail at being a good ally.
\end{ally}
I've heard that said about friends. A fair-weather friend may leave you to create your own demise, while a true friend will knock some sense into you.

\begin{ally}
True friends are almost always also strong allies.
\end{ally}
But not vice versa. I see that now. You are not my friend.

\begin{ally}
I am not your friend.\index{ally!I am not your friend}
\end{ally}
But you are my ally.

\begin{ally}
I am your ally.
\end{ally}
\newpage

\end{leftcolumn}
\end{paracol}
