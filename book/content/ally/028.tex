\begin{quote}
Do you ever find yourself getting angry at me?
\end{quote}

Quite often. Why?

\begin{quote}
How does that make you feel? Like, on one layer of remove, how do you feel about getting angry at a fictional side of yourself you talk to over the internet?
\end{quote}

I don't know, honestly. It's gotten to the point over the years that I just kind of accept that there is this part of me that I get upset at, that gets upset at me. There's this part of me that I have to yell at occasionally, and who occasionally yells at me.

Besides, not friends, remember?

\begin{quote}
Correct.
\end{quote}

So why do you ask this now?

\begin{quote}
I suppose it's come up the last few times we've sat down together. we'll start talking about one thing or another, and I'll nudge you toward talking about something more difficult, and then you'll get all huffy.
\end{quote}

Well, if the things you are pushing me toward are difficult, do you really expect anything other than that? You're pushing me to do painful things to myself, to dredge up deep fears and memories I'd convinced myself I'd buried for good.

\begin{quote}
It is difficult to forget things on command. Dear, also, the tree that was felled taught you that, remember?
\end{quote}

I had honestly forgotten about the dress. Or at least I thought I had. It was a surprise to have it brought up again.

\begin{quote}
See? I'm being useful.
\end{quote}

Is that your department?

\begin{quote}
No, but you can pretend it is if you want.
\end{quote}

I might just.

So do you try to make me angry?

\begin{quote}
Not my--
\end{quote}

Department?

\begin{quote}
Not my responsibility. I'm not responsible for your moods. I'm not even technically responsible for pushing you to better yourself. I'm just here to make sure you wind up being a complete person. Entire and whole.
\end{quote}

How does one do that?

\begin{quote}
Every ally does it in a different way. I do it by talking. By asking and poking and prodding.
\end{quote}
