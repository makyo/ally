When I was young, back before I knew what mental health entailed, what anxiety and abuse and depression really meant, I was convinced I was having semi-regular mental breakdowns. That was the phrase I used then, because I was unsure of what it meant to have a panic attack.

This was before LiveJournal, of course. This was before I was writing on the internet, or even really on the internet at all. This was before you.

\begin{quote}
No, it wasn't.
\end{quote}

Right.

When I \href{https://writing.drab-makyo.com/blog/running-away/}{ran away}, my dad found my paper journal. I had kept it infrequently, as something about daily journaling to a seventh-grader felt dishonest, stupid. What could I possibly write about?

In the journal, I mentioned on a few occasions that I'd had a mental breakdown. My dad called me several times over the next few days after my mom found me, and in one of those calls, he yelled at me about that. ``Do you really think you're crazy?'' he said. ``Do you need to be taken to an asylum?''

I told him no. I whispered it. I murmured it. I wasn't crazy. I didn't need to go to an asylum. I just felt like time stopped for me and the world around me sped up. I just felt like I was holding on by the barest amount of friction on my fingertips. The whorls of my fingerprints providing my only grasp on reality.

\begin{quote}
That was me saying hi.
\end{quote}

Blunt-force greeting?

\begin{quote}
I was quiet as a mouse.
\end{quote}

I have the words now. I have the vocabulary. I can say derealization, depersonalization, dissociation. I can say panic attack and anxiety and depression and hypomania. I can say \emph{ah, \textbf{this} is what is happening now}.

\begin{quote}
You have emotions now, is what you have. Those were your mental breakdowns.
\end{quote}

Dad didn't believe in those. Not for boys. \emph{Mood's a thing for cattle and loveplay}, right? Emotions are for women.

\begin{quote}
He was half-right.
\end{quote}

I suppose he was.
