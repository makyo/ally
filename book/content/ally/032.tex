\label{ally:32}
\begin{paracol}{2}
  \begin{leftcolumn}
\begin{ally}
My turn.
\end{ally}
Shoot.

\begin{ally}
You said: ``you are not the project, but there is no project without you.''\footnote{Ibid.}\index{ally!throwing stones}
\end{ally}
Yes, that applies to us both.

\begin{ally}
You have spoken to your compulsive need to overshare, and you have spoken to the fact that the act of writing and selling a book is, in its own way, the act of selling yourself. \textbf{Restless Town} and \textbf{Eigengrau} are not so firmly tied to you as this, however. \textbf{Rum and Coke} certainly is not. I don't think you could say the same about this. Speak to your ties to this project.
\end{ally}
Do you suspect that it is too personal to sell?

\begin{ally}
You asked my feelings on the matter.
\end{ally}
I'm of two minds on the matter.

\begin{ally}
Har har.
\end{ally}
Thank you. Seriously, though, I can see two different sides of this.

I feel like I'm putting my maddest edges\index{Mental health}, as Jon Ronson puts it, on display. In the process of working on this project, I was forced to confront some of the most difficult aspects of my life by its very nature.

In the process of pulling the book together, I was forced to reread much of what I had written, and there are parts of it where my words burn too hot, or get too slippery to hold. There's a feverish quality to them. It's something that felt good to write purely for the sensation of it bursting forth from me in uncontrollable torrents.

These maddest edges are something that are integral to the project. You, after all, are one of the, and this project is named after you.

\begin{ally}
Is it mad to have a six month long therapy session with an imaginary interlocutor?
\end{ally}
This is both more and less than that, and you know it.

\begin{ally}
Yes.
\end{ally}
It would be `mad', I suppose, were I to believe that you were an \emph{actual} interlocutor. It would be `mad' were I to present these things as a universal worldview. It would be `mad', awful as that word is, were I anything but deliberate with this project.

As it is, I summoned you. I started pulling down bits of nostalgia\index{Nostalgia} when my I was shutting down my Dreamhost account, when I went to lock my ancient LiveJournal. I got the idea to write, so I did. It was a deliberate effort.

\begin{ally}
Is that mad?
\end{ally}
\ldots{}huh.

\begin{ally}
A question for another time. Tell me of your two minds.
\end{ally}
Right.

On the one hand, I read back through all of this and I find myself tasting blood. Who is this Madison? Is she okay? She seems to be having a rough time of things sometimes, and at others she doesn't seem wholly sane, or at least not wholly healthy. That's a scary thing for someone to put on display. What could possibly lead someone to do that? Some strange form of self-flagellation?

And on the other, while I'm most certainly not wholly healthy, I am, at my core, a storyteller. A young one, and certainly one of uneven quality, but I'm learning and improving by doing. There are stories to be told here, with my life, and that's what I'm doing. I'm making them interesting. I'm embellishing some of them. Hell, I'm making some stuff up wholesale. And I'm doing all of this for the specific purpose of it being read as a story.

In the end, it's the storyteller that wins out over the concerned, private individual. If I can't \emph{not} overshare, if I must compulsively tell stories, then I'm going to tell the stories I have and I'm going to make them worth reading.

\begin{ally}
A friend once asked Maddy, ``Why do you shout carefully constructed, thoroughly edited, well rehearsed speeches into the void?''\index{Koan}
\end{ally}
Maddy replied, ``It pays the bills.''
\index{ally|)}
\newpage
  \end{leftcolumn}
\end{paracol}
