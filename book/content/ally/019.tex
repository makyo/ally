The tragic core to all this, to this whole project, is that I am not an interesting person. Or maybe interesting, but unremarkable.

\begin{quote}
You're in a mood.
\end{quote}

\emph{Coming to terms with being a terrible person}, I wrote, but I'm not even that. I'm just a person.

I'll be the first to admit that I'm largely just a boring person. I know that. There's nothing remarkable about my life. Middle class, middling intelligence, average looks --- at least for a trans girl --- okay sense of humor, no unusual challenges, unless the movement disorders count.

\begin{quote}
So?
\end{quote}

What's this, then? A memoir? What would that accomplish?

\begin{quote}
Validation? I've already mentioned that.
\end{quote}

What would the written account of an ordinary life validate?

\begin{quote}
Sometimes it's worthwhile just hearing that ordinary people living ordinary lives can get by in the world. That despite being trans, despite feeling like garbage sometimes, you can still function. That even the drabbest of makyō still have stories to tell.
\end{quote}

I suppose that's fair. Literary fiction exists separately from genre fiction, as silly a distinction that is to make, because of the validation we find in the unfantastic.

\begin{quote}
Where is this heading? What is the future? What are we leading to?
\end{quote}

In the context of this project, or just life in general?

\begin{quote}
Is there an end? A goal?
\end{quote}

I'm not sure.

\begin{quote}
What will the last page say?
\end{quote}

{[}\ldots{}{]} Endings were writ on your face, your hands, and your steps --- your very pace spoke of completion.

\begin{quote}
Are you thinking of ending this project?
\end{quote}

Not at all. I've got a list of side quests I need to complete in order to make you happy, and their very nature makes it easy to complete. One or two thousand words, an hour or two's conversation with you, and then they're done and I don't have to pick up where I left off.

I'm just tired.
