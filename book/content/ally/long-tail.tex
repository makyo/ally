\label{ally:19}
\begin{paracol}{2}
  \begin{leftcolumn}
\index{ally|(}
\noindent The tragic core to all this, to this whole project, is that I am not an interesting person. Or maybe interesting, but unremarkable.

\begin{ally}
You're in a mood.
\end{ally}
\emph{Coming to terms with being a terrible person}, I wrote, but I'm not even that. I'm just a person.

I'll be the first to admit that I'm largely just a boring person. I know that. There's nothing remarkable about my life. Middle class, middling intelligence, average looks --- at least for a trans girl --- okay sense of humor, no unusual challenges, unless the movement disorders count.

\begin{ally}
So?
\end{ally}
What's this, then? A memoir? What would that accomplish?

\begin{ally}
Validation? I've already mentioned that.
\end{ally}
What would the written account of an ordinary life validate?

\begin{ally}
Sometimes it's worthwhile just hearing that ordinary people living ordinary lives can get by in the world. That despite being trans, despite feeling like garbage sometimes, you can still function. That even the drabbest of makyō still have stories to tell.
\end{ally}
I suppose that's fair. Literary fiction exists separately from genre fiction, as silly a distinction that is to make, because of the validation we find in the unfantastic.

\begin{ally}
Where is this heading? What is the future? What are we leading to?
\end{ally}
In the context of this project, or just life in general?
\end{leftcolumn}
\end{paracol}

\renewfontfamily\pagenumfont{Gentium Book Basic}[Color=CCCCDCFF]
\backgroundcolor{c[1]}[rgb]{0,0,0.2}
\backgroundcolor{C[1](0.6\columnsep,10000pt)(10000pt,10000pt)}[rgb]{0,0,0.2}
\begin{paracol}{2}
  \begin{rightcolumn*}
    \label{dad:as-a-person}
\index{Family!dad|(}
\renewcommand*{\footnoterule}{%
  \kern-3pt%
  \color[HTML]{ccccdc}\hrule width 0.4\columnwidth
  \kern2.6pt}
\fontspec{Gentium Book Basic}[Color=CCCCDCFF,Ligatures=TeX]
\renewfontfamily\allyFont{Merriweather Sans}[Scale=0.9,Color=BBBBCBFF,Ligatures=TeX]
\begin{quotation}
  Dad,

  It's been a while since we've had the chance to catch up on things. That's on me; not only has life been pretty nuts of late, but I've also kind of lost track of keeping in touch with family and a whole slew of friends.

  There are a bunch of reasons for that. Chief among them is probably that I'm struggling a lot with figuring out where I stand with folks. It seems like there's this whole class of people that I'm just not sure how to interact with. In our case, it's sort of, ``Are we friends? Are we family? Is it cool for us to just chat? Should our relationship be cordial? Friendly? Chatting only when necessary, or regularly?'' Lots of questions like that.

  I think a lot of the reason I've been asking myself a lot of these questions lately has been that I've kinda hit one of those mid-life crisis moments. I know 34 isn't exactly a number preceded by `the ripe old age of', but I suppose this is the type of thing that can strike at just about any time. At least, that's what my therapist promises me.

  I burned out pretty hard at Internet Archive, and left after only a year to go work as a contractor for a small software company based in the UK (still working remote, natch) called New Vector. They work on encrypted communications stuff, with their primary selling point being that their service is federated --- anyone can run a server and talk to anyone else on other servers. It makes for a much more robust network.

  Neat as the opportunity was, I hated it. Every time I opened up my code editor, I'd just stare at it and think about how much I hated my job. Then I'd start feeling hopeless, because this thing I was hating was my chosen career path.

  Burnout's a hell of a drug, I guess.

  Neither work nor I were happy with me there, so rather than renewing my contract, I decided to start looking elsewhere. Rather than looking for yet another software job that I'd probably hate, I started looking at tech writing positions. It'd be a lot of working through a piece of software --- both using it and looking at the code --- and writing documentation, blog posts, etc. My biggest lead right now is actually for a company I used to work for, helping to write the curriculum for their certification program, similar to Microsoft's A+ cert.

  I've been writing and editing a lot lately. I've got a small publishing publishing company that I run (very small; only have three books out so far), and three books of my own out, with another one coming out in a few months. I figure since that's the direction my hobbies have gone, might as well find a synthesis of that and the thing I'm good at in terms of dayjobs. Tech writing sure as hell makes more money than publishing, after all.

  Things are going alright on my end other than that. Found a meds combination that is working really well for bipolar (and doesn't cause any more of those movement disorders!), and a hormone regimen that's been stable for a few years now. We went down to San Jose, CA around my birthday for a convention and to meet up with some of our polycule (if you graph them out, polyamorous relationships start to look like molecules, so the name has stuck). Was good to have a little vacation.

  James is doing alright as well, though he's moved to working almost entirely with property management and real estate these days, rather than machining. He's been working through some health fiascoes. Found out he was low on testosterone, and supplementing that helped out a ton. He was back to the James I met back in 2005 or so. Then he found out he has celiac disease, so we had to go gluten free. Now he's got twice the energy he used to, since his body is actually digesting nutrients.

  The dogs are both slowing down. They're getting pretty old (at least for German Shepherds), and both have arthritis. Still, they're happy and lazy. It seems like a good life. We also adopted a piece of shit cat, dumb as dirt and soft as hell. I love her.

  How are things on your end? Been a bit since we've caught up about the day-to-day stuff. Curious to hear how work is going. How's Maurine?

  It's a bit early yet, but happy upcoming birthday! Hope it treats you well.

  Love,

  Madison
\end{quotation}

\newpage

\begin{ally}
  Why?
\end{ally}

Why what?

\begin{ally}
  Why send this? Why email your dad? Why now?
\end{ally}

This project, mostly.

\begin{ally}
  My fault?
\end{ally}

Well, maybe the book's. The possibility that he may wind up with a copy.

I talk about my dad off and on during therapy. I suppose he comes up with some frequency because of all the hangups I still have. It seems like ever few months I'll discover a new one.

\begin{ally}
  Ain't that just the way of things.
\end{ally}

I think it's a credit to my therapist, honestly. Were I paying all that money to simply go chat about my week with someone, getting nothing out of it but company, I'd feel quite let down by the whole process. That I'm coming away from sessions with improved understandings of myself is a good thing.

That said, a lot of the time those therapy sessions where dad has come up have been productive mostly for me understanding the present through my past without necessarily moving forward.

\begin{ally}
  Do you blame your therapist for that?
\end{ally}

Of course not. She's wonderful, and has helped me out a ton.

I just also think that she's got a different approach to this than you do. Or I do. Whatever.

\begin{ally}
  Whatever.
\end{ally}

On her end, she is happy to help me explore and offer suggestions, but she's less keen on beating me up. She is an ally, yes, but a bit more of a friend than you are. She is happy to help me move forward, but also happy to let me just learn.

\begin{ally}
  ``I think at some point I just need to accept that it's not worth the trouble trying to reconnect with him,'' you said.
\end{ally}

Yes, to which she responded, ``I suppose that's true, though is that something you'd recommend others who are transitioning?''

``Yes,'' was my immediate response. ``At some point, with family, it has to be okay to make the cost-benefit analysis and decide whether it's even worth it to keep trying.''

\begin{ally}
  And did you make that analysis?
\end{ally}

Yes.

\begin{ally}
  And was it worth it?
\end{ally}

No.

\begin{ally}
  So, why the sudden change of heart? Why now?
\end{ally}

That Madison --- the one who struggled to square living earnestly with lying to dad --- is dying. She may have died already. Maybe she died on August 9th of last year, when she first decided to summon her ally.

\newpage

\begin{quotation}
  Madison

  What a nice surprise. Thank you so much for all of the information and insights as to how you have been and are doing. I loved it!

  You happened to catch me down in Tucson. I come down here by myself when Maurine is working just to get used to working remotely with my work crew. It's a bit clunky but the VPN and various tools make it doable. Hope one day in the not too distant future to be able to come down to Tucson for a few months over the winter months and work. I'd have to go back a week a month for meetings but otherwise I should be able to pull it off.

  Without being maudlin, I will always love you and am proud of you and your life. There are no thoughts here but good ones and hope that you are comfortable with our relationship. I'd love to hear from you more but know how life can get in the way. Maurine and I both had a great time seeing you two over Thanksgiving. I know the dinner was a bit over the top but I still think about the visits then. I hope to get out to Seattle again later this year and visiting you was on the top of my wish list. I often think of the times we both went through while you were growing up and I have to smile at the fun we had. Hopefully there is more ahead. You will always be a part of me.

  I know the burnout feeling. I can start to feel that creeping into my work routines. The clients seem to be more demanding and the work more of a grind. Luckily I have two employees that pick up a huge amount of the burden now. I hope to slowly turn much of the day to day stuff over to them. The problem is that Greg is still around and does little if any work. That salary stream keeps me from picking up the additional employee that I need to really step back and relax. Anyway, we have paid off both the Lakewood and Tucson houses so the slow retirement plan is starting to look like something that can be done. Now I just need to learn how to value my self-worth without it being tied to the company.

  Overall I still am pretty healthy. I am getting over a stomach reflux problem that was probably related to stress and my getting high. Got both of those sources under control and picked up my exercise routine. That has helped quite a bit. Only smoke a couple of hits at night now and that's it. The exercise also seems to help the hand tremors that I have at times. The doctor thinks it was related to anxiety but the drug they prescribed did not go with my life. So I continually to learn to relax and take things easier. You'd think I would have learned all of this by now. Life can be a squirrely thing.

  Maurine is doing well and is probably closer to a retirement change than I am. They made her the shop teacher at the school so that has given her a new lease on work but she is getting tired of that also. The kids are not what they used to be. They talk back and argue with her constantly and many are really rude. She is lucky that she hasn't lost her cool and slapped the shit out of one of them so far. As a result, she is going to let her teaching certificate expire next year so she has about a year and a few months left to work. I think she will probably become a substitute teacher and work part time. She will also be coming down to Tucson more. I told her that you wrote and she wanted to make sure that I let you know she says hi and is looking forward to seeing you again.

  A publisher --- Who'da thunk.

  Love Dad
\end{quotation}

\newpage

\begin{ally}
  Is this what you were expecting?
\end{ally}

Not at all. Or perhaps some very small part of me was hoping for something like this, but it was one of those 'hope against all hope' type things.

\begin{ally}
  What were you expecting?
\end{ally}

I suppose I was expecting something along the lines of what I got after my dumb-as-hell coming-out letter: an acknowledgment of receipt and thank you for the information. Perhaps I was expecting a phone call in return, and I'm not sure whether that would be better or worse than a response, no matter how curt.

Were I to get a call, I would have frozen up and not been able to talk about anything of import.

\begin{ally}
  And so what does this mean?
\end{ally}

I suppose it means a few things.

It means that I was spending rather a lot of time catastrophizing. That I spent all of my time defaulting to the idea that he was somehow unwilling to engage with me on a very real level may have been informed by times in the past, but clearly is not the default.

This, in turn, means that I need to somehow reorganize my conceptualization of my dad around this new version of reality. I was holding this picture of him in my head that was based solely on those times with him that left the strongest impression. My view of him was limited to the man I ran away from juxtaposed against the man who was finally able to interact with me on an equal level when we were able to drink together. It was not based on an interpretation of him as someone who was constantly improving --- constantly striving to improve --- and who, yes, may have been able to interact with me better as an adult but who nonetheless enjoyed the fact that I was his kid.

\begin{ally}
  And?
\end{ally}

And it also means that there is far more that my dad doesn't know about me that I had first imagined.

\newpage

Does my dad know that I'm trans? Does he truly, \emph{truly} know? Does he accept it?

Does my dad know know about HRT? About surgery?

Does my dad know I'm poly? Is that something he has internalized?

Does my dad know about self-harm? Does he know about suicide? Has he seen the scars?

Does he know about you?

\begin{ally}
  Does it matter?
\end{ally}

The joy that I felt at his response is tempered by a whole new set of anxieties.

\begin{ally}
  Did you feel joy?
\end{ally}

Honestly? Yeah.

It was a relief, in a way to see that he was not the dad I grew up with. That I could see change in him is not only something that's good for our relationship, but also something that makes me feel better about myself. It makes me think that I, too, have the ability to change, to grow and become a better person.

\begin{ally}
  Was that in doubt?
\end{ally}

Yes.

\begin{ally}
  Really? Given this project? The core theme of the death of Matthew?
\end{ally}

Oh yes. So many times when I was writing about that, it felt like I was writing about someone else. I feel so stuck sometimes. So static. It's easy to lose perspective until it's rubbed in your face.

\begin{ally}
  Will you talk to him about your anxieties?
\end{ally}

Yes. After hearing back from him, I think I probably should, too.

Just over time.

Slowly.

Carefully.

\begin{ally}
  Take your time.
\end{ally}
\index{Family!dad|)}

  \end{rightcolumn*}
  \begin{leftcolumn}

\begin{ally}
Is there an end? A goal?
\end{ally}
I'm not sure.

\begin{ally}
What will the last page say?
\end{ally}
{[}\ldots{}{]} Endings were writ on your face, your hands, and your steps --- your very pace spoke of completion.

\begin{ally}
Are you thinking of ending this project?\index{ally!meta}
\end{ally}
Not at all. I've got a list of side quests I need to complete in order to make you happy, and their very nature makes it easy to complete. One or two thousand words, an hour or two's conversation with you, and then they're done and I don't have to pick up where I left off.

I'm just tired.

\label{ally:20}

\begin{ally}
Let me ask this another way, perhaps. Why are we doing this? Why are we talking? Why did you start?
\end{ally}
Let's put a pin in just why exactly you're asking these questions. I'd like to know what the origin after I give you the whys and wherefores.

\begin{ally}
Okay.
\end{ally}
To the question at hand, though, I think I covered that before, right? I started this project in a fit of nostalgia\index{Nostalgia} and one of the end results of an unstoppable wave of nostalgia\index{Nostalgia} plus a sort of graphomania is the need to write about the past, and to do so in such a way as to invoke the past in the process.

\begin{ally}
I guess I'm trying to decide whether or not to believe you.
\end{ally}
What's not to believe here? I spend page after page digging through old LJ entries, old poetry, old pictures and art and logs--

\begin{ally}
Let's talk about TS.
\end{ally}
Don't derail me. These are your questions.

\begin{ally}
Point.
\end{ally}
What's not to believe about a project filled to overflowing with nostalgia\index{Nostalgia} being borne from nostalgia\index{Nostalgia}?

\begin{ally}
I don't doubt the roots in nostalgia\index{Nostalgia}, I doubt the intentionality.
\end{ally}
You doubt that I started this on purpose?

\begin{ally}
Did you summon me? Answer truly.
\end{ally}
I don't know.

\begin{ally}
I say that I've always been here, but that's only a part-way truth. That's only half-meaning drizzled over too many words. It's easy enough for someone to say that an abstract concept, a loose portion of someone's personality has always been there. Of course that's the case. Why did you summon \textbf{me}, though? Are you in need of an ally?
\end{ally}
I'm surrounded by friends and chosen family, these days. Most of them are my allies.

\begin{ally}
Well, maybe we should disentangle what exactly an ally is before we continue down the path of why you summoned me.
\end{ally}
Okay. I was going to call you my shadow, but that's not exactly right, is it?

\begin{ally}
No.
\end{ally}
You share some similarities, I guess. You have these aspects of myself that are submerged beneath the surface, usually. You see me from a distance. You know everything about me.

\begin{ally}
I do. But by its very definition, I'm not your shadow. Like I told you, I'm not your id.
\end{ally}
And like I told \emph{you}, it was a joke.

\begin{ally}
You'll have to imagine me laughing.
\end{ally}
Right.

\begin{ally}
I'm not your shadow or your id because those are not necessarily things you can see. They are the things that are, by definition, unknown and unknowable by the ego.
\end{ally}
Or at least heavily obscured. Dr Jekyll knew of Mr Hyde. Perhaps you're not my shadow, but maybe the personification of enantiodromia. Perhaps this is my process of assimilation. Perhaps this is me airing my dirty laundry.

\begin{ally}
It's not \textbf{not} that. There are enough parts of me that are opposite of you for the similarities to be more than superficial. Enantiodromia carries too many implications of balance and equilibrium, however. That there are parts of me that are opposite of you does not make me the opposite of you. You could not press us together, merge us completely, and wind up with some more complete self.
\end{ally}
Right. You'd have to be the same size as me, and you're not.

\begin{ally}
I don't have a size.
\end{ally}
You'd have to be in the same place as me, and you're not.

\begin{ally}
I don't have a place.
\end{ally}
Right. \emph{You're not person shaped,} I said. \emph{You're the shape of my hands displaced half an inch behind my own, navy blue and trimmed with sea-foam green.}\index{Numinous!colors}

\begin{ally}
I don't have physicality. I don't have boundaries.
\end{ally}
You are bounded by me. I am your boundaries.

\begin{ally}
Are you?
\end{ally}
Can an ally move beyond a mind? Can allyship --- true, individual allyship --- move beyond the allegiance?

\begin{ally}
You tell me.
\end{ally}
I don't know that I can.

\begin{ally}
I am a liminal creature. I told you that. I'm almost a shadow but miss the mark. I'm near to the concept of a back-stage persona but miss the mark. I get close to being you, but never quite come into focus enough for the outlines to match up.\index{Liminal}
\end{ally}
Are you not just me? Just a part of me?

\begin{ally}
There is no me without you.
\end{ally}
Is there a me without you?

\begin{ally}
Can you imagine so dull a life?
\end{ally}
You're not that exciting.

\begin{ally}
Not my department.\index{ally!Not my department}
\end{ally}
Right.

So an allegiance in the orthocosmic sense\footnote{wiki.postfurry.net/wiki/Metacosmology} is a relationship two entities where they help each other. Or at least trust that they can rely on the help of the other at need. It's not contingent upon friendship, as you are so fond of saying, but that's not to say that they're mutually exclusive.

\begin{ally}
I am an endocosmic ally.
\end{ally}
Are you helping me, then?

\begin{ally}
Do you not feel my aid?
\end{ally}
I suppose I do. Sometimes it feels like you're just here to kick my ass.

\begin{ally}
Ass-kicking is well within the bailiwick of an ally. To not kick your ass when you need it would be to fail at being a good ally.
\end{ally}
I've heard that said about friends. A fair-weather friend may leave you to create your own demise, while a true friend will knock some sense into you.

\begin{ally}
True friends are almost always also strong allies.
\end{ally}
But not vice versa. I see that now. You are not my friend.

\begin{ally}
I am not your friend.\index{ally!I am not your friend}
\end{ally}
But you are my ally.

\begin{ally}
I am your ally.
\end{ally}
\newpage

\label{ally:21}

\begin{ally}
When you started this project, several people asked if you were okay.
\end{ally}
Yes.

\begin{ally}
Were you?
\end{ally}
I think so. I was swinging up toward hypomania, and plowing heedlessly through nostalgia\index{Nostalgia}. Some of it was good, some of it was bad, but I don't think that had much bearing on me starting the project.\index{Mental health!bipolar!mania|(}

\begin{ally}
Robin\index{Relationships!Robin} asked if you were okay. ``I just want to make sure,'' she said once. ``You asked me to check in on you if you ever started talking about geese.''\index{Numinous!birds}
\end{ally}
Perhaps this has a similar feel to it. A similar scent of ritual, a similar flavor of mysticism\index{Numinous}, a similar sense of some other reality vignetting my vision.

\begin{ally}
lorxus\index{Relationships!lorxus} asked if you were okay. ``People normally write memoirs at the ends of their lives.''
\end{ally}
Life is a series of beginnings and endings dovetailed messily together.

\begin{ally}
There is a final ending, though.
\end{ally}
I don't think I'm near that, despite what passive ideation might tell me. I'm not writing some drawn out farewell.

\begin{ally}
So, why are we talking, you and I? Where is this going?
\end{ally}
We're talking because this project, self indulgent as it is, is leading me to confront and process a lot of different things, which I'd call a net positive. We're talking because how can I know what I think until I say --- or write --- it? We're talking because I've got a lot on my mind.

This is going nowhere.

\begin{ally}
I don't know whether to be proud or insulted by that.
\end{ally}
Can you feel either?

\begin{ally}
Not my department. The metaphor is still useful.
\end{ally}
Well, fair enough. I didn't mean that idiom, anyway. This is going nowhere because it's a project that needn't have a direction.

It's not a directed thing.

It is a river.

It is the movement of the tides.

It's guided only by gravity and the lay of the land.

It is its own \emph{musica universalis}.

It's a conversation.

\begin{ally}
Conversations have direction.
\end{ally}
Not all of them.

It's one of those late-night conversations that go where they will, in which sometimes very little is said.

It is not a minded thing. It has no autonomy and yet has no guiding force. No sapient guiding force, at least.

It is a way. It is a path, and yet the path is not the walker.

\begin{ally}
This is going nowhere.
\end{ally}
Maybe, but maybe that's the point.
\newpage
\label{ally:22}

\noindent My turn.

\begin{ally}
Shoot.
\end{ally}
Why ask this now? Why ask about the core instead of a side quest?

\begin{ally}
I did. I asked about TS.
\end{ally}
Don't deflect.

\begin{ally}
Okay.
\end{ally}
Why ask about the project? Why ask about yourself?

\begin{ally}
You had job interviews. You had the convention. You're visiting Barac. You stopped writing for a bit.\index{Relationships!Barac}
\end{ally}
I started again, didn't I?

\begin{ally}
Yes. Hypomania is fading into the comfortable static of a ground state, though. You're \textbf{still} writing. That's why I'm asking. Why are you writing this if you're not hypomanic?\index{Mental health!bipolar!mania|)}
\end{ally}
I wrote a bunch of \emph{Restless Town} when I wasn't hypomanic.

\begin{ally}
Yes.
\end{ally}
I wrote some of \emph{Rum and Coke} when I wasn't hypomanic.

\begin{ally}
It shows, in the last one.
\end{ally}
I've grown as a writer. I've grown as a person. I can continue projects whose inception lay in hypomania.

\begin{ally}
And yet you say that you know a thing is right if you feel the same when depressed as when hypomanic. You can tell a decision is worth making if something other than strange energies birthed it.
\end{ally}
Yes.

\begin{ally}
Did strange energies not birth me?
\end{ally}
I don't know. Maybe. I don't think they birthed this project, though. I think this project is\ldots{}hmm.

\begin{ally}
An honest one? A true one? A worthwhile one?
\end{ally}
Sort of.

Maybe I think it's an earnest one. One that was borne out of a real desire, birthed by a need beyond what might be imbued by hypomania. A more grounded need, not one based in those non-Newtonian laws that govern that other space, where mechanics break down and strange energies spring, palladial, into being.
\newpage
\label{ally:30}

\noindent Does death take more than one form? Can it be anything other than it is? Can it sneak up on you while you aren't looking, and then when next you take a breath, you realize that you are in some afterlife?

\begin{ally}
I suppose it must, given this lead in.
\end{ally}
Have I died? Has some part of me already rotted and sloughed off? Is this, in some very literal way, an afterlife?

\begin{ally}
Do you feel as though, another seven\index{Numinous!seven} years having passed, you are moving on from the life that you built up?
\end{ally}
Yes.

\begin{ally}
Then I see no reason not to label it as such.
\end{ally}
Perhaps lorxus was right. Perhaps I am writing this at the end of a life.\index{Relationships!lorxus}

\begin{ally}
What are you leaving behind?
\end{ally}
I think I'm leaving behind that bit of me who was struggling to live earnestly.

\begin{ally}
Are you not, now?
\end{ally}
No, I think I am. Or, well, I think I am living fairly earnestly. I think what has happened over the last few years is that the struggle changed its shape.

The Madison who was struggling to come to terms with a post-Matthew life is not me any longer. She spent the last seven years mourning him, in a way. She spent the last seven years figuring out how to live without him, throwing away his stuff, leaving behind family and homes and states.\index{The Death of Matthew}

\begin{ally}
Is this her memoir? Or yours?
\end{ally}
I don't know, honestly.

All I can say is that, for some reason, at some point while working on this project, I might have died. I have entered a liminal space once again. It's a different one, to be sure, but it's somewhere in between who I was and some undefinable potential self.

Perhaps some early whiff of this liminality is what got to start this project in the first place, to summon you. Perhaps it was burnout reaching a head that signaled the death of that version of me.

Perhaps I have simply, like Theseus' well-worn ship, become something completely new while I wasn't looking.\index{Ship of Theseus}
\newpage

\label{ally:31}
\noindent I've been pulling this all into a book. Like, a physical one. A paperback.\index{ally!meta}

\begin{ally}
I know.
\end{ally}
How do you feel about that?

\begin{ally}
Not my department.\index{ally!Not my department}
\end{ally}
That feels like an evasion to me. You had opinions on me streaming the process of writing. You had opinions on the process itself: you called me out on writing stuff in commit messages, on having our conversations in comments in the source code. You had opinions on me buying the domain name. Do you have no opinions about our words on something to be bought and sold?

\begin{ally}
A friend once asked Maddy, ``Why do you shout into the void?''\index{Koan}
\end{ally}
I write all of this down because the very act of putting it into words brings a sense of clarity that I lack without. By taking these moments of my life and articulating them, I almost automatically get another viewpoint on them.

\begin{ally}
And by articulating them as a conversation, you get two. That is not the friend's question.
\end{ally}
No, I suppose it isn't.

I write for the clarity, but I share out of some perverse need. \emph{The chances that ally will pick up any sizeable audience are slim, so I almost feel like I'm publishing it as an extension of my compulsive need to overshare,} I prophesied. I share because I have to.

\begin{ally}
Does Maddy shout into the void because she must shout into the void?
\end{ally}
Perhaps. Sometimes.

Sometimes I have to speak so that someone will hear me out of some desire for feeling justified. I need to be heard, to be seen, so that even if I'm going through something alone, others will know that I am doing so. I need to be witnessed.

\begin{ally}
There is power in the word, as you say, but there is also power in the act of speaking it. There is no value-judgment for me or anyone else to make in that. Words have power, speaking has power.
\end{ally}
There is value-judgment in the content, though.

\begin{ally}
Yes. There is value-judgment in intent, as well. That you are publishing these words is not something that I \textbf{can} have an opinion about.
\end{ally}
Okay. What about my intent?

\begin{ally}
Your compulsive need to do overshare is an implicit part of our relationship.
\end{ally}
Shall I throw your words in your face?\index{ally!throwing stones}

\begin{ally}
By all means.
\end{ally}
\emph{Am I something to be bought and sold? Am I something to be traded and marketed?}\footnote{ally.id/aside/2}

\begin{ally}
Have you answered the question? \textbf{Am} I something to be bought and sold? Me, here, being a part of yourself.
\end{ally}
Since having that conversation, I've released two books, and yes, I suppose you are. \emph{I} am. We are a brand to be built up and marketed, parceled up and sold to any comer.

``The tragic core to all this,'' I wrote, ``is that I'm not an interesting person.'' I \emph{am} a writer, though. This will be my fourth book, something I never thought I'd say back in seventh grade, when I discovered I actually rather enjoyed writing those silly five paragraph essays. I never thought that I'd be the type of person to sit down and actually write things.

Hell, I never thought I'd be the type of person to sit down and actually finish\ldots{}well, anything. It's the type of thing that continually feels out of reach for me, someone who is up to her neck in stalled projects and who justifies them with phrases like ``the process is the art''.

That said, I can't stop. I can't not make more things. I can't not write. If I have to write, and if, implicit in that need to write is a need for my writing to be read, then yes, you are something to be bought and sold. I am. We are.

\begin{ally}
See? Not my department.
\end{ally}
\newpage

\label{ally:32}
\begin{ally}
My turn.
\end{ally}
Shoot.

\begin{ally}
You said: ``you are not the project, but there is no project without you.''\footnote{Ibid.}\index{ally!throwing stones}
\end{ally}
Yes, that applies to us both.

\begin{ally}
You have spoken to your compulsive need to overshare, and you have spoken to the fact that the act of writing and selling a book is, in its own way, the act of selling yourself. \textbf{Restless Town} and \textbf{Eigengrau} are not so firmly tied to you as this, however. \textbf{Rum and Coke} certainly is not. I don't think you could say the same about this. Speak to your ties to this project.
\end{ally}
Do you suspect that it is too personal to sell?

\begin{ally}
You asked my feelings on the matter.
\end{ally}
I'm of two minds on the matter.

\begin{ally}
Har har.
\end{ally}
Thank you. Seriously, though, I can see two different sides of this.

I feel like I'm putting my maddest edges\index{Mental health}, as Jon Ronson puts it, on display. In the process of working on this project, I was forced to confront some of the most difficult aspects of my life by its very nature.

In the process of pulling the book together, I was forced to reread much of what I had written, and there are parts of it where my words burn too hot, or get too slippery to hold. There's a feverish quality to them. It's something that felt good to write purely for the sensation of it bursting forth from me in uncontrollable torrents.

These maddest edges are something that are integral to the project. You, after all, are one of the, and this project is named after you.

\begin{ally}
Is it mad to have a six month long therapy session with an imaginary interlocutor?
\end{ally}
This is both more and less than that, and you know it.

\begin{ally}
Yes.
\end{ally}
It would be `mad', I suppose, were I to believe that you were an \emph{actual} interlocutor. It would be `mad' were I to present these things as a universal worldview. It would be `mad', awful as that word is, were I anything but deliberate with this project.

As it is, I summoned you. I started pulling down bits of nostalgia\index{Nostalgia} when my I was shutting down my Dreamhost account, when I went to lock my ancient LiveJournal. I got the idea to write, so I did. It was a deliberate effort.

\begin{ally}
Is that mad?
\end{ally}
\ldots{}huh.

\begin{ally}
A question for another time. Tell me of your two minds.
\end{ally}
Right.

On the one hand, I read back through all of this and I find myself tasting blood. Who is this Madison? Is she okay? She seems to be having a rough time of things sometimes, and at others she doesn't seem wholly sane, or at least not wholly healthy. That's a scary thing for someone to put on display. What could possibly lead someone to do that? Some strange form of self-flagellation?

And on the other, while I'm most certainly not wholly healthy, I am, at my core, a storyteller. A young one, and certainly one of uneven quality, but I'm learning and improving by doing. There are stories to be told here, with my life, and that's what I'm doing. I'm making them interesting. I'm embellishing some of them. Hell, I'm making some stuff up wholesale. And I'm doing all of this for the specific purpose of it being read as a story.

In the end, it's the storyteller that wins out over the concerned, private individual. If I can't \emph{not} overshare, if I must compulsively tell stories, then I'm going to tell the stories I have and I'm going to make them worth reading.

\begin{ally}
A friend once asked Maddy, ``Why do you shout carefully constructed, thoroughly edited, well rehearsed speeches into the void?''\index{Koan}
\end{ally}
Maddy replied, ``It pays the bills.''
\index{ally|)}
\newpage
  \end{leftcolumn}
\end{paracol}
\resetbackgroundcolor
\renewfontfamily\pagenumfont{Gentium Book Basic}[Color=000000FF]
