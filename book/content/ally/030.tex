\label{ally:30}
\begin{paracol}{2}
  \begin{leftcolumn}
\noindent Does death take more than one form? Can it be anything other than it is? Can it sneak up on you while you aren't looking, and then when next you take a breath, you realize that you are in some afterlife?

\begin{ally}
I suppose it must, given this lead in.
\end{ally}
Have I died? Has some part of me already rotted and sloughed off? Is this, in some very literal way, an afterlife?

\begin{ally}
Do you feel as though, another seven years having passed, you are moving on from the life that you built up?
\end{ally}
Yes.

\begin{ally}
Then I see no reason not to label it as such.
\end{ally}
Perhaps lorxus was right. Perhaps I am writing this at the end of a life.

\begin{ally}
What are you leaving behind?
\end{ally}
I think I'm leaving behind that bit of me who was struggling to live earnestly.

\begin{ally}
Are you not, now?
\end{ally}
No, I think I am. Or, well, I think I am living fairly earnestly. I think what has happened over the last few years is that the struggle changed its shape.

The Madison who was struggling to come to terms with a post-Matthew life is not me any longer. She spent the last seven years mourning him, in a way. She spent the last seven years figuring out how to live without him, throwing away his stuff, leaving behind family and homes and states.

\begin{ally}
Is this her memoir? Or yours?
\end{ally}
I don't know, honestly.

All I can say is that, for some reason, at some point while working on this project, I might have died. I have entered a liminal space once again. It's a different one, to be sure, but it's somewhere in between who I was and some undefinable potential self.

Perhaps some early whiff of this liminality is what got to start this project in the first place, to summon you. Perhaps it was burnout reaching a head that signaled the death of that version of me. 
\newpage
  \end{leftcolumn}
\end{paracol}
