\label{ally:31}
\begin{paracol}{2}
  \begin{leftcolumn}
\noindent I've been pulling this all into a book. Like, a physical one. A paperback.

\begin{ally}
I know.
\end{ally}
How do you feel about that?

\begin{ally}
Not my department.
\end{ally}
That feels like an evasion to me. You had opinions on me streaming the process of writing. You had opinions on the process itself: you called me out on writing stuff in commit messages, on having our conversations in comments in the source code. You had opinions on me buying the domain name. Do you have no opinions about our words on something to be bought and sold?

\begin{ally}
A friend once asked Maddy, ``Why do you shout into the void?''
\end{ally}
I write all of this down because the very act of putting it into words brings a sense of clarity that I lack without. By taking these moments of my life and articulating them, I almost automatically get another viewpoint on them.

\begin{ally}
And by articulating them as a conversation, you get two. That is not the friend's question.
\end{ally}
No, I suppose it isn't.

I write for the clarity, but I share out of some perverse need. \emph{The chances that ally will pick up any sizeable audience are slim, so I almost feel like I'm publishing it as an extension of my compulsive need to overshare,} I wrote. I share because I have to.

\begin{ally}
Does Maddy shout into the void because she must shout into the void?
\end{ally}
Perhaps. Sometimes.

Sometimes I have to speak so that someone will hear me out of some desire for feeling justified. I need to be heard, to be seen, so that even if I'm going through something alone, others will know that I am doing so. It's being witnessed.

\begin{ally}
There is power in the word, as you say, but there is also power in the act of speaking it. There is no value-judgment for me or anyone else to make in that. Words have power, speaking has power.
\end{ally}
There is value-judgment in the content, though.

\begin{ally}
Yes. There is value-judgment in intent, as well. That you are publishing these words is not something that I \textbf{can} have an opinion about.
\end{ally}
What about my intent?

\begin{ally}
Your compulsive need to do overshare is an implicit part of our relationship.
\end{ally}
Shall I throw your words in your face?

\begin{ally}
By all means.
\end{ally}
``Am I something to be bought and sold? Am I something to be traded and marketed?''\footnote{https://ally.id/aside/2}

\begin{ally}
Have you answered the question? \textbf{Am} I something to be bought and sold? Me, here, being a part of yourself.
\end{ally}
Since having that conversation, I've released two books, and yes, I suppose you are. I am. We are a brand to be built up and marketed, parceled up and sold to any comer.

``The tragic core to all this,'' I wrote, ``is that I'm not an interesting person.'' I \emph{am} a writer, though. This will be my fourth book, something I never thought I'd say back in seventh grade, when I discovered I actually rather enjoyed writing those silly five paragraph essays. I never thought that I'd be the type of person to sit down and actually write things.

Hell, I never thought I'd be the type of person to sit down and actually finish\ldots{}well, anything. It's the type of thing that continually feels out of reach for me, someone who is up to her neck in stalled projects and who justifies them with phrases like ``the process is the art''.

That said, I can't stop. I can't not make more things. I can't not write. If I have to write, and if, implicit in that need to write is a need for my writing to be read, then yes, you are something to be bought and sold. I am. We are.

\begin{ally}
See? Not my department.
\end{ally}
  \end{leftcolumn}
\end{paracol}
