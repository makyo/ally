Shortly after we learned that Margaras died--

\begin{quote}
Less than twenty-four hours. That's pretty short.
\end{quote}

--I wound up in Montreal on the first of many work `sprints'. These were to become a common fixture for the next six years. After all, working from home only gets you so far. Gotta get together, actually learn how the others on your team work. Meet.

\begin{quote}
You had just started at Canonical. Are you sure that wasn't the death of Matthew? Or maybe it was getting married? Creating Younes?
\end{quote}

Matthew was sick for a while. Can we put it that way? He was struggling to hold on, his time was at an end, he was looking rather pale.

\begin{quote}
He was fading.
\end{quote}

Yes.

\begin{quote}
And Madison faded in in 2014.
\end{quote}

I was a transparent person. I was less than real. I was empty, unable to contain an identity. I was a fetch. I was held together with Blu-Tack and paperclips. I was not myself.

\begin{quote}
Are you now?
\end{quote}

Held together with Blu-Tack? I like to think I'm moderately better put together these days.

\begin{quote}
No, yourself. Are you yourself yet?
\end{quote}

Six months after death, remember?

\begin{quote}
Fair. What did you do during your two years as a half-entity?
\end{quote}

Failed. Like, a lot. I failed like it was my job. I failed friends when we moved to Loveland and effectively disappeared from their lives. I failed work when I burned so hard that I burnt out. I failed at communicating. I failed in a lot of ways.

I drank, too. I stopped composing.

\begin{quote}
Was it so negative a time?
\end{quote}

No, of course not. I'm still here. A lot of that failure was the valuable sort. I failed my years at university when I stopped composing, but found that I could still be creative when writing. I failed work when I burned out, but I also learned how to pace myself better (something I definitely hadn't learned up until that point). I learned how to talk, how to listen. At least, how to listen better, how to express myself better.

There's a lot of folks to whom I could credit those being successful failures, if there is such a thing. In a round about way, my boss from the job prior kicking my ass and making me go to therapy, even if not to the ideal therapist, set me on the path to learning how to slow down when I needed to and speed up when that was called for. Writing got me better at putting my ideas --- and, at times, emotions --- into words. Friends, countless friends, helped me become who I am.

\begin{quote}
What's that I'm tasting? Sweet'n Low?
\end{quote}

Is it really that saccharine to be able to look back and say that you sucked, and that you're getting better?

She wears a pendant of stamped brass Saying ``Non sum qualis eram.''

Like, obviously, it sucks to get that regretrospect feeling of looking back and realizing that you were a terrible person, but it's also a good sign that you've improved. If you don't like who you were, at least it's good that you're not that, now.

\begin{quote}
Unless you don't like who you are now.
\end{quote}

That's a different problem. Same class of problem, maybe, but a different problem.
