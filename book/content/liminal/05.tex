\begin{quote}
And so when was Madison born?
\end{quote}

On, September 2, 2014, I got this email:

\begin{verbatim}
I recently discovered your Twitter page and I wasn't sure if I should say something or not.  When I saw that you are stressing out about telling me about your name change I thought I'd better 'fess up.  

I love the name "Madison".  It may take me a while to get used to calling you by your new name so forgive me if I make a mistake.  Madison, whatever direction your life takes you, I'll accept you, support you and love you unconditionally.  Please don't stress out about my reaction.

See you Friday.

Hugs,
Mom
\end{verbatim}

And, two days later:

\begin{verbatim}
Hey Madison,

Maybe I shouldn't have opened up to you about seeing your Twitter thingy.  I felt like I was being dishonest by not saying anything but it looks like you are really, really anxious about knowing that I've seen it.  Yikes!  

Are you OK with me visiting tomorrow?  I'd love to see you but I don't want to add to your anxiety any more than I already have.  Let me know if you have enough spoons.

Love,
Mom
\end{verbatim}

\begin{quote}
Did you not want her to come up?
\end{quote}

No, I did. I told her:

\begin{verbatim}
Mom,

I'm anxious, but please come up tomorrow. I think I need that more than anything right now.

~M
\end{verbatim}

That's when I was born. September 4, 2014 at 3:18 PM. Madison Scott-Clary, 230 pounds, 73 inches.

\begin{quote}
You were born when you could own yourself.
\end{quote}

Yes. I was born when I could share that with my mom. It was all well and good for me to be out on Twitter and what not, and it was great that JD could accept me, but the fact that I could start to regain my biological family without any lies in the way was when I opened my eyes for the first time.

\begin{quote}
How was the visit?
\end{quote}

I don't know. I don't remember. I think it was fine. We talked about me starting hormones--

\begin{quote}
Did you talk about TIASAP?
\end{quote}

\emph{No.}

No, we did not. If she's reading this, which she may very well be, this will be how she learns about that.

How could I possibly talk to my mom about something like that? I hid my arms and legs from her for years before, and it wouldn't be for another year before I could even bring up the concept of self-harm.

\begin{quote}
That's not true.
\end{quote}

I\ldots{}well, no, it's not.

\begin{quote}
Let's talk about suicide.
\end{quote}

Not yet.

Please.

\begin{quote}
Why not?
\end{quote}

I'd like it to be a cohesive thing. I'd like to be able to think about it on its own, none of this coming at it sideways. I'd like to be deliberate about it.

\begin{quote}
Soon.
\end{quote}

Yes, soon.
